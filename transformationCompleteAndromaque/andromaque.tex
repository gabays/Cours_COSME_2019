\documentclass{book}
\usepackage[french]{babel}
\usepackage{hyperref}
\usepackage{glossaries}
\usepackage[noend,series={A},noeledsec, noledgroup]{reledmac}
\usepackage{reledpar}
\usepackage{libertinus}
\usepackage[paperwidth=10cm, paperheight=15cm]{geometry}

% Pas de numéro pour les chapitres, vers etc
\setcounter{secnumdepth}{-2}

% Réglage de la poésie
\setstanzaindents{2,0}
\setcounter{stanzaindentsrepetition}{1}
\sethangingsymbol{[\,}
\newcommand{\antilabe}{\skipnumbering\unskip\hspace{2\stanzaindentbase}}

% Annonce des personnages
\newcommand{\personscene}[1]{\par\hspace{2\stanzaindentbase}\emph{#1}}
\newcommand{\enonciateur}[1]{\par\hspace{\stanzaindentbase}\textbf{#1}}

% Apparat textuel (positif?)

% #1 Le lemme tel que dans le texte
% #2 Le ms concerné
% #3 ce qu'il y a a droite (les variantes)
\newcommand{\app}[3]{\edtext{#1}{%
    \lemma{#1 \wit{#2}}%
    \Afootnote{#3}%
  }%
}
\newcommand{\wit}[1]{(#1)}

% Pas de flag pour les lignes de droites
\setRlineflag{}

\renewcommand{\pagelinesep}{.}
\glstoctrue
\makeglossaries 
\longnewglossaryentry{Achille}{name=Achille}{
    Héros légendaire de la guerre de Troie, fils de Pélée, roi de
 Phthie en Thessalie, et de Thétis, une Néréide (nymphe marine).
    Référence: \url{-->}
}
    \longnewglossaryentry{Agamemnon}{name=Agamemnon}{
    Héros grec et roi de Mycènes. Marié à Clytemnestre, ils ont trois
 filles, Iphigénie, Chrysothémis et Électre/Laodicé, ainsi qu'un fils, Oreste.
 Il assume le commandement de l'armée achéenne durant la guerre de Troie.
    Référence: \url{-->}
}
    \longnewglossaryentry{Aristote}{name=Aristote}{
    Philosophe grec de l'Antiquité. Avec Platon, dont il fut le
 disciple à l'Académie, il est l'un des penseurs les plus influents que le monde
 ait connus.
    Référence: \url{-->}
}
    \longnewglossaryentry{Astyanax}{name=Astyanax}{
    Fils d'Hector et d'Andromaque, et par conséquent le petit-fils de
 Priam, roi de Troie.
    Référence: \url{-->}
}
    \longnewglossaryentry{Cassandre}{name=Cassandre}{
    Dans la mythologie grecque, Cassandre est la fille de Priam (roi
 de Troie) et d'Hécube.
    Référence: \url{-->}
}
    \longnewglossaryentry{Euripide}{name=Euripide}{
    Un des trois grands tragiques de l'Athènes classique, avec
 Eschyle et Sophocle.
    Référence: \url{-->}
}
    \longnewglossaryentry{Hector}{name=Hector}{
    Héros troyen de la guerre de Troie. Fils du roi Priam et de la
 reine Hécube, il est tué par Achille qui veut venger la mort de
 Patrocle.
    Référence: \url{-->}
}
    \longnewglossaryentry{Hécube}{name=Hécube}{
    Dans la mythologie grecque, Hécube est l'épouse de Priam et la
 reine de Troie.
    Référence: \url{-->}
}
    \longnewglossaryentry{Horace}{name=Horace}{
    Poète latin né à Vénose dans le sud de l'Italie, le 8 décembre 65
 av. J.-C. et mort à Rome le 27 novembre 8 av. J.-C.
    Référence: \url{-->}
}
    \longnewglossaryentry{Hélène}{name=Hélène}{
    Dans la mythologie grecque, Hélène est la fille de Zeus et de
 Léda. Elle est mariée à Ménélas, roi de Sparte, avant d'être enlevée par Pâris,
 prince troyen — cet événement déclenchant la guerre de Troie.
    Référence: \url{-->}
}
    \longnewglossaryentry{Ménélas}{name=Ménélas}{
    Personnage de la mythologie grecque, roi de Sparte, fils d'Atrée
 et d'Érope. Mari d'Hélène et frère d'Agamemnon, il est l'un des héros achéens
 de la guerre de Troie.
    Référence: \url{-->}
}
    \longnewglossaryentry{Polyxène}{name=Polyxène}{
    Fille de Priam et d'Hécube, est une princesse troyenne.
    Référence: \url{-->}
}
    \longnewglossaryentry{Priam}{name=Priam}{
    Dans la mythologie grecque, Priam est le roi mythique de Troie au
 moment de la guerre de Troie. Il est fils de Laomédon et de la nymphe Strymo ou
 de Zeuxippe et a pour épouse Hécube.
    Référence: \url{-->}
}
    \longnewglossaryentry{Sénèque}{name=Sénèque}{
    Philosophe de l'école stoïcienne, un dramaturge et un homme
 d'État romain du ier siècle.
    Référence: \url{-->}
}
    \longnewglossaryentry{Ulysse}{name=Ulysse}{
    Roi d'Ithaque, fils de Laërte et d'Anticlée, il est marié à
 Pénélope dont il a un fils, Télémaque.
    Référence: \url{-->}
}
    \longnewglossaryentry{Virgile}{name=Virgile}{
    Poète latin contemporain de la fin de la République romaine et du
 début du règne de l'empereur Auguste.
    Référence: \url{-->}
}
    
\begin{document}

\tableofcontents
\begin{pages}
\begin{Leftside}
\beginnumbering\stanza[\chapter{ACTE PREMIER.}
\section{SCENE PREMIERE.} 
    \personscene{ORESTE, PYLADE.}  
    \enonciateur{ORESTE.} 
    ]
    
    O VY, puis que ie retrouue vn Amy ſi fidelle,&
       Ma Fortune va prendre vne face nou-uelle;&
       Et déja ſon courroux ſemble s'eſtre adouci,&
       Depuis qu'elle a pris ſoin de nous
 rejoindre ici.&
       Qui m'euſt
 dit,
        \app%
        {Qui m'euſst dit,}%
        {1668a}%
        {Qui l’euſst dit? \wit{1697}} qu'vn riuage à mes vœux ſi funeſte,&
       Préſenteroit d'abord Pylade aux yeux d'Oreſte,&
       Qu'apres plus de ſix mois que ie t'auois perdu,&
       A la Cour de Pyrrhus tu me ſerois rendu!\&
       
\stanza[
\enonciateur{PYLADE.}
]
                
                I'en rends graces au Ciel, qui m'arreſtant ſans ceſſe,&
       Sembloit m'auoir fermé le chemin de
 la Gréce,&
       Depuis le jour fatal que la fureur des Eaux,&
       Preſque aux yeux de Mycéne, écarta nos Vaiſſeaux.
        \app%
        {Mycéne}%
        {1668a}%
        {l’Epire \wit{1675 1687 1697}}&
       Combien dans cét exil ay-je ſouffert
 d'allarmes?&
       Combien à vos malheurs ay-je donné de larmes?&
       Craignant toûjours pour vous quelque nouueau danger&
       Que ma triſte Amitié ne pouuoit partager.&
       Sur tout ie redoutois cette
 Mélancolie&
       Où j'ay veu ſi long-temps voſtre Ame enſeuelie.&
       Ie craignois que le Ciel, par vn cruel ſecours,&
       Ne vous offrît la mort, que vous cherchiez
 toûjours.&
       Mais ie vous voy, Seigneur, \ampersand\ ſi j'oſe le dire,&
       Vn Deſtin plus
 heureux vous conduit en Epire.&
       Le pompeux Appareil qui ſuit icy vos
 pas,&
       N'eſt point d'vn
 Malheureux qui cherche le trépas.\&
       
\stanza[
\enonciateur{ORESTE.}
]
                
                Helas! qui peut ſçavoir le Deſtin qui m'ameine?&
       L'Amour me fait icy chercher vne
 Inhumaine.&
       Mais qui ſçait ce qu'il doit
 ordonner de mon Sort,&
       Et ſi ie viens
 chercher, ou la vie, ou la mort?\&
       
\stanza[
\enonciateur{PYLADE.}
]
                
                Quoy! voſtre Ame à l'Amour, en Eſclaue aſſeruie,&
       Se repoſe ſur luy
 du ſoin de voſtre vie?&
       Par quels charmes, apres
 tant de tourmens ſoufferts
        \app%
        {apres tant}%
        {1668a}%
        {oubliant \wit{1697}}&
       Peut-il vous inuiter à rentrer dans ſes fers?&
       Penſez-vous qu'Hermionne, à Sparte inéxorable,&
       Vous prépare en Epirevn Sort plus fauorable?&
       Honteux d'auoir pouſsé tant de vœux ſuperflus,&
       Vous l'abhorriez. Enfin, vous ne m'en parliez
 plus.&
       Vous me trompiez, Seigneur.\&
       
\stanza[
\enonciateur{ORESTE.}
]
                
                \antilabe Ie me trompois moy-méme.&
       Amy, n'inſulte point vn Malheureux
 qui t'aime.&
       T'ay-je iamais caché mon cœur \ampersand\ mes deſirs?&
       Tu vis naiſtre ma flâme \ampersand\ mes
 premiers ſoûpirs.&
       Enfin, quand Menelas diſpoſa de ſa Fille&
       En faueur de Pyrrhus, vangeur de ſa
 Famille;&
       Tu vis mon deſeſpoir, \ampersand\ tu m’as veu depuis&
       Traîner de Mers en Mers ma chaîne \ampersand\ mes
 ennuis.&
       Ie te vis à regret, en cét eſtat funeſte,&
       Preſt à ſuiure par
 tout le déplorable Oreſte,&
       Toûjours de ma fureur interrompre le cours,&
       Et de moy-meſme enfin me ſauuer tous les jours.&
       Mais quand ie me ſouuins, que parmy tant d’al-larmes&
       Hermionne à
 Pyrrhus prodiguoit tous ſes charmes,&
       Tu ſçais de quel courroux mon cœur
 alors épris&
       Voulut, en l’oubliant, vanger tous ſes mépris.&
       Ie fis croire, \ampersand\ ie crûs ma victoire certaine.&
       Ie pris tous mes tranſports pour des tranſports de haine;&
       Déteſtant ſes
 rigueurs, rabaiſſant ſes attraits,&
       Ie défiois ſes yeux
 de me troubler iamais.&
       Voila comme ie crûs étouffer ma
 tendreſſe.&
       Dans ce
 calme trompeur j’arriuay dans la Gréce;&
       Et ie trouuay
 d’abord ſes Princes raſſemblez,&
       Qu’vn péril aſſez grand ſembloit auoir troublez.&
       I’y courus. Ie
 penſay que la Guerre, \ampersand\ la Gloire,&
       De ſoins plus importans rempliroient
 ma memoire;&
       Que mes ſens reprenant leur premiere
 vigueur,&
       L’Amour acheueroit de ſortir de mon Cœur.&
       Mais admire auec moy le Sort, dont
 la pourſuite&
       Me fait courir moy-meſme au piege que j’éuite.&
       I’entens de tous coſtez qu’on menace Pyrrhus.&
       Toute la Gréce éclate en murmures confus.&
       On ſe plaint, qu’oubliant ſon Sang, \ampersand\ ſa promeſſe,&
       Il éleue en ſa Cour
 l’Ennemy de la Gréce,&
       Aſtyanax, d'Hector
 jeune \ampersand\ malheureux Fils,&
       Reſte de tant de Roys ſous Troye enſeuelis.&
       I’apprens, que pour rauir ſon enfance au Suplice,&
       Andromaque
 trompa l’ingénieux Vlyſſe,&
       Tandis qu’vn autre Enfant arraché de
 ſes bras,&
       Sous le nom de ſon Fils, fut conduit au trépas.&
       On dit, que peu ſenſible aux charmes d’Hermionne,&
       Mon Riual porte ailleurs ſon Cœur \ampersand\ ſa Couronne;&
       Ménelas, ſans le croire, en paroiſt affligé,&
       Et ſe plaint d’vn
 Hymen ſi long-temps negligé.&
       Parmy les déplaiſirs où ſon Ame ſe noye,&
       Il s’éleue en la mienne vne ſecrette joye.&
       Ie triomphe; \ampersand\ pourtant ie me flate d’abord&
       Que la ſeule vengeance excite ce
 tranſport.&
       Mais l’Ingrate en mō Cœur
 reprit bientoſt ſa place,&
       De mes feux mal éteints ie reconnus
 la trace,&
       Ie ſentis que ma
 haine alloit finir ſon cours,&
       Ou plûtoſt ie ſentis que ie l’aimois toûjours.&
       Ainſi de tous les Grecs ie brigue le ſuffrage.&
       On m’enuoye à Pyrrhus. I’entreprens ce voyage.&
       Ie viens voir ſi
 l’on peut arracher de ſes bras&
       Cét Enfant, dont la vie allarme tant d’Eſtats.&
       Heureux, ſi ie
 pouuois dans l’ardeur qui me preſſe,&
       Au lieu d’Aſtyanax, luy rauir ma Princeſſe.&
       Car enfin n’attens pas que mes feux redoublez,&
       Des périls les plus grands, puiſſent eſtre troublez.&
       Puis qu’apres tant d’efforts ma reſiſtance eſt vaine,&
       Ie me liure en aueugle au tranſport qui m’entraîne,&
       I’aime, ie viens
 chercher Hermionne en ces lieux,&
       La fléchir, l’enleuer, ou mourir à
 ſes yeux.&
       Toy qui connois Pyrrhus, que penſes-tu qu’il faſſe?&
       Dans ſa Cour, dans ſon Cœur, dy-moy ce qui ſe paſſe.&
       Mon Hermionne
 encor le tient-elle aſſeruy?&
       Me rendra-t'il, Pylade, vn Cœurqu’il m’a rauy?\&
       
\stanza[
\enonciateur{PYLADE.}
]
                
                Ie vous abuſerois, ſi i’oſois vous promettre&
       Qu’entre vos mains, Seigneur, il voulut la
 remettre.&
       Non, que de ſa Conqueſte il paroiſſe flaté.&
       Pour la Veuue d’Hectorſes feux ont éclaté.&
       Il l’aime. Mais enfin cette Veuue
 inhumaine&
       N’a payé jusqu’icy ſon amour que
 de haine,&
       Et chaque jour encore on luy voit tout tenter,&
       Pour fléchir ſa Captive, ou pour
 l’épouuanter.&
       Il luy cache ſon Fils, il menaſſe ſa teſte,&
       Et fait couler des pleurs, qu’auſſi-toſt il arreſte.&
       Hermionne
 elle-meſme a veu plus de cent fois&
       Cet Amant irrité reuenir ſous ſes loix,&
       Et de ſes vœux troublez luy
 rapportant l’hommage,&
       Soûpirer à ſes pieds moins
 d’amour, que de rage.&
       Ainſi n’attendez pas, que l’on
 puiſſe aujourd’huy&
       Vous répondre d’vn Cœur, ſi peu maiſtre de luy.&
       Il peut, Seigneur, il peut dans ce deſordre extré-me,&
       Epouſer ce qu’il hait, \ampersand\
 perdre ce qu’il
 aime.\&
       
\stanza[
\enonciateur{ORESTE.}
]
                
                Mais dy-moy, de quel œil Hermionne peut voir&
       Ses attraits offenſez, \ampersand\ ſes yeux ſans pouuoir.\&
       
\stanza[
\enonciateur{PYLDADE.}
]
                
                Hermionne,
 Seigneur, au moins en apparance,&
       Semble de ſon Amant dédaigner
 l’inconſtance,&
       Et croit que trop heureux d’appaiſerſa rigueur,&
       Il la viendra preſſer de reprendre ſon Cœur.&
       Mais ie l’ay veuë enfin me confier
 ſes larmes.&
       Elle pleure en ſecret le mépris de
 ſes charmes.&
       Toûjours preſte à partir, \ampersand\
 demeurant toûjours,&
       Quelquefois elle appelle Oreſte à ſon ſecours.\&
       
\stanza[
\enonciateur{ORESTE.}
]
                
                Ah! ſi ie le
 croyois, i’irois bientoſt, Pylade,&
       Me jetter....\&
       
\stanza[
\enonciateur{PYLADE.}
]
                
                \antilabe Acheuez, Seigneur, voſtre Ambaſſade.&
       Vous attendez le Roy. Parlez, \ampersand\ luy montrez&
       Contre le Fils d’Hector tous
 les Grecs conjurez.&
       Loin de leur accorder ce Fils de ſa Maiſtreſſe,&
       Leur haine ne fera qu’irriter ſa
 tendreſſe.&
       Plus on les veut broüiller, plus on va les vnir.&
       Preſſez. Demandez
 tout, pour ne rien obtenir.&
       Il vient.\&
       
\stanza[
\enonciateur{ORESTE.}
]
                
                \antilabe Hé bien, va donc diſpoſer la Cruelle&
       A reuoir vn Amant
 qui ne vient que pour elle.\&
       
                      
\stanza[\section{SCENE II.}
\personscene{PYRRHVS, ORESTE, PHOENIX.}
\enonciateur{ORESTE.}
                ]
                
                AVant que tous les Grecs
 vous parlent par ma voix,&
       Souffrez que ie me flate en
 ſecret de leur choix,&
       Et qu’à vos yeux, Seigneur, je montre quelque
 joye&
       De voir le Fils d’Achille, \ampersand\ le Vainqueur de
 Troye.&
       Oüy: Comme ſes exploits, nous
 admirons vos coups;&
       Hector tomba ſous luy; Troye expira ſous vous;&
       Et vous auez montré, par vne heureuſe audace,&
       Que le Fils ſeul d’Achille a pû remplir ſa
 place.&
       Mais ce qu’il n’euſt point fait,
 la Gréce auec douleur&
       Vous voit du Sang Troyen releuer
 le malheur,&
       Et vous laiſſant
 toucher d’vne pitié funeſte,&
       D’vne Guerre ſi
 longue entretenir le reſte.&
       Ne vous ſouuient-il plus, Seigneur, quel fut Hector?&
       Nos peuples affoiblis s’en ſouuiennent encor.&
       Son nom ſeul fait frémir nos Veuues, \ampersand\ nos Filles,&
       Et dans toute la Gréce, il
 n’eſt point de Familles,&
       Qui ne demandent conte à ce malheureux Fils,&
       D’vn Pere, ou d’vn Epoux, qu’Hector leur a rauis.&
       Et qui ſçait ce qu’vn jour ce Fils peut entreprendre?&
       Peut-eſtre dans nos Ports nous le
 verrons deſcẽdre,&
       Tel qu’on a veu ſon Pere embrazer
 nos Vaiſſeaux,&
       Et la flâme à la main, les ſuiure
 ſur les Eaux.&
       Oſeray-je, Seigneur, dire ce que
 ie penſe?&
       Vous-meſme de vos ſoins craignez la recom-penſe,&
       Et que dans voſtre ſein ce Serpent éleué&
       Ne vous puniſſe
 vn jour de l’auoir conſerué.&
       Enfin, de tous les Grecs ſatisfaites l’enuie,&
       Aſſurez leur
 vangeance, aſſurez voſtre vie.&
       Perdez vn Ennemy d’autant plus
 dangereux,&
       Qu’il s’eſſayra
 ſur vous à combattre contre eux.\&
       
\stanza[
\enonciateur{PYRRHVS.}
]
                
                La Gréce en ma faueur eſt trop inquiétée.&
       De ſoins plus importans ie l’ay cruë agitée,&
       Seigneur, \ampersand\ ſur le nom de ſon Ambaſſadeur,&
       I’auois dans ſes projets conceu plus de grandeur.&
       Qui croiroit en effet, qu’vne
 telle entrepriſe&
       Du Fils d’Agamemnon meritaſt l’entremiſe,&
       Qu’vn Peuple tout entier, tant de
 fois triom-phant,&
       N’euſt daigné conſpirer que la mort d’vn Enfant?&
       Mais à qui pretend-on que ie le ſacrifie?&
       La Gréce a-t'elle encor quelque droit ſur ſa vie?&
       Et ſeul de tous les Grecs ne m’eſt-il pas permis&
       D’ordonner des
 Captifs que le Sort m’a ſoûmis?&
       Oüy, Seigneur, lors qu’au pied des murs fumans de
 Troye,&
       Les Vainqueurs tout ſanglans
 partagerẽt leur Proye,&
       Le Sort, dont les Arreſts furent
 alors ſuiuis,&
       Fit tomber en mes mains Andromaque \ampersand\ ſon Fils.&
       Hécube, pres
 d’Vlyſſe, acheua ſa miſere;&
       Caſſandre, dans Argos, a ſuiuy voſtre
 Pere.&
       Sur eux, ſur leurs Captifs, ay-je
 étendu mes droicts?&
       Ay-je enfin diſpoſé du fruit de leurs Exploits?&
       On craint, qu’auec HectorTroyevn jour ne re-naiſſe:&
       Son Fils peut me rauir le jour que
 ie luy laiſſe:&
       Seigneur, tant de prudence entraiſne trop de ſoin.&
       Ie ne ſçay point
 préuoir les malheurs de ſi loin.&
       Ie ſonge quelle
 eſtoit autrefois cette Ville,&
       Si ſuperbe en Rampars, en Héros ſi fertile,&
       Maiſtreſſe de l’Aſie, \ampersand\ je regarde
 enfin&
       Quel fut le Sort de Troye, \ampersand\ quel eſt ſon Deſtin.&
       Ie ne voy que des Tours, que la
 cendre a couuertes,&
       Vn fleuue teint
 de ſang, des Campagnes deſertes,&
       Vn Enfant dans les fers, \ampersand\ je
 ne puis ſonger&
       Que Troye en
 cet eſtat aſpire à ſe
 vanger.&
       Ah! ſi du Fils d’Hector la perte eſtoit
 jurée,&
       Pourquoy d’vn an entier l’auons-nous differée?&
       Dans le ſein de Priam n’a-t'on pû l’immoler?&
       Sous tant de Morts, ſous
 Troye, il falloit l’accabler.&
       Tout eſtoit juſte
 alors. La Vieilleſſe \ampersand\
 l’Enfance&
       En vain ſur leur foibleſſe appuyoient leur defance.&
       La Victoire, \ampersand\ la Nuit, plus cruelles que
 nous,&
       Nous excitoient au meurtre, \ampersand\ confondoient nos
 coups.&
       Mon courroux aux Vaincus ne fut que trop ſeuere.&
       Mais que ma Cruauté ſuruiue à ma Colere?&
       Que malgré la pitié dont ie me ſens ſaiſir,&
       Dans le ſang d’vn
 Enfant ie me baigne à loiſir?&
       Non, Seigneur. Que les Grecs cherchent quelque
 autre Proye,&
       Qu’ils pourſuiuent ailleurs ce qui reſte de Troye,&
       De mes inimitiez le cours eſt
 acheué,&
       L’Epire ſauuera
 ce que Troye a ſauué.\&
       
\stanza[
\enonciateur{ORESTE.}
]
                
                Seigneur, vous ſçauez trop, auec quel artifice&
       Vn faux Aſtianax fut offert au Suplice&
       Où le ſeul Fils d’Hector deuoit eſtre conduit.&
       Ce n’eſt pas les Troyens, c’eſt Hector qu’on
 pour-ſuit.&
       Oüy, les Grecs ſur le Fils perſecutent le Pere.&
       Il a par trop de ſang acheté leur
 colere.&
       Ce n’eſt que dans le ſien qu’elle peut expirer,&
       Et juſques dans l’Epire il les
 peut attirer.&
       Préuenez les.\&
       
\stanza[
\enonciateur{PVRRHVSPYRRHVS}
]
                
                \antilabe Non, non. I’y conſens auec joye.&
       Qu’ils cherchent dans l’Epire vne
 ſeconde Troye.&
       Qu’ils confondent leur haine, \ampersand\ ne diſtinguent plus&
       Le ſang qui les fit vaincre, \ampersand\
 celuy des Vaincus.&
       Auſſi-bien ce
 n’eſt pas la premiere injuſtice,&
       Dont la Gréce, d’Achille a payé le ſeruice.&
       Hector en
 profita, Seigneur, \ampersand\ quelque jour&
       Son Fils en pourroit bien profiter à ſon tour.\&
       
\stanza[
\enonciateur{ORESTE.}
]
                
                Ainſi la Gréce en vous trouue vn Enfant rebelle?\&
       
\stanza[
\enonciateur{PYRRHVS.}
]
                
                Et ie n’ay donc vaincu que pour
 dépendre d’elle?\&
       
\stanza[
\enonciateur{ORESTE.}
]
                
                Hermionne,
 Seigneur, arreſtera vos coups;&
       Ses yeux s’oppoſeront entre ſon Pere \ampersand\ vous.\&
       
\stanza[
\enonciateur{PYRRHVS.}
]
                
                Hermionne,
 Seigneur, peut m’eſtre toûjours chere,&
       Ie puis l’aimer, ſans eſtre Eſclaue de ſon Pere.&
       Et ie ſçauray
 peut-eſtre accorder en ce
 jour&
       Les ſoins de ma grandeur, \ampersand\
 ceux de mon amour.&
       Vous pouuez cependant voir la
 Fille d’Helene.&
       Du ſang qui vous vnit ie ſçay l’étroite
 chaîne.&
       Apres cela, Seigneur, ie ne vous
 retiens plus,&
       Et vous pourrez aux Grecs annoncer mon refus.\&
       
                      
\stanza[\section{SCENE III.}
\personscene{PYRRHVS, PHOENIX.}
\enonciateur{PHOENIX.}
                ]
                
                AInſi vous
 l’enuoyez aux pieds de ſa Maiſtreſſe?\&
       
\stanza[
\enonciateur{PYRRHVS.}
]
                
                On dit qu’il a long-temps brûlé pour la Princeſſe.\&
       
\stanza[
\enonciateur{PHOENIX.}
]
                
                Mais ſi ce feu, Seigneur, vient à
 ſe rallumer,&
       S’il luy rendoit ſon Cœur, s’il
 s’en faiſoit aimer?\&
       
\stanza[
\enonciateur{PYRRHVS.}
]
                
                Ah! qu’ils s’aiment, Phœnix, i’y
 conſens. Qu’elle parte.&
       Que charmez l’vn de l’autre, ils
 retournẽt à Sparte.&
       Tous nos Ports ſont ouuerts \ampersand\ pour elle \ampersand\ pour luy.&
       Qu’elle m’épargneroit de contrainte \ampersand\
 d’ennuy!\&
       
\stanza[
\enonciateur{PHOENIX.}
]
                
                Seigneur....\&
       
\stanza[
\enonciateur{PYRRHVS.}
]
                
                \antilabe Vne autre fois ie
 t’ouuriray mon Ame,&
       Andromaque paroiſt.\&
       
                      
\stanza[\section{SCENE IV.}
\personscene{PYRRHVS, ANDROMAQVE, CEPHISE.}
\enonciateur{PYRRHVS.}
                ]
                
                \antilabe ME cherchiez-vous, Madame?&
       Vn eſpoir ſi charmant me ſeroit-il permis?\&
       
\stanza[
\enonciateur{ANDROMAQVE.}
]
                
                Ie paſſois juſqu’aux lieux, où l’on garde mon
 Fils.&
       Puis qu’vne fois le jour vous ſouffrez que ie voye&
       Le ſeul bien qui me reſte, \ampersand\ d’Hector \ampersand\ de Troye,&
       I’allois, Seigneur, pleurer vn moment auec luy,&
       Ie ne l’ay point encore embraſſé d’aujourd’huy.\&
       
\stanza[
\enonciateur{PYRRHVS.}
]
                
                Ah, Madame! les Grecs, ſi i’en croy leurs allar-mes,&
       Vous donneront bientoſt d’autres
 ſujets de larmes.\&
       
\stanza[
\enonciateur{ANDROMAQVE.}
]
                
                Et quelle eſt cette peur dont leur
 Cœur eſt frappé,&
       Seigneur? Quelque Troyen vous eſt-il échappé?\&
       
\stanza[
\enonciateur{PYRRHVS.}
]
                
                Leur haine pour Hector n’eſt pas encore éteinte.&
       Ils redoutent ſon
 Fils.\&
       
\stanza[
\enonciateur{ANDROMAQVE.}
]
                
                \antilabe Digne Objet de leur crainte!&
       Vn Enfant malheureux, qui ne ſçait pas encor&
       Que Pyrrhus eſt ſon Maiſtre, \ampersand\
 qu’il eſt Fils d’He-ctor.\&
       
\stanza[
\enonciateur{PYRRHVS.}
]
                
                Tel qu’il eſt, tous les Grecs
 demandent qu’il periſſe.&
       Le Fils d’Agamemnon vient haſter ſon ſuplice.\&
       
\stanza[
\enonciateur{ANDROMAQVE.}
]
                
                Et vous prononcerez vn Arreſt ſi cruel?&
       Eſt-ce mon intereſt qui le rend criminel?&
       Helas! on ne craint point qu’il vange vn jour ſon Pere.&
       On craint qu’il n’eſſuyaſt les larmes de ſa
 Mere.&
       Il m’auroit tenu lieu d’vn Pere,
 \ampersand\ d’vn Epoux,&
       Mais il me faut tout perdre, \ampersand\ toûjours par vos
 coups.\&
       
\stanza[
\enonciateur{PYRRHVS.}
]
                
                Madame, mes refus ont préuenu vos larmes.&
       Tous les Grecs m’ont déja menaſſé de leurs Armes.&
       Mais dûſſent-ils
 encore, en repaſſant les Eaux,&
       Demander voſtre Fils, auec mille Vaiſſeaux:&
       Couſtaſt- il tout le ſang
 qu’Helene a fait répandre,&
       Dûſſay-je apres
 dix ans voir mon Palais en cendre,&
       Ie ne balance point, ie vole à ſon ſecours,&
       Ie defendray ſa
 vie aux deſpens de mes jours.&
       Mais parmy ces perils, où ie cours
 pour vous plaire,&
       Me refuſerez-vous vn regard moins ſeuere?&
       Haï de tous les Grecs, preſſé de tous coſtez,&
       Me faudra-t'il combattre encor vos cruautez?&
       Ie vous offre mon Bras. Puis-je
 eſperer encore&
       Que vous accepterez vn Cœur qui
 vous adore?&
       En combattant pour vous, me ſera-t’il permis&
       De ne vous point conter parmy mes Ennemis?\&
       
\stanza[
\enonciateur{ANDROMAQVE.}
]
                
                Seigneur, que faites-vous, \ampersand\ que dira la
 Gréce?&
       Faut-il qu’vn ſi
 grand Cœur montre tant de foi-bleſſe?&
       Voulez-vous qu’vn deſſein ſi beau, ſi genereux,&
       Paſſe pour le
 tranſport d’vn Eſprit amoureux?&
       Captiue, toûjours triſte, importune à moy-méme,&
       Pouuez-vous ſouhaiter qu’Andromaque vous
 aime?&
       Que feriez-vous, helas! d’vn Cœur
 infortuné&
       Qu’à des pleurs éternels vous avez condamné?&
       Non, non, d’vn Ennemy reſpecter la Miſere,&
       Sauuer des Malheureux, rendre vn Fils à ſa Mere,&
       De cent Peuples pour luy combattre la rigueur,&
       Sans me faire payer ſon ſalut de mon Cœur,&
       Malgré moy, s’il le faut, luy donner vn azile,&
       Seigneur, voila des ſoins dignes
 du Fils d’Achille.\&
       
\stanza[
\enonciateur{PYRRHVS.}
]
                
                Hé quoy? Voſtre courroux
 n’a-t’il pas eû ſon cours?&
       Peut-on haïr ſans ceſſe? Et punit-on toûjours?&
       I’ay fait des Malheureux, ſans doute, \ampersand\ la Phrygie&
       Cent fois de voſtre ſang a veu ma main rougie.&
       Mais que vos yeux ſur moy ſe ſont bien exercez!&
       Qu’ils m’ont vendu bien cher les pleurs qu’ils ont
 verſez!&
       De combien de remords m’ont-ils rendu la Proye?&
       Ie souffre tous les maux que i’ay faits deuãt Troye.&
       Vaincu, chargé de fers, de regrets conſumé,&
       Brûlé de plus de feux que ie n’en
 allumé,&
       Tant de ſoins, tant de pleurs,
 tant d’ardeurs in-quiétes....&
       Helas! fus-je iamais ſi cruel que
 vous l’eſtes?&
       Mais enfin, tour à tour, c’eſt aſſez nous punir.&
       Nos Ennemis communs déuroient nous reünir.&
       Madame, dites-moy ſeulement que i’eſpere,&
       Ie vous rens voſtre Fils, \ampersand\ ie luy ſers de
 Pere.&
       Ie l’inſtruiray
 moy-meſme à vanger les Troyens.&
       I’iray punir les Grecs de vos maux
 \ampersand\ des miens.&
       Animé d’vn regard, ie puis tout entreprendre.&
       Voſtre Ilion encor peut ſortir de ſa cendre.&
       Ie puis, en moins de tẽps que les Grecs ne l’ont pris,&
       Dans ſes Murs releuez couronner voſtre Fils.\&
       
\stanza[
\enonciateur{ANDROMAQVE.}
]
                
                Seigneur, tant de grandeurs ne nous touchent plus
 guére,&
       Ie les luy promettois tant qu’a
 veſcu ſon Pere.&
       Non, vous n’eſperez plus de nous
 reuoir encor,&
       Sacrez Murs, que n’a pû conſeruer
 mon Hector.&
       A de moindres faueurs des
 Malheureux prétendent,&
       Seigneur. C’eſt vn Exil que mes pleurs vous de-mandent.&
       Souffrez que loin des Grecs, \ampersand\ meſme loin de vous,&
       I’aille cacher mon Fils, \ampersand\
 pleurer mon Epoux.&
       Voſtre amour contre nous allume
 trop de haine.&
       Retournez, retournez à la Fille d’Helene.\&
       
\stanza[
\enonciateur{PYRRHVS.}
]
                
                Et le puis-je, Madame? Ah, que vous me geſnez!&
       Comment luy rẽdre vn Cœur que vous me retenez?&
       Ie ſçay que de
 mes veux on luy promit l’empire.&
       Ie ſçay que pour
 regner elle vint dans l’Epire.&
       Le Sort vous y voulut l’vne \ampersand\
 l’autre amener,&
       Vous pour porter des fers, Elle pour en donner.&
       Cependant ay-je pris quelque ſoin
 de luy plaire?&
       Et ne diroit-on pas, en voyant au contraire,&
       Vos charmes tout-puiſſans, \ampersand\ les ſiens dédaignez,&
       Qu’elle eſt icy Captiue, \ampersand\ que vous y regnez?&
       Ah! qu’vn ſeul
 des ſoûpirs, que mon Cœur vous enuoye,&
       S’il s’échapoit vers elle, y porteroit de joye!\&
       
\stanza[
\enonciateur{ANDROMAQVE.}
]
                
                Et pourquoy vos ſoûpirs ſeroient-ils repouſſez?&
       Auroit-elle oublié vos ſeruices
 paſſez?&
       Troye,
 Hector, contre vous reuoltent-ils ſon Ame?&
       Aux cendres d’vn Epoux doit-elle
 enfin ſa flâme?&
       Et quel Epoux encore! Ah ſouuenir cruel!&
       Sa mort ſeule a rendu voſtre Pere immortel.&
       Il doit au ſang d’Hector tout l’éclat de ſes
 armes,&
       Et vous n’eſtes tous deux connus
 que par mes larmes.\&
       
\stanza[
\enonciateur{PYRRHVS.}
]
                
                Hé bien, Madame, hé bien, il faut vous obeïr.&
       Il faut vous oublier, ou plûtoſt
 vous haïr.&
       Oüy, mes vœux ont trop loin pouſſé leur violence,&
       Pour ne plus s’arreſter que dans
 l’indifference.&
       Songez-y bien. Il faut deſormais
 que mon Cœur,&
       S’il n’aime auec tranſport, haïſſe auec fureur.&
       Ie n’épargneray rien dans ma juſte colere.&
       Le Fils me répondra des mépris de la Mere,&
       La Gréce le demande, \ampersand\ je ne prétens pas&
       Mettre toûjours ma gloire à ſauuer des Ingrats.\&
       
\stanza[
\enonciateur{ANDROMAQVE.}
]
                
                Helas! il mourra donc. Il n’a pour ſa défenſe,&
       Que les pleurs de ſa Mere, \ampersand\
 que ſon Innocence.&
       Et peut-eſtre apres tout, en l’eſtat où ie ſuis,&
       Sa mort auancera la fin de mes
 ennuis.&
       Ie prolongeois pour luy ma vie,
 \ampersand\ ma miſere.&
       Mais enfin ſur ſes pas j’iray reuoir ſon
 Pere.&
       Ainſi tous trois, Seigneur, par
 vos ſoins reünis,&
       Nos Cœurs…\&
       
\stanza[
\enonciateur{PYRRHVS.}
]
                
                \antilabe Allez, Madame, allez voir voſtre
 Fils.&
       Peut-eſtre, en le voyant, voſtre amour plus timide,&
       Ne prendra pas toûjours ſa Colere
 pour guide.&
       Pour ſçauoir nos
 Deſtins, j’iray vous retrouuer.&
       Madame, en l’embraſſant, ſongez à le ſauuer.\&
       \stanza[\chapter{ACTE II.}
\section{SCENE PREMIERE.} 
    \personscene{HERMIONNE, CLEONNE.}  
    \enonciateur{HERMIONNE.} 
    ]
    
    I E fay ce que tu veux. Ie conſens qu’il me voye.&
       Ie luy veux bien encore accorder
 cette joye;&
       Pylade va
 bien-toſt conduire icy ſes
 pas.&
       Mais, ſi ie m’en
 croyois, ie ne le verrois pas.\&
       
\stanza[
\enonciateur{CLEONNE.}
]
                
                Et qu’eſt-ce que ſa veuë a pour vous de funeſte?&
       Madame, n’eſt-ce pas toûjours le
 meſme Oreſte,&
       Dont vous auez cent fois ſouhaité le retour,&
       Et dont vous regrettiez la conſtance \ampersand\ l’amour?\&
       
\stanza[
\enonciateur{HERMIONNE.}
]
                
                C’eſt cet amour payé de trop
 d’ingratitude,&
       Qui me rend en ces Lieux ſa preſence ſi rude.&
       Quelle honte pour moy! Quel triomphe pour luy,&
       De voir mon infortune égaler ſon
 ennuy!&
       Eſt-ce là, dira-t’il, cette fiere
 Hermionne?&
       Elle me dédaignoit, vn autre
 l’abandonne.&
       L’Ingrate, qui mettoit ſon Cœur à
 ſi haut prix,&
       Apprend donc à ſon tour à ſouffrir des mépris?&
       Ah Dieux!\&
       
\stanza[
\enonciateur{CLEONNE.}
]
                
                \antilabe Ah! diſſipez ces
 indignes allarmes.&
       Il a trop bien ſenty le pouuoir de vos charmes.&
       Vous croyez qu’vn Amant vienne
 vous inſulter?&
       Il vous rapporte vn Cœur qu’il n’a
 pû vous oſter.&
       Mais vous ne dites point ce que vous mande vn Pere.\&
       
\stanza[
\enonciateur{HERMIONNE.}
]
                
                Dans ſes retardemens ſi Pyrrhus perſeuere,&
       A la mort du Troyen s’il ne veut conſentir,&
       Mon Pere auec les Grecs m’ordonne
 de partir.\&
       
\stanza[
\enonciateur{CLEONNE.}
]
                
                Hé bien, Madame, hé bien, écoutez donc Oreſte.&
       Pyrrhus a
 commencé, faites au moins le reſte.&
       Pour bien faire, il faudroit que vous le préuinſſiez.&
       Ne m’auez-vous pas dit que vous le
 haïſſiez?\&
       
\stanza[
\enonciateur{HERMIONNE.}
]
                
                Si ie le hais Cleonne? Il y va de ma gloire,&
       Apres tant de bontez dont il perd la memoire.&
       Luy qui me fut ſi cher, \ampersand\ qui
 m’a pû trahir?&
       Ah! ie l’ay trop aimé, pour ne le
 point haïr.\&
       
\stanza[
\enonciateur{CLEONNE.}
]
                
                Fuyez-le donc, Madame. Et puis qu’on vous
 adore....\&
       
\stanza[
\enonciateur{HERMIONNE.}
]
                
                Ah! laiſſe à ma
 fureur le temps de croiſtre encore.&
       Contre mon Ennemy laiſſe-moy m’aſſurer,&
       Cleonne, auec horreur ie m’en veux ſeparer.&
       Il n’y trauaillera que trop bien,
 l’Infidelle.\&
       
\stanza[
\enonciateur{CLEONNE.}
]
                
                Quoy! vous en attendez quelque injure nouuelle?&
       Aimer vne Captiue, \ampersand\ l’aimer à vos yeux,&
       Tout cela n’a donc pû vous le rendre odieux?&
       Apres ce qu’il a fait, que ſçauroit-il donc faire?&
       Il vous auroit déplû, s’il pouuoit
 vous déplaire.\&
       
\stanza[
\enonciateur{HERMIONNE.}
]
                
                Pourquoy veux-tu, Cruelle, irriter mes ennuis?&
       Ie crains de me connoiſtre, en l’eſtat où ie ſuis.&
       De tout ce que tu vois tâche de ne rien croire.&
       Croy que ie n’aime plus. Vante moy
 ma victoire.&
       Croy que dans ſon dépit mon Cœur
 eſt endurcy.&
       Helas! \ampersand\ s’il ſe peut, fay-le
 moy croire auſſy.&
       Tu veux que ie le fuye. Hé bien,
 rien ne m’arreſte.&
       Allons. N’enuions plus ſon indigne
 conqueſte.&
       Que ſur luy ſa
 Captiue étende ſon pouuoir.&
       Fuyons. Mais ſi l’Ingrat rentroit
 dans ſon de-uoir!&
       Si la Foy dans ſon Cœur retrouuoit quelque place!&
       S’il venoit à mes pieds me demander ſa Grace!&
       Si ſous mes Loix, Amour, tu pouuois l’engager!&
       S’il vouloit!… Mais l’Ingrat ne veut que m’outrager.&
       Demeurons toutefois, pour troubler leur
 fortune.&
       Prenons quelque plaiſir à leur eſtre importune.&
       Ou le forçant de rompre vn nœud ſi ſolemnel,&
       Aux yeux de tous les Grecs rendons-le criminel.&
       I’ay déja ſur le
 Fils attiré leur colere.&
       Ie veux qu’on viẽne encor luy demander la Mere.&
       Rendons-luy les tourmens qu’elle me fait ſouffrir.&
       Qu’elle le perde, ou bien qu’il la faſſe périr.\&
       
\stanza[
\enonciateur{CLEONNE.}
]
                
                Penſez-vous que des yeux toûjours
 ouuerts aux larmes,&
       Songent à balancer le pouuoir de
 vos charmes?&
       Et qu’vn Cœur accablé de tant de
 déplaiſirs,&
       De ſon Perſecuteur ait brigué les ſoûpirs?&
       Voyez ſi ſa
 douleur en paroiſt ſoulagée.&
       Pourquoy dõ les chagrins
 où ſon Ame eſt plõgée?&
       Pourquoy tant de froideurs? Pourquoy cette
 fierté?\&
       
\stanza[
\enonciateur{HERMIONNE.}
]
                
                Helas! pour mon malheur ie l’ay
 trop écouté.&
       Ie n’ay point du ſilence affecté le myſtere.&
       Ie croyois ſans
 péril pouuoir eſtre ſincere.&
       Et ſans armer mes yeux d’vn moment de rigueur,&
       Ie n’ay pour luy parler, conſulté que mon Cœur.&
       Et qui ne ſe ſeroit comme moy declarée,&
       Sur la foy d’vne amour ſi ſaintement jurée?&
       Me voyoit-il de l’œil qu’il me voit
 aujourd’huy?&
       Tu t’en ſouuiens
 encor, tout conſpiroit pour luy.&
       Ma Famille vangée, \ampersand\ les Grecs dans la
 joye,&
       Nos Vaiſſeaux
 tout chargez des dépoüilles de Troye,&
       Les Exploits de ſon Pere, effacez
 par les ſiens,&
       Ses feux que ie croyois plus
 ardans que les miens,&
       Mon Cœur, toy-meſme enfin de ſa gloire ébloüye,&
       Auant qu’il me trahiſt, vous m’auez tous trahie.&
       Mais c’en eſt trop, Cleonne, \ampersand\ quel que ſoit
 Pyr-rhus,&
       Hermionne eſt ſenſible, Oreſte a des vertus.&
       Il ſçait aimer du moins, \ampersand\
 meſme ſans qu’on l’aime;&
       Et peut-eſtre il ſçaura ſe faire aimer luy-méme.&
       Allons. Qu’il vienne enfin.\&
       
\stanza[
\enonciateur{CLEONNE.}
]
                
                \antilabe Madame, le voicy.\&
       
\stanza[
\enonciateur{HERMIONNE.}
]
                
                Ah! ie ne croyois pas qu’il fuſt ſi prés d’icy.\&
       
                      
\stanza[\section{SCENE II.}
\personscene{HERNMIONNE, ORESTE, CLEONNE.}
\enonciateur{HERMIONNE.}
                ]
                
                LE croiray-je, Seigneur,
 qu’vn reſte de tendreſſe&
       Ait ſuſpendu les
 ſoins dont vous charge la Gréce?&
       Ou ne dois-je imputer qu’à voſtre
 ſeul deuoir,&
       L’heureux empreſſemẽt qui vous porte à me voir?\&
       
\stanza[
\enonciateur{ORESTE.}
]
                
                Tel eſt de mon amour l’aueuglement funeſte.&
       Vous le ſçauez,
 Madame, \ampersand\ le destin d’Oreſte&
       Eſt de venir ſans
 ceſſe adorer vos attraits,&
       Et de jurer toûjours qu’il n’y viendra iamais.&
       Ie ſçay que vos
 regards vont rouurir mes bleſſures,&
       Que tous mes pas vers vous ſont
 autãt de parjures.&
       Ie le ſçay, i’en rougis. Mais i’atteſte les Dieux,&
       Témoins de la fureur de mes derniers adieux,&
       Que i’ay couru par tout, où ma
 perte certaine&
       Dégageoit mes ſermens, \ampersand\
 finiſſoit ma peine.&
       I’ay mandié la Mort, chez des
 Peuples cruels&
       Qui n’apaiſoient leurs Dieux que
 du ſang des Mor-tels:&
       Ils m’ont fermé leur Temple, \ampersand\ ces Peuples
 barbares&
       De mon ſang prodigué sont deuenus auares.&
       Enfin ie viens à vous, \ampersand\ ie me voy reduit&
       A chercher dans vos yeux vne mort,
 qui me fuit.&
       Mon deſeſpoir
 n’attend que leur indifference,&
       Ils n’ont qu’à m’interdire vn
 reste d’eſperance.&
       Ils n’ont, pour auancer cette mort
 où ie cours,&
       Qu’à me dire vne fois ce qu’ils
 m’ont dit toûjours.&
       Voila depuis vn an le ſeul ſoin qui m’anime.&
       Madame, c’eſt à vous de prendre vne Victime,&
       Que les Scythes auroient dérobée à vos coups,&
       Si i’en auois
 trouué d’auſſi
 cruels que Vous.\&
       
\stanza[
\enonciateur{HERMIONNE.}
]
                
                Non, non, ne penſez pas
 qu’Hermionne diſpoſe.&
       D’vn ſang, ſur qui la Gréce aujourd’huy ſe repoſe&
       Mais vous-meſme, eſt-ce ainſi que vous executez&
       Les vœux de tant d’Eſtats que vous
 repreſentez?&
       Faut-il que d’vn tranſport leur Vangeance dépende?&
       Eſt-ce le sang d’Oreſte enfin qu’on vous
 demande?&
       Dégagez-vous des ſoins dont vous
 eſtes chargé.\&
       
\stanza[
\enonciateur{ORESTE.}
]
                
                Les refus de Pyrrhus m’ont aſſez
 dégagé,&
       Madame, il me renuoye, \ampersand\ et quelque autre Puiſſance&
       Luy fait du Fils d’Hector embraſſer la
 défence.\&
       
\stanza[
\enonciateur{HER MIONNHERMIONNE.}
]
                
                L’infidelle!\&
       
\stanza[
\enonciateur{ORESTE.}
]
                
                \antilabe Ainſi donc il ne me reſte rien,&
       Qu’à venir prendre icy la place du Troyen:&
       Nous sõmes Ennemis, luy
 des Grecs, moy le voſtre,&
       Pyrrhus protege
 l’vn, \ampersand\ ie vous liure
 l’autre.\&
       
\stanza[
\enonciateur{HERMIONNE.}
]
                
                Hé quoy? Dans vos chagrins ſans
 raiſon affermy,&
       Vous croirez-vous toûjours, Seigneur, mon En-nemy?&
       Quelle eſt cette rigueur tant de
 fois alleguée?&
       I’ay paſſé dans l’Epire où j’eſtois releguée.&
       Mon Pere l’ordonnoit. Mais qui ſçait ſi depuis,&
       Ie n’ay point en ſecret partagé vos ennuis?&
       Penſez-vous auoir
 ſeul éprouué des allarmes?&
       Que l’Epire iamais n’ait veû
 couler mes larmes?&
       Enfin, qui vous a dit, que malgré mon deuoir,&
       Ie n’ay pas quelquefois ſouhaitté de vous voir?\&
       
\stanza[
\enonciateur{ORESTE.}
]
                
                Souhaitté de me voir? Ah diuine
 Princeſſe....&
       Mais de grace, eſt-ce à moy que ce
 diſcours s’a-dreſſe?&
       Ouurez les yeux. Songez
 qu’Oreſte eſt deuant vous,&
       Oreſteſi long-temps l’objet de leur courroux.\&
       
\stanza[
\enonciateur{HERMIONNE.}
]
                
                Oüy, c’eſt vous dont l’amour naiſſant auec leurs
 charmes,&
       Leur apprit le premier le pouuoir
 de leurs armes,&
       Vous que mille vertus me forçoient d’eſtimer,&
       Vous que i’ay plaint, enfin que ie voudrois aimer.\&
       
\stanza[
\enonciateur{ORESTE.}
]
                
                Ie vous entens. Tel eſt mon partage funeſte.&
       Le Cœur eſt pour Pyrrhus, \ampersand\ les vœux pour Oreſte.\&
       
\stanza[
\enonciateur{HERMIONNE.}
]
                
                Ah! ne ſouhaittez-pasſouhaittez pas le deſtin de Pyrrhus,&
       Ie vous haïrois trop.\&
       
\stanza[
\enonciateur{ORESTE.}
]
                
                \antilabe Vous m’en aimeriez plus.&
       Ah! que vous me verriez d’vn
 regard bien con-traire!&
       Vous me voulez aimer, \ampersand\ ie ne
 puis vous plaire,&
       Et l’Amour ſeul alors ſe faiſant obeïr,&
       Vous m’aimeriez, Madame, en me voulant haïr.&
       O dieux! Tant de reſpects, vne amitié ſi tendre…&
       Que de raiſons pour moy, ſi vous pouuiez m’en-tendre!&
       Vous ſeule pour Pyrrhus diſputez
 aujourd’huy,&
       Peut-eſtre malgré vous, ſans doute malgré luy.&
       Car enfin il vous hait. Son ame ailleurs épriſe&
       N’a plus…\&
       
\stanza[
\enonciateur{HERMIONNE.}
]
                
                \antilabe Qui vous l’a dit. Seigneur, qu’il me mépriſe?&
       Ses regards, ſes diſcours vous l’ont-ils donc appris?&
       Iugez vous que ma veuë inſpire des
 mépris?&
       Qu’elle allume en vn cœur des feux
 ſi peu durables?&
       Peut-eſtre d’autres yeux me ſont plus fauorables.\&
       
\stanza[
\enonciateur{ORESTE.}
]
                
                Pourſuiuez. Il
 eſt beau de m’inſulter ainſi.&
       Cruelle, c’eſt donc moy qui vous
 mépriſe ici.&
       Vos yeux n’ont pas aſſez éprouué ma conſtance.&
       Ie ſuis donc vn témoin de leur peu de puiſſance.&
       Ie les ay mépriſez? Ah. Qu’ils voudroient bien voir&
       Riual, comme moy, mépriſer leur pouuoir.\&
       
\stanza[
\enonciateur{HERMIONNE.}
]
                
                Que m’importe, Seigneur, ſa haine,
 ou ſa tendreſſe?&
       Allez contre vn Rebelle armer
 toute la Gréce.&
       Rapportez-luy le prix de ſa
 rebellion.&
       Qu’on faſſe de
 l’Epire vn ſecond Ilion.&
       Allez. Apres cela, direz-vous que ie l’aime?\&
       
\stanza[
\enonciateur{ORESTE.}
]
                
                Madame, faites plus, \ampersand\ venez-y vous-meſme.&
       Voulez-vous demeurer pour oſtage
 en ces lieux?&
       Venez dans tous les cœurs faire parler vos
 yeux.&
       Faiſons de noſtre
 haine vne commune attaque.\&
       
\stanza[
\enonciateur{HERMIONNE.}
]
                
                Mais, Seigneur, cependant s’il épouſe Andromaque?\&
       
\stanza[
\enonciateur{ORESTE.}
]
                
                Hé Madame!\&
       
\stanza[
\enonciateur{HERMIONNE.}
]
                
                \antilabe Songez quelle honte pour nous,&
       Si d’vne Phrygienne il deuenoit l’Eſpoux.\&
       
\stanza[
\enonciateur{ORESTE.}
]
                
                Et vous le haïſſez? Auoüez-le, Madame,&
       L’Amour n’eſt pas vn feu qu’on renferme en vne ame.&
       Tout nous trahit, la voix, le ſilence, les yeux.&
       Et les feux mal couuerts n’en
 éclatent que mieux.\&
       
\stanza[
\enonciateur{HERMIONNE.}
]
                
                Seigneur, ie le voy bien, voſtre ame préuenuë&
       Répand ſur mes diſcours le venin qui la tuë,&
       Toûjours dans mes raiſons cherche
 quelque dé-tour,&
       Et croit qu’en moy la haine eſt vn effort d’amour.&
       Il faut donc m’expliquer. Vous agirez en ſuite.&
       Vous ſçauez qu’en
 ces lieux mon deuoir m’a con-duite,&
       Mon deuoir m’y retient, \ampersand\ ie n’en puis partir,&
       Que mon Pere, ou Pyrrhus ne m’en faſſe ſortir.&
       Au nom de Menelas allez luy faire entendre,&
       Que l’Ennemy des Grecs ne peut eſtre ſon Gendre.&
       Du Troyen, ou de moy, faites le decider:&
       Qu’il ſonge qui des deux, il veut
 rendre, ou garder.&
       Enfin qu’il me renuoye, ou bien qu’il vous le
 liure.&
       Adieu, s’il y conſent, ie ſuis preſte à vous
 ſuiure.\&
       
                      
\stanza[\section{SCENE III.}
\personscene{ORESTE.}
\enonciateur{}
                ]
                
                OVy, oüy, vous me ſuivrez, n’en doutez nul-lement.&
       Ie vous reſpons
 déja de ſon conſentement.&
       Ie ne crains pas, enfin, que
 Pyrrhus la retienne,&
       Il n’a deuant les yeux que ſa chere Troyenne,&
       Tout autre objet le bleſſe, \ampersand\ peut-eſtre au-jourd’huy&
       Il n’attend qu’vn prétexte à
 l’éloigner de luy.&
       Nous n’auons qu’à parler. C’en eſt fait. Quelle joye&
       D’enleuer à l’Epire vne ſi belle Proye!&
       Sauue tout ce qui reſte, \ampersand\ de Troye, \ampersand\
 d’Hector.&
       Garde ſon Fils, ſa Veuue, \ampersand\ mille autres encor:&
       Epire, c’eſt aſſez qu’Hermionne
 renduë&
       Perde à jamais tes bords, \ampersand\ ton Prince de
 veuë.&
       Mais vn heureux deſtin le conduit en ces lieux.&
       Parlons. A tant d’attraits, Amour, ferme ſes yeux.\&
       
                      
\stanza[\section{SCENE IV.}
\personscene{PYRRHVS, ORESTE, PHOENIX.}
\enonciateur{PYRRHVS.}
                ]
                
                IE vous cherchois,
 Seigneur. Vn peu de violence&
        M’a fait de vos raiſons combattre
 la puiſſance,&
       Ie l’auouë. Et
 depuis que ie vous ay quitté,&
       I’en ay ſenty la
 force, \ampersand\ connu l’équité.&
       I’ay ſongé comme
 vous, qu’à la Grece, à mon Pere,&
       A moy-meſme en vn
 mot ie deuenois contraire,&
       Que ie releuois
 Troye, \ampersand\ rendois imparfait&
       Tout ce qu’a fait Achille, \ampersand\ tout ce que i’ay fait.&
       Ie ne condamne plus vn courroux legitime,&
       Et l’on vous va, Seigneur, liurer voſtre Victime.\&
       
\stanza[
\enonciateur{ORESTE.}
]
                
                Seigneur, par ce conſeil prudent
 \ampersand\ rigoureux,&
       C’eſt acheter la Paix du ſang d’vn Malheureux.\&
       
\stanza[
\enonciateur{PYRRHVS.}
]
                
                Oüy. Mais ie veux, Seigneur, l’aſſurer dauantage.&
       D’vne eternelle Paix Hermionne eſt le gage.&
       Ie l’eſpouſe. Il ſembloit qu’vn
 ſpectacle ſi doux&
       N’attendiſt en ces lieux qu’vn Teſmoin tel que vous.&
       Vous y repréſentez tous les Grecs
 \ampersand\ ſon Pere,&
       Puis qu’en vous Menelas voit reuiure ſon
 Frere.&
       Voyez-la donc. Allez. Dites-luy que demain&
       I’attens, auec la
 Paix, ſon cœur de voſtre Main.\&
       
\stanza[
\enonciateur{ORESTE.}
]
                
                Ah dieux!\&
       
                      
\stanza[\section{SCENE V.}
\personscene{PYRRHVS, PHOENIX.}
\enonciateur{PYRRHUS.}
                ]
                
                \antilabe HE bien, Phœnix, l’Amour eſt-il le
 Maiſtre?&
       Tes yeux refuſent-ils encor de me
 connaiſtre?\&
       
\stanza[
\enonciateur{PHOENIX.}
]
                
                Ah! ie vous reconnois, \ampersand\ ce
 juste courroux&
       Ainſi qu’à tous les Grecs,
 Seigneur, vous rend à vous.&
       Et qui l’auroit penſé, qu’vne ſi noble audace&
       D’vn long abbaiſſement prendroit ſi-toſt la place?&
       Que l’on pût ſi-toſt vaincre vn poiſon ſi charmant?&
       Mais Pyrrhus,
 quand il veut, ſçait vaincre en vn
 moment.&
       Ce n’eſt plus le jouët d’vne flamme ſeruile.&
       C’eſt Pyrrhus. C’est le Fils, \ampersand\ le Riual
 d’Achille,&
       Que la Gloire à la fin rameine ſous ſes lois,&
       Qui triomphe de Troyevne ſeconde fois.\&
       
\stanza[
\enonciateur{PYRRHUS.}
]
                
                Dy plutoſt, qu’aujourd’huy
 commence ma Vi-ctoire.&
       D’aujourd’huy ſeulement ie jouïs de ma gloire,&
       Et mon cœur auſſi
 fier, que tu l’as veû ſoûmis,&
       Croit auoir en l’Amour vaincu
 mille Ennemis.&
       Conſidere, Phœnix, les troubles que j’éuite,&
       Quelle foule de maux l’Amour traiſne à ſa ſuite;&
       Que d’Amis, de deuoirs j’allois ſacrifier;&
       Quels perils.... Vn regard m’euſt tout fait oublier.&
       Tous les Grecs conjurez fondoient ſur vn Rebelle.&
       Ie trouuois du
 plaiſir à me perdre pour Elle.\&
       
\stanza[
\enonciateur{PHOENIX.}
]
                
                Oüy, ie benis, Seigneur,
 l’heureuſe cruauté&
       Qui vous rend....\&
       
\stanza[
\enonciateur{PYRRHUS.}
]
                
                \antilabe Tu l’as veû comme elle m’a traitté.&
       Ie penſois, en
 voyant ſa tendreſſe
 allarmée,&
       Que ſon Fils me la duſt renuoyer deſarmée.&
       I’allois voir le ſuccez de ſes embraſſemens.&
       Ie n’ay trouué
 que pleurs mélez d’emportemens.&
       Sa miſere l’aigrit. Et touſiours plus farouche&
       Cent fois le nom d’Hector eſt ſorti de ſa
 bouche.&
       Vainement à ſon Fils j’aſſurois mon ſecours,&
       C’eſt Hector,
 (diſoit-elle en l’embraſſant toû-jours;)&
       Voila ſes yeux, ſa bouche, \ampersand\ déja ſon audace,&
       C’eſt luy-meſme,
 c’eſt toy cher Eſpoux que j’em-braſſe.&
       Et quelle eſt ſa
 penſée? Attend-elle en ce iour&
       Que ie luy laiſſe vn Fils pour nourrir ſon amour?\&
       
\stanza[
\enonciateur{PHOENIX.}
]
                
                Sans doute. C’eſt le prix que vous
 gardoit l’Ingrate.&
       Mais laiſſez-la, Seigneur.\&
       
\stanza[
\enonciateur{PYRRHUS.}
]
                
                \antilabe Ie voy ce qui la flatte.&
       Sa beauté la raſſure, \ampersand\ malgré mon courroux.&
       L’Orgueilleuſe m’attend encore à
 ſes genoux.&
       Ie la verrois aux miens, Phœnix, d’vn œil trãquile.&
       Elle eſt Veuue
 d’Hector. Et ie ſuis Fils d’Achile.&
       Trop de haine ſepare Andromaque \ampersand\ Pyrrhus.\&
       
\stanza[
\enonciateur{PHOENIX.}
]
                
                Commencez donc, Seigneur, à ne m’en parler
 plus.&
       Allez voir Hermionne, \ampersand\ content de luy plaire,&
       Oubliez à ſes pieds iuſqu’à voſtre colere.&
       Vous-meſme à cét hymen venez la
 diſpoſer?&
       Eſt-ce ſur vn Riual qu’il s’en faut repoſer?&
       Il ne l’aime que trop.\&
       
\stanza[
\enonciateur{PYRRHUS.}
]
                
                \antilabe Crois-tu, ſi ie l’eſpouſe,&
       Qu’Andromaque en ſecret n’en ſera pas jalouſe?\&
       
\stanza[
\enonciateur{PHOENIX.}
]
                
                Quoy touſiours Andromaque occupe voſtre
 eſprit?&
       Que vous importe, ô Dieux! ſa
 joye, ou ſon deſpit?&
       Quel charme malgré vous vers elle vous attire?\&
       
\stanza[
\enonciateur{PYRRHUS.}
]
                
                Non, ie n’ay pas bien dit tout ce
 qu’il luy faut dire.&
       Ma colere à ſes yeux n’a paru qu’à
 demy.&
       Elle ignore à quel point ie ſuis ſon Ennemy.&
       Retournons-y. Ie veux la brauer à ſa veuë,&
       Et donner à ma haine une libre eſtenduë.&
       Vien voir tous ſes attraits,
 Phœnix, humiliez.&
       Allons.\&
       
\stanza[
\enonciateur{PHOENIX.}
]
                
                \antilabe Allez, Seigneur, vous jeter à ſes
 piez.&
       Allez, en luy jurant que voſtre
 ame l’adore,&
       A de nouveaux meſpris l’encourager
 encore.\&
       
\stanza[
\enonciateur{PYRRHUS.}
]
                
                Ie le voy bien, tu crois que preſt à l’excuſer.&
       Mon Cœur court apres elle, \ampersand\ cherche à s’apaiſer.\&
       
\stanza[
\enonciateur{PHOENIX.}
]
                
                Vous aimez, c’eſt aſſez.\&
       
\stanza[
\enonciateur{PYRRHUS.}
]
                
                \antilabe Moy l’aimer? Vne Ingrate,&
       Qui me hait d’autant plus que mon amour la
 flate?&
       Sans Parens, ſans Amis, ſans eſpoir que ſur
 moy.&
       Ie puis perdre ſon Fils, peut-eſtre ie le
 doy.&
       Eſtrangere.... Que dis-je? Eſclaue dans l’Epire,&
       Ie luy donne ſon
 Fils, mon Ame, mon Empire,&
       Et ie ne puis gagner dans ſon perfide Cœur&
       D’autre rang que celuy de ſon
 Perſecuteur?&
       Non, non, ie l’ay juré, ma
 vangeance eſt certaine.&
       Il faut bien vne fois juſtifier ſa haine.&
       I’abandonne ſon
 Fils. Que de pleurs vont couler!&
       De quel nom ſa douleur me
 va-t’elle appeller?&
       Quel ſpectacle pour elle
 aujourd’huy ſe diſpoſe!&
       Elle en mourra, Phœnix, \ampersand\ j’en ſeray la cauſe.&
       C’eſt luy mettre moy-meſme vn poignard dans le ſein.\&
       
\stanza[
\enonciateur{PHOENIX.}
]
                
                Et pourquoy donc en faire éclater le deſſein?&
       Que ne conſultiez-vous tantoſt voſtre foibleſſe?\&
       
\stanza[
\enonciateur{PYRRHUS.}
]
                
                Ie t’entens. Mais excuſe vn reſte de tendreſſe.&
       Crains-tu pour ma colere vn ſi foible combat?&
       D’vn amour qui s’eſteint c’eſt le dernier éclat.&
       Allons. A tes conſeils, Phœnix, ie m’abandonne.&
       Faut-il liurer ſon Fils? Faut-il
 voir Hermionne?\&
       
\stanza[
\enonciateur{PHOENIX.}
]
                
                Oüy, voyez-la, Seigneur, \ampersand\ par des vœux ſoûmis&
       Proteſtez-luy…\&
       
\stanza[
\enonciateur{PYRRHUS.}
]
                
                \antilabe Faiſons tout ce que i’ay promis.\&
       \stanza[\chapter{ACTE III.}
\section{SCENE PREMIERE.} 
    \personscene{ORESTE, PYLADE.}  
    \enonciateur{PYLADE.} 
    ]
    
    M oderez donc, Seigneur, cette fureur extréme.&
       Ie ne vous connoy plus. Vous n’eſtes plus vous-meſme.&
       Souffrez....\&
       
\stanza[
\enonciateur{ORESTE.}
]
                
                \antilabe Non, tes conſeils ne ſont plus de ſaiſon,&
       Pylade, ie ſuis las d’écouter la raiſon.&
       C’eſt traiſner
 trop long-temps ma vie \ampersand\ mon ſup-plice.&
       Il faut que ie l’enleue, ou bien que ie periſſe.&
       Le deſſein en eſt pris, ie le veux acheuer.&
       Oüy ie le veux.\&
       
\stanza[
\enonciateur{PYLADE.}
]
                
                \antilabe Hé bien il la faut enleuer.&
       I’y conſens. Mais
 ſongez cependant où vous eſtes.&
       Que croira-t’on de vous à voir ce que vous
 faites?\&
       
\stanza[
\enonciateur{ORESTE.}
]
                
                Que ſçay-je? De moy-meſme eſtois-je alors le maiſtre?&
       La fureur m’emportoit, \ampersand\ ie
 venois peut-eſtre &
       Menaſſer à la
 fois l’Ingrate \ampersand\ ſon Amant.\&
       
\stanza[
\enonciateur{PYLADE.}
]
                
                Et quel eſtoit le fruit de cet
 emportement?\&
       
\stanza[
\enonciateur{ORESTE.}
]
                
                Et quelle ame, dy-moy, ne ſeroit
 éperduë&
       Du coup dont ma raiſon vient d’eſtre confonduë?&
       Il épouſe, dit-il, Hermionne demain.&
       Il veut pour m’honorer la tenir de ma main.&
       Ah! pluſtoſt
 cette main dans le ſang du Barbare....\&
       
\stanza[
\enonciateur{PYLADE.}
]
                
                Vous l’accuſez, Seigneur, de ce
 deſtin bizare.&
       Cependant tourmenté de ſes propres
 deſſeins,&
       Il eſt peut-eſtre
 à plaindre, autant que ie vous plains.\&
       
\stanza[
\enonciateur{ORESTE.}
]
                
                Non, non, ie le connoy, mon deſeſpoir le flate&
       Sans moy, ſans mon amour il
 dédaignoit l’Ingrate&
       Ses charmes iuſques-là n’auoient pû le toucher.&
       Le Cruel ne la prend que pour me l’arracher.&
       Ah dieux! c’en eſtoit fait.
 Hermionne gagnée&
       Pour iamais de ſa veuë alloit eſtre éloignée.&
       Son cœur entre l’amour \ampersand\ le dépit confus.&
       Pour ſe donner à moy n’attendoit
 qu’vn refus.&
       Ses yeux s’ouuroient, Pylade. Elle écoutoit Oreſte,&
       Luy parloit, le plaignoit. Vn mot
 euſt fait le reſte.\&
       
\stanza[
\enonciateur{PYLADE.}
]
                
                Vous le croyez.\&
       
\stanza[
\enonciateur{ORESTE.}
]
                
                \antilabe Hé quoy? ce courroux enflammé&
       Contre vn Ingrat…\&
       
\stanza[
\enonciateur{PYLADE.}
]
                
                \antilabe Iamais il ne fut plus aimé.&
       Penſez-vous, quand Pyrrhus vous l’auroit accordée,&
       Qu’vn prétexte tout preſt ne l’euſt pas retardée?&
       M’en croirez-vous? Laſſé de ſes trompeurs attraits,&
       Au lieu de l’enleuer, Seigneur, ie la fuirais.&
       Quoy? Voſtre amour ſe veut charger d’vne Furie&
       Qui vous déteſtera, qui toute voſtre vie&
       Regrettant vn hymen tout preſt à s’acheuer,&
       Voudra…\&
       
\stanza[
\enonciateur{ORESTE.}
]
                
                \antilabe C’eſt pour cela que ie
 veux l’enleuer.&
       Tout luy riroit, Pylade, \ampersand\ moy, pour mon partage,&
       Ie n’emporterois donc qu’vne inutile rage?&
       I’irois loin d’elle encor, taſcher de l’oublier?&
       Non, non, à mes tourmens ie veux
 l’aſſocier.&
       C’eſt trop gémir tout ſeul. Ie ſuis las qu’on me
 plaigne.&
       Ie prétens qu’à mon tour
 l’Inhumaine me craigne,&
       Et que ſes Yeux cruels à pleurer
 condannez,&
       Me rendent tous les noms, que ie
 leur ay donnez.\&
       
\stanza[
\enonciateur{PYLADE.}
]
                
                Voila donc le ſuccez qu’aura voſtre Ambaſſade,&
       Oreſte rauiſſeur.\&
       
\stanza[
\enonciateur{ORESTE.}
]
                
                \antilabe Et qu’importe, Pylade?&
       Quand nos Eſtats vangez jouïront
 de mes ſoins,&
       L’Ingrate de mes pleurs jouïra-t’elle moins?&
       Et que me ſeruira que la Gréce
 m’admire&
       Tandis que ie ſeray la fable de l’Epire?&
       Que veux-tu? Mais, s’il faut ne te rien déguiſer,&
       Mon Innocence enfin commence à me peſer.&
       Ie ne ſçay de
 tout temps quelle injuſte Puiſſance&
       Laiſſe le Crime
 en paix, \ampersand\ pourſuit l’Innocence.&
       De quelque part ſur moy que ie tourne les yeux,&
       Ie ne voy que malheurs qui
 condannent les Dieux.&
       Meritons leur courroux, juſtifions
 leur haine,&
       Et que le fruit du Crime en précede la peine.&
       Mais toy, par quelle erreur veux-tu toûjours ſur toy&
       Deſtourner vn
 courroux qui ne cherche que moy?&
       Aſſez \ampersand\ trop
 long-temps mon amitié t’accable.&
       Euite vn Malheureux, abandonne vn Coupable.&
       Cher Pylade,
 croy-moy, mon tourment me ſuffit,&
       Laiſſe-moy des
 perils dont j’attens tout le fruit.&
       Porte aux Grecs cet Enfant que Pyrrhus m’abandonne.&
       Va-t’en.\&
       
\stanza[
\enonciateur{PYLADE.}
]
                
                \antilabe Allons, Seigneur, enleuons Hermionne.&
       Au trauers des perils vn grand Cœur ſe fait iour.&
       Que ne peut l’amitié conduite par l’amour?&
       Allons de tous vos Grecs encourager le zele.&
       Nos vaiſſeaux ſont tous preſts, \ampersand\ le vent nous
 appelle.&
       Ie ſçay de ce
 Palais tous les détours obſcurs.&
       Vous voyez que la Mer en vient battre les Murs.&
       Et cette Nuit ſans peine vne ſecrette voye,&
       Iuſqu’en voſtre
 vaiſſeau conduira voſtre Proye.\&
       
\stanza[
\enonciateur{ORESTE.}
]
                
                I’abuſe, cher
 Amy, de ton trop d’amitié.&
       Mais pardonne à des maux, dont toy ſeul as pitié.&
       Excuſe vn
 Malheureux, qui perd tout ce qu’il aime,&
       Que tout le monde hait, \ampersand\ qui ſe hait luy-meſme.&
       Que ne puis-je à mon tour, dans vn
 ſort plus heu-reux…\&
       
\stanza[
\enonciateur{PYLADE.}
]
                
                Diſſimulez,
 Seigneur, c’eſt tout ce que ie
 veux.&
       Gardez qu’auant le coup voſtre deſſein
 n’éclate.&
       Oubliez iuſque-là qu’Hermionne eſt ingrate.&
       Oubliez voſtre amour. Elle vient,
 ie la voy.\&
       
\stanza[
\enonciateur{ORESTE.}
]
                
                Va-t’en. Répons-moy d’elle, \ampersand\ ie répons de moy.\&
       
                      
\stanza[\section{SCENE II.}
\personscene{HERMIONNE, ORESTE, CLEONNE.}
\enonciateur{ORESTE.}
                ]
                
                HE bien? Mes ſoins vous ont rendu voſtre Conqueſte.&
       I’ay veû Pyrrhus, Madame, \ampersand\ voſtre hymen
 s’apreſte.\&
       
\stanza[
\enonciateur{HERMIONNE.}
]
                
                On le dit. Et de plus, on vient de m’aſſurer,&
       Que vous ne me cherchiez que pour m’y préparer.\&
       
\stanza[
\enonciateur{ORESTE.}
]
                
                Et voſtre ame à ſes vœux ne ſera pas rebelle?\&
       
\stanza[
\enonciateur{HERMIONNE.}
]
                
                Qui l’euſt crû, que Pyrrhus ne fuſt pas
 infidelle?&
       Que ſa flamme attendroit ſi tard pour éclater,&
       Qu’il reuiendroit à moy, quand ie l’allois quitter?&
       Ie veux croire auec vous, qu’il redoute la Grece,&
       Qu’il ſuit ſon
 intereſt plûtoſt que ſa tendreſſe,&
       Que mes yeux ſur voſtre ame eſtoiẽt plus
 abſolus.\&
       
\stanza[
\enonciateur{ORESTE.}
]
                
                Non, Madame, il vous aime, \ampersand\ ie n’en doute plus.&
       Vos yeux ne font-ils pas tout ce qu’ils veulent
 faire?&
       Et vous ne vouliez pas ſans doute
 luy déplaire.\&
       
\stanza[
\enonciateur{HERMIONNE.}
]
                
                Mais que puis-je, Seigneur? On a promis ma foy.&
       Luy rauiray-je vn
 bien, qu’il ne tient pas de moy?&
       L’Amour ne regle pas le ſort d’vne Princeſſe.&
       La gloire d’obeïr eſt tout ce
 qu’on nous laiſſe.&
       Cependant ie partois, \ampersand\ vous
 auez pû voir&
       Combien ie relaſchois pour vous de mon deuoir.\&
       
\stanza[
\enonciateur{ORESTE.}
]
                
                Ah! que vous ſçauiez bien, Cruelle… Mais, Ma-dame,&
       Chacun peut à ſon choix diſpoſer de ſon ame.&
       La voſtre eſtoit
 à vous. I’eſperois. Mais enfin&
       Vous l’auez pû donner ſans me faire vn larcin.&
       Ie vous accuſe
 auſſi, bien moins que la
 Fortune.&
       Et pourquoy vous laſſer d’vne plainte importune?&
       Tel eſt voſtre
 deuoir, ie l’auouë.
 Et le mien&
       Eſt de vous épargner vn ſi triſte entretien.\&
       
                      
\stanza[\section{SCENE III.}
\personscene{HERNMIONNE, CLEONNE.}
\enonciateur{HERMIONNE.}
                ]
                
                ATendois-tu, Cleonne, vn courroux ſi mo-deſte?\&
       
\stanza[
\enonciateur{CLEONNE.}
]
                
                La douleur qui ſe taiſt n’en eſt que plus funeſte.&
       Ie le plains. D’autant plus
 qu’Auteur de ſon ennuy,&
       Le coup qui l’a perdu n’eſt parti
 que de luy.&
       Contez, depuis quel tẽps
 voſtre hymen ſe prépare.&
       Il a parlé, Madame, \ampersand\ Pyrrhusſe déclare.\&
       
\stanza[
\enonciateur{HERMIONNE.}
]
                
                Tu crois que Pyrrhus craint? Et que craint-il encor?&
       Des Peuples, qui dix ans ont fuy deuant Hector?&
       Qui cent fois effrayez de l’abſence d’Achille,&
       Dans leurs vaiſſeaux brulans ont cherché leur azile,&
       Et qu’on verroit encor, ſans
 l’appuy de ſon Fils,&
       Redemander Helene aux Troyens impunis?&
       Non, Cleonne, il
 n’eſt point Ennemy de luy-meſme,&
       Il veut tout ce qu’il fait, \ampersand\ s’il m’eſpouſe, il m’aime.&
       Mais qu’Oreſte à ſon gré m’impute ſes douleurs.&
       N’auons-nous d’entretien que celuy
 de ſes pleurs?&
       Pyrrhus reuient à nous. Hé bien, chere Cleonne,&
       Conçois-tu les tranſports de
 l’heureuſe Hermiõne?&
       Sçais-tu quel eſt Pyrrhus? T’éses-tu fait raconter&
       Le nombre des Exploits… Mais qui les peut
 conter?&
       Intrepide, \ampersand\ par tout ſuiuy de
 la Victoire,&
       Charmant, Fidelle enfin, rien ne mãque à ſa Gloire.&
       Songe....\&
       
\stanza[
\enonciateur{CLEONNE.}
]
                
                \antilabe Diſſimulez. Voſtre Riuale en pleurs,&
       Vient à vos pieds ſans doute
 apporter ſes douleurs.\&
       
\stanza[
\enonciateur{HERMIONNE.}
]
                
                Dieux! ne puis-je à ma joye abandonner mon ame?&
       Sortons. Que luy dirois-je?\&
       
                      
\stanza[\section{SCENE IV.}
\personscene{ANDROMAQVE, HERNMIONNE, CLEONNE, CEPHISE.}
\enonciateur{ANDROMAQVE.}
                ]
                
                \antilabe OV fuyez-vous, Madame?&
       N’eſt-ce point à vos yeux, vn ſpectacle aſſez doux&
       Que la Veuue d’Hector pleurante à vos genoux?&
       Ie ne viens point icy, par de
 jalouſes larmes,&
       Vous enuier vn Cœur, qui ſe rend à vos charmes.&
       Par les mains de ſon Pere, helas!
 i’ay veû percer&
       Le ſeul, où mes regards
 pretendoient s’adreſſer.&
       Ma flamme par Hector fut jadis allumée,&
       Auec luy dans la tombe elle s’eſt
 enfermée.&
       Mais il me reſte vn Fils. Vous ſçaurez quelque iour,&
       Madame, pour vn Fils iuſqu’où va noſtre amour.&
       Mais vous ne ſçaurez pas, du moins
 ie le ſouhaitte,&
       En quel trouble mortel ſon
 intereſt nous jette,&
       Lors que de tant de biens, qui pouuoient nous flatter,&
       C’eſt le ſeul qui
 nous reſte, \ampersand\ qu’on veut nous l’oſter.&
       Helas! Lors que laſſez de dix ans de miſere,&
       Les Troyens en courroux menaçoient voſtre Mere,&
       I’ay ſçeû de mon
 Hector luy procurer l’appuy;&
       Vous pouuez ſur
 Pyrrhus, ce que i’ay pû ſur luy.&
       Que craint-on d’vn Enfant, qui ſuruit à ſa perte?&
       Laiſſez-moy le
 cacher en quelque Iſle deſerte.&
       Sur les ſoins de ſa Mere on peut s’en aſſurer,&
       Et mon Fils auec moy n’aprendra
 qu’à pleurer.\&
       
\stanza[
\enonciateur{HERMIONNE.}
]
                
                Ie conçoy vos douleurs. Mais vn devoir auſtere,&
       Quand mon Pere a parlé, m’ordonne de me taire.&
       C’eſt luy, qui de Pyrrhus fait agir le courroux.&
       S’il faut fléchir Pyrrhus, qui le peut mieux que vous?&
       Vos yeux aſſez
 long-temps ont regné ſur ſon ame.&
       Faites-le prononcer, j’y ſouſcriray, Madame.\&
       
                      
\stanza[\section{SCENE V.}
\personscene{ANDROMAQVE, CEPHIZE.}
\enonciateur{ANDROMAQVE.}
                ]
                
                QVel mépris la Cruelle
 attache à ſes refus!\&
       
\stanza[
\enonciateur{CEPHIZE.}
]
                
                Ie croirois ſes
 conſeils, \ampersand\ ie verrois Pyrrhus.&
       Vn regard confondroit Hermionne \ampersand\ la Gréce..&
       Mais luy-meſme il vous
 cherche.\&
       
                      
\stanza[\section{SCENE VI.}
\personscene{PYRRHVS, ANDROMAQVE, PHOENIX, CEPHIZE.}
\enonciateur{PYRRHVS}
                ]
                
                \antilabe OV donc eſt la
 Princeſſe?&
       Ne m’auois-tu pas dit qu’elle eſtoit en ces lieux?\&
       
\stanza[
\enonciateur{PHOENIX.}
]
                
                Ie le croyois.\&
       
\stanza[
\enonciateur{ANDROMAQVE}
]
                
                \antilabe Tu vois le pouuoir de mes yeux.\&
       
\stanza[
\enonciateur{PYRRHVS.}
]
                
                Que dit-elle, Phœnix?\&
       
\stanza[
\enonciateur{ANDROMAQVE.}
]
                
                \antilabe Helas! tout m’abandonne.\&
       
\stanza[
\enonciateur{PHOENIX.}
]
                
                Allons, Seigneur, marchons ſur les
 pas d’Hermionne.\&
       
\stanza[
\enonciateur{CEPHISE.}
]
                
                Qu’attendez-vous? Forcez ce ſilence obſtiné.\&
       
\stanza[
\enonciateur{ANDROMAQVE.}
]
                
                Il a promis mon Fils.\&
       
\stanza[
\enonciateur{CEPHISE.}
]
                
                \antilabe Il ne l’a pas donné.\&
       
\stanza[
\enonciateur{ANDROMAQVE.}
]
                
                Non, non, i’ay beau pleurer, ſa mort eſt reſoluë.\&
       
\stanza[
\enonciateur{PYRRHVS.}
]
                
                \antilabe Daigne-t’elle ſur nous tourner au moins
 la veuë?&
       Quel orgueil!\&
       
\stanza[
\enonciateur{ANDROMAQVE.}
]
                
                Ie ne fay que l’irriter encor.&
       Sortons.\&
       
\stanza[
\enonciateur{PYRRHVS.}
]
                
                \antilabe Allons aux Grecs liurer le Fils
 d’Hector.\&
       
\stanza[
\enonciateur{ANDROMAQVE.}
]
                
                Ah, Seigneur, arreſtez. Que
 prétendez-vous faire?&
       Si vous liurez le Fils, liurez-leur donc la
 Mere.&
       Vos ſermens m’ont tantoſt iuré tant d’amitié.&
       Dieux! N’en reſte-t’il pas du
 moins quelque pitié?&
       Sans eſpoir de pardon m’auez-vous condamnée?\&
       
\stanza[
\enonciateur{PYRRHVS.}
]
                
                Phœnix vous le
 dira, ma parole eſt donnée.\&
       
\stanza[
\enonciateur{ANDROMAQVE.}
]
                
                Vous qui brauiez pour moy tant de
 perils diuers?\&
       
\stanza[
\enonciateur{PYRRHVS.}
]
                
                I’eſtois aueugle alors, mes yeux ſe ſont ouuers.&
       Sa grace à vos deſirs pouuoit eſtre accordée.&
       Mais vous ne l’auez pas ſeulement demandée.&
       C’en eſt fait.\&
       
\stanza[
\enonciateur{ANDROMAQVE.}
]
                
                \antilabe Ah! Seigneur, vous entendiez aſſez&
       Des ſoupirs, qui craignoient de ſe voir repouſſez.&
       Pardonnez à l’éclat d’vne illuſtre fortune&
       Ce reſte de fierté, qui craint
 d’eſtre importune.&
       Vous ne l’ignorez pas, Andromaqueſans vous&
       N’auroit iamais d’vn Maiſtre embraſſé les
 genoux.\&
       
\stanza[
\enonciateur{PYRRHVS.}
]
                
                Non, vous me haïſſez. Et dans le fonds de l’ame&
       Vous craignez de deuoir quelque
 choſe à ma flâme.&
       Ce Fils meſme, ce Fils, l’objet de
 tant de ſoins,&
       Si je l’auois ſauué, vous l’en aimeriez moins.&
       La haine, le meſpris, contre moy
 tout s’aſſemble.&
       Vous me haïſſez
 plus que tous les Grecs enſemble.&
       Ioüiſſez à loiſir d’vn ſi noble
 courroux.&
       Allons, Phœnix.\&
       
\stanza[
\enonciateur{ANDROMAQVE.}
]
                
                \antilabe Allons rejoindre mon Eſpoux.\&
       
\stanza[
\enonciateur{CEPHISE.}
]
                
                Madame....\&
       
\stanza[
\enonciateur{ANDROMAQVE.}
]
                
                \antilabe Et que veux-tu que je lui diſe
 encore?&
       Auteur de tous mes maux crois-tu qu’il les
 ignore?&
       Seigneur, voyez l’eſtat où vous me
 reduiſez?&
       I’ay veu mon Pere mort, \ampersand\ nos
 Murs embraſez,&
       I’ay veû trancher les iours de ma
 Famille entiere,&
       Et mon Eſpoux ſanglant traiſné ſur la pouſſiere,&
       Son Fils ſeul avec moy reſerué pour les fers.&
       Mais que ne peut vn Fils, ie reſpire, ie ſers.&
       I’ay fait plus. Ie me suis quelquefois conſolée&
       Qu’icy plûtoſt qu’ailleurs le ſort m’euſt exhilée;&
       Qu’heureux dans ſon malheur, le
 Fils de tant de Rois.&
       Puis qu’il deuoit ſeruir, fuſt tombé ſous vos
 lois.&
       I’ay crû que ſa
 Priſon deuiendroit ſon Azile.&
       Iadis Priamſoûmis fut reſpecté
 d’Achile.&
       I’attendois de ſon Fils encor plus de bonté.&
       Pardonne, cher Hector, à ma credulité.&
       Ie n’ay pû ſoupçonner ton Ennemy d’vn crime,&
       Malgré luy-meſme enfin je l’ay crû
 magnanime.&
       Ah! s’il l’eſtoit aſſez, pour nous laiſſer du moins&
       Au Tombeau qu’à ta Cendre ont éleué mes ſoins;&
       Et que finiſſant-là ſa haine \ampersand\ nos miſeres,&
       Il ne ſéparaſt
 point des deſpoüilles ſi cheres!\&
       
\stanza[
\enonciateur{PYRRHVS.}
]
                
                Va m’attendre, Phœnix.\&
       
                      
\stanza[\section{SCENE VII.}
\personscene{PYRRHVS, ANDROMAQVE, CEPHISE.}
\enonciateur{PYRRHVS}
                ]
                
                \antilabe MAdame, demeurez;&
       On peut vous rendre encor ce Fils que vous
 pleurez.&
       Oüy, ie ſens à
 regret, qu’en excitant vos larmes,&
       Ie ne fay contre moy que vous
 donner des armes.&
       Ie croyois apporter plus de haine
 en ces lieux.&
       Mais, Madame, du moins tournez vers moy les
 yeux.&
       Voyez ſi mes regards ſont d’vn Iuge ſeuere,&
       S’ils ſont d’vn
 Ennemy qui cherche à vous déplaire.&
       Pourquoy me forcez-vous
 vous-meſme à vous
 trahir?&
       Au nom de voſtre Fils, ceſſons de nous haïr.&
       A le ſauuer
 enfin, c’eſt moy qui vous conuie.&
       Faut-il que mes ſoûpirs vous
 demandent ſa vie?&
       Faut-il qu’en ſa faueur j’embraſſe vos genoux?&
       Pour la derniere fois, ſauuez-le, ſauuez-vous.&
       Ie ſçay de quels
 ſermens ie romps pour vous les
 chaiſnes,&
       Combien ie vais ſur moy faire éclater de haines.&
       Ie renuoye Hermionne, \ampersand\ ie mets
 ſur ſon front,&
       Au lieu de ma Couronne, vn éternel
 affront.&
       Ie vous conduis au Temple, où ſon Hymen s’ap-preſte.&
       Ie vous ceins du Bandeau, préparé
 pour ſa Teſte.&
       Mais ce n’eſt plus, Madame, vne offre à dédai-gner.&
       Ie vous le dis, il faut ou perir,
 ou regner.&
       Mon cœur, deſeſperé d’vn an d’ingratitude,&
       Ne peut plus de ſon ſort ſouffrir l’incertitude.&
       C’eſt craindre, menaſſer, \ampersand\ gemir trop long-temps.&
       Ie meurs, ſi ie vous pers, mais ie meurs, ſi j’attens.&
       Songez-y, ie vous laiſſe, \ampersand\ ie viendray vous
 prendre,&
       Pour vous mener au Temple, où ce Fils doit
 m’at-tendre.&
       Et là vous me verrez ſoûmis, ou
 furieux,&
       Vous couronner, Madame, ou le perdre à vos
 yeux.\&
       
                      
\stanza[\section{SCENE VIII.}
\personscene{ANDROMAQVE, CEPHISE.}
\enonciateur{CEPHISE.}
                ]
                
                HE bien, ie vous l'ay dit, qu'en deſpit de la 
 Grece&
       De voſtre ſort
 encor vous ſeriez la Maiſtreſſe.\&
       
\stanza[
\enonciateur{ANDROMAQVE.}
]
                
                Helas! De quel effet tes diſcours
 ſont ſuiuis?&
       Il ne me reſtoit plus qu’à
 condamner mon Fils.\&
       
\stanza[
\enonciateur{CEPHISE.}
]
                
                Madame, à voſtre Eſpoux c’eſt eſtre aſſez fidelle.&
       Trop de vertu pourroit vous rendre criminelle.&
       Luy-meſme il porteroit voſtre ame à la douceur.\&
       
\stanza[
\enonciateur{ANDROMAQVE.}
]
                
                Quoy, ie luy donnerois Pyrrhus pour ſucceſſeur?\&
       
\stanza[
\enonciateur{CEPHISE.}
]
                
                Ainſi le veut ſon
 Fils, que les Grecs vous rauiſſent.&
       Penſez-vous qu’aprés tout ſes Manes en rougiſ-ſent?&
       Qu’il meſpriſaſt, Madame, vn Roy victorieux,&
       Qui vous fait remonter au rang de vos Ayeux;&
       Qui foule aux pieds pour vous vos Vainqueurs en
 colere,&
       Qui ne ſe ſouuient plus qu’Achille eſtoit ſon
 Pere,&
       Qui dément ſes Exploits, \ampersand\ les
 rend ſuperflus?\&
       
\stanza[
\enonciateur{ANDROMAQVE.}
]
                
                Dois-je les oublier, s’il ne s’en ſouuient plus?&
       Dois-je oublier Hector priué de funerailles,&
       Et traiſné ſans
 honneur autour de nos murailles?&
       Dois-je oublier ſon Pere à mes
 pieds renuerſé,&
       Enſanglantant l’Autel qu’il
 tenoit embraſſé?&
       Songe, ſonge, Cephiſe, à cette Nuit
 cruelle,&
       Qui fut pour tout vn Peuple vne Nuit éternelle.&
       Figure-toy Pyrrhus les yeux étincelans,&
       Entrant à la lueur de nos Palais brûlans;&
       Sur tous mes Freres morts ſe
 faiſant vn paſſage,&
       Et de ſang tout couuert échauffant le carnage.&
       Songe aux cris des Vainqueurs, ſonge aux cris des Mourans,&
       Dans la flamme étouffez, ſous le
 fer expirans.&
       Peins-toy dans ces horreurs Andromaque eſ-perduë.&
       Voila comme Pyrrhus vint s’offrir à ma veuë,&
       Voila par quels exploits il ſçeût ſe couronner,&
       Enfin voila l’Eſpoux que tu me
 veux donner.&
       Non, ie ne ſeray point complice de ſes crimes.&
       Qu’il nous prenne, s’il veut, pour dernieres
 Victimes.&
       Tous mes reſſentimens luy seroient aſſeruis.\&
       
\stanza[
\enonciateur{CEPHISE.}
]
                
                Hé bien, allons donc voir expirer voſtre Fils.&
       On n’attend plus que vous. Vous fremiſſez, Ma-dame?\&
       
\stanza[
\enonciateur{ANDROMAQVE.}
]
                
                Ah! de quel ſouuenir viens-tu frapper mon ame?&
       Quoy, Cephiſe, j’iray voir expirer encor&
       Ce Fils, ma ſeule joye, \ampersand\
 l’image d’Hector?&
       Ce Fils que de ſa flamme il me
 laiſſa pour gage?&
       Helas! il m’en ſouuient, le iour que ſon courage,&
       Luy fit chercher Achille, ou
 plûtoſt le treſpas;&
       Il demanda ſon Fils, \ampersand\ le
 prit dans ſes bras.&
       Chere Eſpouſe,
 dit-il, en eſſuyant mes larmes,&
       I’ignore quel ſuccez le ſort garde à mes armes,&
       Ie te laiſſe mon Fils, pour gage de ma foy;&
       S’il me perd, ie prétens qu’il
 me retrouue en toy.&
       Si d’vn heureux hymen la memoire
 t’eſt chere,&
       Montre au Fils à quel point tu cheriſſois le Pere.&
       Et ie puis voir reſpandre vn sang ſi
 pretieux?&
       Et ie laiſſe auec luy perir tous ſes Ayeux?&
       Roy barbare, faut-il que mon crime l’entraiſne?&
       Si ie te haïs, eſt-il coupable de ma haine?&
       T’a-t’il de tous les ſiens
 reproché le treſpas?&
       S’eſt-il plaint à tes yeux des
 maux qu’il ne ſent pas?&
       Mais cependant, mon Fils, tu meurs, ſi ie n’arreſte&
       Le fer, que ce Cruel tient leué ſur ta teſte.&
       Ie l’en puis détourner, \ampersand\ ie t’y vais offrir?&
       Non tu ne mourras point, ie ne
 le puis ſouffrir.&
       Allons trouver Pyrrhus. Mais non, chere Cephiſe,&
       Va le trouuer pour
 moy.\&
       
\stanza[
\enonciateur{CEPHISE.}
]
                
                \antilabe Que faut-il que ie diſe?\&
       
\stanza[
\enonciateur{ANDROMAQVE.}
]
                
                Dy-luy que de mon Fils l’amour eſt aſſez fort…&
       Crois-tu que dans ſon cœur il
 ait juré ſa mort?&
       L’amour peut-il ſi loin pousser
 ſa barbarie?\&
       
\stanza[
\enonciateur{CEPHISE.}
]
                
                Madame, il va bien-toſt reuenir en furie.\&
       
\stanza[
\enonciateur{ANDROMAQVE.}
]
                
                Hé bien, va l’aſſurer…\&
       
\stanza[
\enonciateur{CEPHISE.}
]
                
                \antilabe De quoy? de voſtre foy?\&
       
\stanza[
\enonciateur{ANDROMAQVE.}
]
                
                Helas! pour la promettre eſt-elle encore à moy?&
       O cendres d’vn Eſpoux! ô Troyens! ô mon Pere!&
       O mon Fils, que tes iours couſtent cher à ta Mere!&
       Allons.\&
       
\stanza[
\enonciateur{CEPHISE.}
]
                
                \antilabe Où donc, Madame, \ampersand\ que reſoluez-vous?\&
       
\stanza[
\enonciateur{ANDROMAQVE.}
]
                
                Allons, ſur ſon
 tombeau, conſulter mon Eſpoux.\&
       \stanza[\chapter{ACTE IV.}
\section{SCENE I.} 
    \personscene{ANDROMAQVE, CEPHISE.}  
    \enonciateur{CEPHISE.} 
    ]
    
    A h! je n’en doute point. C’eſt voſtre Eſpoux, Madame,&
       C’eſt Hector qui produit ce miracle en voſtre ame.&
       Il veut que Troye encor ſe puiſſe re-leuer,&
       Auec cet heureux Fils, qu’il vous fait conſeruer.&
       Pyrrhus vous
 l’a promis. Vous venez de l’entendre,&
       Madame, il n’attendoit qu’vn mot
 pour vous le rendre.&
       Croyez en ſes tranſports. Pere, Sceptre, Alliez,&
       Content de voſtre cœur, il met
 tout à vos pieds.&
       Sur luy, ſur tout ſon Peuple il vous rend Souue-raine.&
       Eſt-ce là ce Vainqueur digne de
 tant de haine?&
       Déja contre les Grecs plein d’vn
 noble courroux,&
       Le ſoin de voſtre Fils le touche autant que vous;&
       Il preuient leur fureur, il luy
 laiſſe ſa
 Garde,&
       Pour ne pas l’expoſer luy-meſme il ſe hazarde.&
       Mais tout s’apreſte au Temple.
 Et vous auez pro-mis....\&
       
\stanza[
\enonciateur{ANDROMAQVE.}
]
                
                Ouy, je m’y trouuerai. Mais
 allons voir mon Fils.\&
       
\stanza[
\enonciateur{CEPHIZE.}
]
                
                Madame, qui vous preſſe? Il ſuffit que ſa
 veuë&
       Deſormais à vos yeux ne ſoit plus deffenduë.&
       Vous luy pourrez bientoſt
 prodiguer vos bontez,&
       Et vos embraſſemens ne ſeront plus contez.&
       Quel plaiſir! d’éleuer vn Enfant qu’on voit craiſtre,&
       Non plus comme vn Eſclaue éleué pour ſon Maî-tre,&
       Mais pour voir auec luy renaiſtre tant de Rois.\&
       
\stanza[
\enonciateur{ANDROMAQVE.}
]
                
                Cephize, allons le voir pour la derniere
 fois.\&
       
\stanza[
\enonciateur{CEPHIZE.}
]
                
                Que dites-vous? ô Dieux!\&
       
\stanza[
\enonciateur{ANDROMAQVE.}
]
                
                \antilabe O! ma chere Cephize,&
       Ce n’eſt point auec toy que mon cœur ſe déguiſe.&
       Ta foy dans mon malheur s’eſt
 montrée à mes yeux.&
       Mais j’ay crû qu’à mon tour tu me connoiſſois mieux.&
       Quoy donc as-tu penſé
 qu’Andromaque infidelle,&
       Pûſt trahir vn
 Eſpoux qui croit reuiure en
 elle,&
       Et que de tant de Morts réueillant la
 douleur,&
       Le ſoin de mon repos me fiſt troubler le leur?&
       Eſt-ce là cette ardeur tant
 promiſe à ſa cendre.&
       Mais ſon Fils periſſoit, il l’a falu défendre?&
       Pyrrhus en
 m’épouſant s’en déclare l’appuy.&
       Il ſuffit. Ie
 veux bien m’en repoſer ſur luy.&
       Ie ſçay quel
 eſt Pyrrhus.
 Violent, mais ſincere,&
       Cephize, il
 fera plus qu’il n’a promis de faire.&
       Sur le courroux des Grecs, je m’en repoſe encor,&
       Leur haine va donner vn Pere au
 Fils d’Hector.&
       Ie vais donc, puiſqu’il faut que je me ſacrifie,&
       Aſſurer à
 Pyrrhus le reſte
 de ma vie.&
       Ie vais en receuant ſa foy ſur les Autels,&
       L’engager à mon Fils par des nœuds îmmortels.&
       Mais auſſi-toſt ma main, à moy ſeule funeſte,&
       D’vne infidelle vie abbregera le
 reſte,&
       Et ſauuant ma
 vertu, rendra ce que ie doy,&
       A Pyrrhus, à
 mon Fils, à mon Eſpoux, à moy.&
       Voila de mon amour l’innocent ſtratagéme;&
       Voila ce qu’vn Eſpoux m’a commandé luy-méme.&
       I’iray ſeule
 rejoindre Hector, \ampersand\ mes Ayeux.&
       Cephize,
 c’eſt à toy de me fermer les yeux.\&
       
\stanza[
\enonciateur{CEPHIZE.}
]
                
                Ah! ne pretendez pas que ie
 puiſſe ſuruiure.\&
       
\stanza[
\enonciateur{ANDROMAQVE.}
]
                
                Non, non, ie te deffens,
 Cephize, de me ſuiure.&
       Ie confie à tes ſoins mon vnique treſor,&
       Si tu viuois pour moy, vy pour
 le Fils d’Hector.&
       De l’eſpoir des Troyens ſeule dépoſitaire,&
       Songe, à combien de Roys tu deuiens neceſſaire.&
       Veille auprés de Pyrrhus. Fay-luy garder ſa foy.&
       S’il le faut, ie conſens que tu parles de moy.&
       Fais-luy valoir l’hymen, où ie
 me ſuis rangée;&
       Dy-luy, qu’auant ma mort ie luy fus engagée,&
       Que ſes reſſentimens doiuent eſtre effacez,&
       Qu’en luy laiſſant mon Fils, c’eſt l’eſtimer aſſez.&
       Fay connoiſtre à mon Fils les
 Heros de ſa Race;&
       Autant que tu pourras, conduy-le ſur leur trace.&
       Dy-luy, par quels exploits leurs noms ont
 éclaté,&
       Pluſtoſt ce
 qu’ils ont fait, que ce qu’ils ont eſté.&
       Parle luy tous les jours des Vertus de ſon Pere,&
       Et quelquefois auſſi parle luy de ſa Mere.&
       Mais qu’il ne ſonge plus,
 Cephize, à nous vanger&
       Nous luy laiſſons vn Maiſtre, il le doit
 ménager.&
       Qu’il ait de ſes Ayeux vn ſouuenir modeſte,&
       Il eſt du ſang
 d’Hector, mais il en eſt le reſte.&
       Et pour ce reſte enfin i’ay moy-meſme en vn
 jour,&
       Sacrifié mon ſang, ma haine,
 \ampersand\ mon amour.\&
       
\stanza[
\enonciateur{CEPHISE.}
]
                
                Helas!\&
       
\stanza[
\enonciateur{ANDROMAQVE.}
]
                
                \antilabe Ne me ſuis point, ſi
 ton cœur en allarmes,&
       Preuoit qu’il ne pourra
 commander à tes larmes,&
       On vient. Cache tes pleurs, Cephize, \ampersand\ ſouuiens-toy,&
       Que le ſort d’Andromaque eſt commis à
 ta foy.&
       C’eſt Hermionne. Allons, fuyons ſa
 violence.\&
       
                      
\stanza[\section{SCENE II.}
\personscene{HERMIONNE, CLEONNE.}
\enonciateur{CLEONNE.}
                ]
                
                NOn, ie
 ne puis aſſez admirer ce ſilence.&
       Vous vous taiſez, Madame, \ampersand\
 ce cruel mépris&
       N’a pas du moindre trouble agité vos eſprits?&
       Vous ſouſtenez
 en paix vne ſi rude attaque?&
       Vous qu’on voyoit fremir au ſeul
 nom d’Andro-maque?&
       Vous qui ſans deſeſpoir ne pouuiez endurer&
       Que Pyrrhus
 d’vn regard la vouluſt honorer?&
       Il l’épouſe. Il luy donne auec ſon Diadéme&
       La foy, que vous venez de receuoir vous-meſme;&
       Et voſtre bouche encor muette à
 tant d’ennuy,&
       N’a pas daigné s’ouurir pour ſe plaindre de luy?&
       Ah! que ie crains, Madame, vn calme ſi funeſte!&
       Et qu’il vaudroit bien mieux....\&
       
\stanza[
\enonciateur{HERMIONNE.}
]
                
                \antilabe Fais-tu venir Oreſte!\&
       
\stanza[
\enonciateur{CLEONNE.}
]
                
                Il vient, Madame, il vient. Et vous pouuez juger,&
       Que bientoſt à vos pieds il
 alloit ſe ranger.&
       Preſt à ſeruir
 toûjours ſans eſpoir de ſalaire,&
       Vos yeux ne ſont que trop aſſurez de luy plaire.&
       Mais il entre.\&
       
                      
\stanza[\section{SCENE III.}
\personscene{ORESTE, HERMIONNE, CLEONNE,.}
\enonciateur{ORESTE.}
                ]
                
                \antilabe AH Madame? Eſt-il vray qu’vne fois&
       Oreſte en vous cherchant obeïſſe à vos lois?&
       Ne m’a-t-on point flatté d’vne
 fauſſe eſperance?&
       Auez-vous en-effet ſouhaitté ma
 preſence?&
       Croiray-ie que vos yeux à la fin deſarmez&
       Veulent.....\&
       
\stanza[
\enonciateur{HERMIONNE.}
]
                
                \antilabe Ie veux ſçauoir, Seigneur, ſi vous m’aimez.\&
       
\stanza[
\enonciateur{ORESTE.}
]
                
                Si ie vous aime? O Dieux! mes ſermens, mes par-jures,&
       Ma fuite, mon retour, mes reſpects, mes injures,&
       Mon deſeſpoir,
 mes yeux de pleurs toûjours noyez,&
       Quels témoins croirez-vous, ſi
 vous ne les croyez?\&
       
\stanza[
\enonciateur{HERMIONNE.}
]
                
                Vangez-moy, ie croy
 tout.\&
       
\stanza[
\enonciateur{ORESTE.}
]
                
                \antilabe Hé bien allons, Madame.&
       Mettons encore vn coup toute la
 Grece en flame.&
       Prenons, en ſignalant mon bras,
 \ampersand\ voſtre nom,&
       Vous la place d’Helene,
 \ampersand\ moy d’Agamemnon.&
       De Troye en ce pays
 réueillons les miſeres,&
       Et qu’on parle de nous, ainſi
 que de nos Peres.&
       Partons, ie ſuis tout preſt.\&
       
\stanza[
\enonciateur{HERMIONNE.}
]
                
                \antilabe Non, Seigneur, demeurons,&
       Ie ne veux pas ſi loin porter de tels affrons.&
       Quoy de mes ennemis couronnant l’inſolence,&
       I’irois attendre ailleurs vne lente vengeance,&
       Et ie m’en remettrois au deſtin des combats,&
       Qui peut-eſtre à la fin ne me
 vangeroit pas?&
       Ie veux qu’à mon depart toute
 l’Epire pleure.&
       Mais ſi vous me vangez,
 vangez-moydans vne heure.&
       Tous vos retardemens ſont pour
 moy des refus.&
       Courez au Temple. Il faut
 immoler....\&
       
\stanza[
\enonciateur{ORESTE.}
]
                
                \antilabe Qui?\&
       
\stanza[
\enonciateur{HERMIONNE.}
]
                
                \antilabe Pyrrhus.\&
       
\stanza[
\enonciateur{ORESTE.}
]
                
                Pyrrhus, Madame?\&
       
\stanza[
\enonciateur{HERMIONNE.}
]
                
                \antilabe Hé! quoy? voſtre haine chancelle?&
       Ah! courez, \ampersand\ craignez que ie ne vous rappelle.&
       N’alleguez point des droits que ie veux oublier.&
       Et ce n’eſt pas à vous à le juſtifier.\&
       
\stanza[
\enonciateur{ORESTE.}
]
                
                Moy, ie l’excuſerois? Ah! vos bontez, Madame,&
       Ont graué trop auant ſes crimes dans mon ame.&
       Vangeons-nous, j’y conſens. Mais
 par d’autres chemins.&
       Soyons ſes Ennemis, \ampersand\ non ſes Aſſaſſins.&
       Faiſons de ſa
 ruine vne juſte Conqueſte.&
       Quoy pour réponſe aux Grecs
 porteray-je ſa teſte?&
       Et n’ay-je pris ſur moy le ſoin de tout l’Eſtat,&
       Que pour m’en acquitter par vn
 Aſſaſſinat?&
       Souffrez, au nom des Dieux, que la Grece
 s’explique,&
       Et qu’il meure chargé de la haine publique.&
       Souuenez-vous qu’il regne, \ampersand\
 qu’vn front cou-ronné…\&
       
\stanza[
\enonciateur{HERMIONNE.}
]
                
                Ne vous ſuffit-il pas que ie l’ay condamné?&
       Ne vous ſuffit-il pas que ma
 Gloire offenſée&
       Demande vne Victime, à moy ſeule adreſſée;&
       Qu’Hermionne eſt le prix d’vn Tyran opprimé,&
       Que ie le hais, enfin, Seigneur,
 que ie l’aimay?&
       Ie ne m’en cache point. L’ingrat
 m’auoit ſçeû plaire,&
       Soit qu’ainſi l’ordonnaſt mon amour, ou mon Pere,&
       N’importe. Mais enfin reglez-vous là-deſſus.&
       Malgré mes vœux, Seigneur, honteuſement deceûs,&
       Malgré la juſte horreur que ſon crime me donne,&
       Tant qu’il viura, craignez que ie ne luy pardonne.&
       Doutez iuſqu’à ſa mort d’vn courroux incertain,&
       S’il ne meurt aujourd’huy, ie
 puis l’aimer demain.\&
       
\stanza[
\enonciateur{ORESTE.}
]
                
                Hé bien, il faut le perdre, \ampersand\ préuenir ſa grace.&
       Il faut.... Mais cependant, que faut-il que ie faſſe?&
       Comment puis-je ſi-toſt ſeruir voſtre
 courroux?&
       Quel chemin iuſqu'à luy peut
 conduire mes coups?&
       A peine ſuis-je encore arriué
 dans l’Epire,&
       Vous voulez par mes mains renuerſer vn Empire.&
       Vous voulez qu’vn Roy meure,
 \ampersand\ pour ſon chaſti-ment,&
       Vous ne donnez qu’vn iour, qu’vne heure, qu’vn moment.&
       Aux yeux de tout ſon Peuple, il
 faut que ie l’opprime?&
       Laiſſez-moy
 vers l’Autel conduire ma Victime.&
       Ie ne m’en défens plus. Et ie ne veux qu’aller&
       Reconnoiſtre la place où ie dois l’immoler.&
       Cette Nuit ie vous ſers. Cette Nuit ie l’attaque.\&
       
\stanza[
\enonciateur{HERMIONNE.}
]
                
                Mais cependant ce Iour il eſpouſe Andromaque.&
       Dans le Temple déja le troſne
 eſt éleué.&
       Ma honte eſt confirmée, \ampersand\ ſon Crime acheué.&
       Enfin qu’attendez-vous? Il vous offre ſa Teſte.&
       Sans Gardes, ſans défenſe il marche à cette Feſte.&
       Autour du Fils d’Hector il les fait tous ranger.&
       Il s’abandonne au bras qui me voudra vanger.&
       Voulez-vous, malgré luy, prendre ſoin de ſa vie?&
       Armez auec vos Grecs, tous ceux
 qui m’ont ſuiuie.&
       Souleuez vos Amis. Tous les
 miens ſont à vous.&
       Il me trahit, vous trompe, \ampersand\ nous meſpriſe tous.&
       Mais quoy? Déja leur haine eſt
 égale à la mienne.&
       Elle eſpargne à regret l’Eſpoux d’vne Troyenne.&
       Parlez. Mon Ennemy ne vous peut échapper.&
       Ou plûtoſt, il ne faut que les
 laiſſer frapper.&
       Conduiſez, ou ſuiuez vne fureur ſi belle.&
       Reuenez tout couuert du ſang de l’Infidelle.&
       Allez, en cét eſtat ſoyez ſeûr de mon cœur.\&
       
\stanza[
\enonciateur{ORESTE.}
]
                
                Mais, Madame, ſongez…\&
       
\stanza[
\enonciateur{HERMIONNE.}
]
                
                \antilabe Ah! c’en eſt trop, Seigneur.&
       Tant de raiſonnemens offenſent ma colere.&
       I’ay voulu vous donner les
 moyens de me plaire,&
       Rendre Oreſte content. Mais enfin ie voy
 bien,&
       Qu’il veut touſiours ſe plaindre, \ampersand\ ne meriter rien.&
       Partez. Allez ailleurs vanter voſtre conſtance,&
       Et me laiſſez
 icy le ſoin de ma vangeance.&
       De mes laſches bontez mon
 courage eſt confus,&
       Et c’eſt trop en vn iour eſſuyer de refus.&
       Ie m’en vais ſeule au Temple, où leur hymen s’apreſte,&
       Où vous n’oſez aller meriter ma
 conqueſte.&
       Là, de mon Ennemy ie ſçauray m’approcher.&
       Ie perceray le Cœur, que ie n’ay pû toucher.&
       Et mes ſanglantes mains ſur moy-meſme tournées,&
       Auſſi-toſt, malgré luy, joindront nos deſtinées,&
       Et tout Ingrat qu’il eſt, il me
 ſera plus doux,&
       De mourir auec luy, que de viure
 auec vous.\&
       
\stanza[
\enonciateur{ORESTE.}
]
                
                Non, ie vous priueray de ce plaiſir funeſte,&
       Madame. Il ne mourra que de la main d’Oreſte.&
       Vos Ennemis par moy vont vous eſtre immolez.&
       Et vous reconnoiſtrez mes ſoins, ſi vous voulez.&
       Mais que dis-je? Ah plûtoſt!
 permettez que j’eſpere.&
       Excuſez vn
 Amant, que trouble ſa miſere,&
       Qui tout preſt d’eſtre heureux, enuie encor le ſort&
       D’vn Ingrat, condamné par
 vous-meſme à la mort.\&
       
\stanza[
\enonciateur{HERMIONNE.}
]
                
                Allez. De voſtre ſort laiſſez-moy la
 conduite.&
       Et que tous vos Vaiſſeaux ſoient preſts pour
 noſtre fuite.\&
       
                      
\stanza[\section{SCENE IV.}
\personscene{HERMIONNE, CLEONNE.}
\enonciateur{}
                ]
                
                VOus vous perdez, Madame.
 Et vous deuez ſonger....\&
       
\stanza[
\enonciateur{HERMIONNE.}
]
                
                Que ie me perde, ou non, ie ſonge à me vanger.&
       Ie ne ſçay meſme encor, quoy qu’il m’ait pû pro-mettre,&
       Sur d’autres que ſur moy, ſi ie doy m’en remettre.&
       Pyrrhus n’eſt pas coupable à ſes yeux, comme aux
 miens,&
       Et ie tiendrois mes coups bien
 plus ſeûrs que les ſiens.&
       Quel plaiſir! de vanger moy-meſme mon injure,&
       De retirer mon bras teint du ſang du Parjure,&
       Et pour rendre ſa peine \ampersand\
 mes plaiſirs plus grands,&
       De cacher ma Riuale à ſes regards mourans.&
       Ah! ſi du moins Oreſte, en puniſſant ſon crime,&
       Luy laiſſoit le
 regret de mourir ma Victime.&
       Va le trouuer. Dy-luy qu’il
 aprenne à l’Ingrat,&
       Qu’on l’immole à ma haine, \ampersand\ non pas à l’Eſtat.&
       Chere Cleonne
 cours. Ma vangeance eſt perduë,&
       S’il ignore, en mourant, que c’eſt moy qui le tuë.\&
       
\stanza[
\enonciateur{CLEONNE.}
]
                
                Ie vous obeïray. Mais qu’eſt-ce que ie voy?&
       O Dieux! Qui l’auroit crû, Madame? C’eſt le Roy.\&
       
\stanza[
\enonciateur{HERMIONNE.}
]
                
                Ah! cours apres Oreſte, \ampersand\ dy-luy, ma Cleonne,&
       Qu’il n’entreprenne rien ſans
 reuoir Hermionne.\&
       
                      
\stanza[\section{SCENE V.}
\personscene{PYRRHVS, HERMIONNE, PHOENIX.}
\enonciateur{PYRRHVS.}
                ]
                
                VOus ne m’attendiez pas,
 Madame, \ampersand\ ie voy bien&
       Que mon abord icy trouble voſtre
 entretien.&
       Ie ne viens point armé d’vn indigne artifice&
       D’vn voile d’équité couurir mon injuſtice.&
       Il ſuffit, que mon cœur me
 condamne tout bas,&
       Et ie ſoûtiendrois mal ce que ie ne croy pas.&
       I’épouſe vne Troyenne. Oüy, Madame, \ampersand\ j’auouë&
       Que ie vous ay promis la foy,
 que ie luy vouë.&
       Vn autre vous diroit, que dans
 les champs Troyens&
       Nos deux Peres ſans nous
 formerent ces liens,&
       Et que ſans conſulter ny mon cœur ny le voſtre,&
       Nous fuſmes ſans amour engagez l’vn à l’autre.&
       Mais c’eſt aſſez pour moy que ie me ſois ſoûmis.&
       Par mes Ambaſſadeurs mon cœur vous fut promis.&
       Loin de les reuoquer, ie voulus y ſouſcrire.&
       Ie vous vis auec eux arriuer en Epire.&
       Et quoy que d’vn autre œil
 l’éclat victorieux&
       Euſt déja préuenu le pouuoir de vos yeux;&
       Ie ne m’arreſtay point à cette ardeur nouuelle.&
       Ie voulus m’obſtiner à vous eſtre fidelle.&
       Ie vous receûs en Reine, \ampersand\
 iuſques à ce jour,&
       I’ay cru que mes ſermens me tiendroient lieu d’a-mour.&
       Mais cét amour l’emporte. Et par vn coup funeſte,&
       Andromaque
 m’arrache vn cœur qu’elle déteſte.&
       L’vn par l’autre entraiſnez, nous courons à l’Autel&
       Nous jurer, malgré nous, vn
 amour immortel.&
       Apres cela, Madame, éclatez contre vn Traiſtre,&
       Qui l’eſt auec
 douleur, \ampersand\ qui pourtant veut l’eſtre.&
       Pour moy, loin de contraindre vn
 ſi iuſte cour-roux,&
       Il me ſoulagera peut-eſtre autant que vous.&
       Donnez-moy tous les noms deſtinez aux Parjures.&
       Ie crains voſtre ſilence, \ampersand\ non pas vos injures,&
       Et mon Cœur ſouleuant mille ſecrets teſmoins,&
       M’en dira d’autant plus que vous m’en direz
 moins.\&
       
\stanza[
\enonciateur{HERMIONNE.}
]
                
                Seigneur, dans cét aueu deſpoüillé d’artifice,&
       I’aime à voir que du moins vous
 vous rendiez ju-ſtice,&
       Et que voulant bien rompre vn
 nœud ſi ſolennel,&
       Vous vous abandonniez au crime en criminel.&
       Eſt-il juſte
 apres tout, qu’vn Conquerant s’abaiſſe&
       Sous la ſeruile loy de garder ſa promeſſe?&
       Non, non, la Perfidie a dequoyde quoy vous tenter.&
       Et vous ne me cherchez que pour vous en
 vanter.&
       Quoy? Sans que ny ſerment, ny
 deuoir vous re-tienne,&
       Rechercher vne Grecque, Amant
 d’vne Troyenne?&
       Me quitter, me reprendre, \ampersand\ retourner
 encor&
       De la Fille d’Helene, à la
 Veuue d’Hector?&
       Couronner tour à tour l’Eſclaue, \ampersand\ la Princeſſe,&
       Immoler Troye aux Grecs, au Fils d’Hector la Grece?&
       Tout cela part d’vn cœur touſiours maiſtre de ſoy,&
       D’vn Heros qui n’eſt point Eſclaue de ſa foy.&
       Pour plaire à voſtre Eſpouſe, il vous faudroit peut-eſtre&
       Prodiguer les doux noms de Parjure, \ampersand\ de
 Traiſtre.&
       Voſtre grand cœur ſans doute attend apres mes pleurs,&
       Pour aller dans ſes bras joüir
 de mes douleurs?&
       Chargé de tant d’honneur il veut qu’on le
 renuoye?&
       Mais, Seigneur, en vn iour ce ſeroit trop de joye.&
       Et ſans chercher ailleurs des
 titres empruntez,&
       Ne vous ſuffit-il pas de ceux
 que vous portez?&
       Du vieux Pere d’Hector la valeur abbattuë&
       Aux pieds de ſa Famille
 expirante à ſa veuë,&
       Tandis que dans ſon ſein voſtre bras enfoncé&
       Cherche vn reſte de ſang que l’âge auoit
 glacé;&
       Dans des ruiſſeaux de ſang Troye ardante plongée,&
       De voſtre propre main Polyxene eſgorgée&
       Aux yeux de tous les Grecs indignez contre
 vous,&
       Que peut-on refuſer à ces
 genereux coups?\&
       
\stanza[
\enonciateur{PYRRHVS.}
]
                
                Madame, ie ſçay
 trop, à quel excez de rage&
       L’ardeur de vous vanger emporta mon courage.&
       Ie puis me plaindre à vous du ſang que i’ay verſé.&
       Mais enfin ie conſens d’oublier le paſſé.&
       Ie rends graces au Ciel, que
 voſtre indifference&
       De mes heureux ſoûpirs m’aprenne
 l’innocence.&
       Mon cœur, ie le voy bien, trop
 prompt à ſe geſner,&
       Deuoit mieux vous cõnoiſtre, \ampersand\ mieux s’examiner.&
       Mes remords vous faiſoient vne injure mortelle,&
       Il faut ſe croire aimé, pour ſe croire infidelle.&
       Vous ne prétendiez point m’arreſter dans vos fers.&
       I’ay craint de vous trahir,
 peut-eſtre ie vous ſers.&
       Nos Cœurs n’eſtoient point faits
 dépendans l’vn de l’autre.&
       Ie ſuivois mon
 deuoir, \ampersand\ vous cediez au voſtre.&
       Rien ne vous engageoit à m’aimer en effet.\&
       
\stanza[
\enonciateur{HERMIONNE.}
]
                
                Ie ne t'ay point aimé, Cruel?
 Qu'ay-je donc fait?&
       I’ay deſdaigné
 pour toy les vœux de tous nos Princes,&
       Ie t’ay cherché moy-meſme au fond de tes Pro-uinces.&
       I’y ſuis encor,
 malgré tes infidelitez,&
       Et malgré tous mes Grecs honteux de mes
 bontez.&
       Ie leur ay commandé de cacher
 mon injure,&
       I’attendois en ſecret le retour d’vn Parjure,&
       I’ay creu que toſt ou tard à ton deuoir rendu,&
       Tu me rapporterois vn Cœur qui
 m’eſtoit dû.&
       Ie t’aimois inconſtant, qu’aurois-je fait fidelle?&
       Et meſme en ce moment, où ta
 bouche cruelle&
       Vient ſi tranquillement
 m’annoncer le treſpas,&
       Ingrat, ie doute encor, ſi ie ne t’aime pas.&
       Mais, Seigneur, s’il le faut, ſi
 le Ciel en colere&
       Reſerue à d’autres yeux la
 gloire de vous plaire,&
       Acheuez voſtre
 hymen, j’y conſens. Mais du moins&
       Ne forcez pas mes yeux d’en eſtre les teſmoins.&
       Pour la derniere fois ie vous
 parle peut-eſtre ,&
       Differez-le d’vn iour, demain
 vous ſerez maiſtre.&
       Vous ne reſpondez point.
 Perfide, ie le voy,&
       Tu contes les momens que tu perds auec moy.&
       Ton cœur impatient de reuoir ſa Troyenne,&
       Ne ſouffre qu’à regret qu’vn autre t’entretienne,&
       Tu luy parles du cœur, tu la cherches des
 yeux.&
       Ie ne te retiens plus, ſauue-toy de ces lieux.&
       Va luy jurer la foy, que tu m’auois jurée.&
       Va profaner des Dieux la Majeſté
 ſacrée.&
       Ces Dieux, ces juſtes Dieux
 n’auront pas ou-blié,&
       Que les meſmes ſermens auec moy t’ont lié.&
       Porte aux pieds des Autels ce Cœur qui m’aban-donne.&
       Va, cours. Mais crains encor d’y trouuer Her-mionne.\&
       
                      
\stanza[\section{SCENE VI.}
\personscene{PYRRHVS, PHOENIX.}
\enonciateur{PHOENIX.}
                ]
                
                SEigneur, vous
 l’entendez. Gardez de negliger&
       Vne Amante en fureur, qui
 cherche à ſe van-ger.&
       Elle n’eſt en ces lieux que trop
 bien appuyée,&
       La querelle des Grecs à la ſienne eſt liée.&
       Oreſte l’aime encore. Et peut-eſtre à ce
 prix…\&
       
\stanza[
\enonciateur{PHOENIXPYRRHVS.}
]
                
                Andromaque m’attend. Phœnix, garde ſon Fils.\&
       \stanza[\chapter{ACTE V.}
\section{SCENE PREMIERE.} 
    \personscene{}  
    \enonciateur{HERMIONNE} 
    ]
    
    O v ſuis-je? Qu’ay-je fait? Que dois-je
 faire encore?&
       Quel tranſport me ſaiſit? Quel cha-grin me deuore?&
       Errante, \ampersand\ ſans deſſein, ie cours dans
  ce Palais.&
       Ah! ne puis-je ſçavoir ſi j’aime, ou ſi ie
 hais!&
       Le Cruel! De quel œil il m’a congédiée?&
       Sans pitié, ſans douleur, au
 moins eſtudiée.&
       Ay-je veû ſes regards ſe troubler vn moment?&
       En ay-je pû tirer vn ſeul gemiſſement?&
       Muet à mes ſoupirs, tranquille à
 mes allarmes,&
       Sembloit-il ſeulement qu’il euſt part à mes larmes?&
       Et ie le plains encore? Et pour
 comble d’ennuy&
       Mon cœur, mon laſche cœur
 s’intereſſe pour luy?&
       Ie tremble au ſeul penſer du coup qui le menaſſe?&
       Et preſte à me vanger ie luy fais déja grace?&
       Non, ne reuoquons point l’arreſt de mon courroux.&
       Qu’il periſſe.
 Auſſi bien il ne vit plus pour
 nous.&
       Le Perfide triomphe, \ampersand\ ſe
 rit de ma rage.&
       Il penſe voir en pleurs diſſiper cét orage.&
       Il croit que touſiours foible,
 \ampersand\ d’vn cœur incer-tain,&
       Ie pareray d’vn
 bras les coups de l’autre main.&
       Il juge encor de moy par mes bontez paſſées.&
       Mais plûtoſt le Perfide a bien
 d’autres penſées.&
       Triomphant dans le Temple, il ne s’informe
 pas&
       Si l’on ſouhaite ailleurs ſa vie, ou ſon treſpas.&
       Il me laiſſe,
 l’Ingrat! cét embarras funeſte.&
       Non, non, encore vn coup, laiſſons agir Oreſte.&
       Qu’il meure, puis qu’enfin il a dû le preuoir,&
       Et puis qu’il m’a forcée enfin à le vouloir.&
       A le vouloir? Hé quoy? C’eſt
 donc moy qui l’or-donne?&
       Sa Mort ſera l’effet de l’amour
 d’Hermionne?&
       Ce Prince, dont mon cœur ſe
 faiſoit autrefois,&
       Auec tant de plaiſir, redire les
 Exploits,&
       A qui meſme en ſecret ie m’eſtois deſtinée,&
       Auant qu’on euſt conclu ce fatal
 hymenée,&
       Ie n’ay donc trauerſé tant de mers, tant d’Eſtats,&
       Que pour venir ſi loin preparer
 ſon treſpas,&
       L’aſſaſſiner, le perdre? Ah deuant qu’il expire…\&
       
                      
\stanza[\section{SCENE II.}
\personscene{HERMIONNE, CLEONNE.}
\enonciateur{HERMIONNE.}
                ]
                
                AH! qu’ay-je fait,
 Cleonne? Et que viens-tu me
 dire?&
       Que fait Pyrrhus?\&
       
\stanza[
\enonciateur{CLEONNE.}
]
                
                \antilabe Il eſt au comble de ſes vœux,&
       Le plus fier des Mortels, \ampersand\ le plus
 amoureux.&
       Ie l’ay veû vers le Temple, où
 ſon hymen s’apreſte,&
       Mener en Conquerant ſa nouuelle Conqueſte,&
       Et d’vn œil qui déja deuoroit ſon eſpoir,&
       S’enyurer, en marchant, du plaiſir de la voir.&
       Andromaque, au trauers de mille cris de
 joye,&
       Porte iuſqu’aux Autels le ſouuenir de Troye,&
       Incapable touſiours d’aimer,
 \ampersand\ de haïr,&
       Sans joye, \ampersand\ ſans murmure
 elle ſemble obeïr.\&
       
\stanza[
\enonciateur{HERMIONNE.}
]
                
                Et l’Ingrat? Iuſqu’au bout il a
 pouſſé l’outrage?&
       Mais as-tu bien, Cleonne, obſerué ſon
 viſage?&
       Gouſte-t’il des plaiſirs tranquilles \ampersand\ parfaits?&
       N’a-t’il point détourné ſes yeux
 vers le Palais?&
       Dy-moy, ne t’es-tu point preſentée à ſa veuë?&
       L’Ingrat a-t’il rougy, lors qu’il t’a
 reconnuë?&
       Son trouble auoüoit-il ſon infidelité?&
       A-t’il iuſqu’à la fin ſoûtenu ſa fierté?\&
       
\stanza[
\enonciateur{CLEONNE.}
]
                
                Madame, il ne voit rien. Son ſalut, \ampersand\ ſa gloire&
       Semble eſtre auec vous ſortis de ſa memoire.&
       Sans ſonger qui le ſuit, Ennemis, ou Sujets,&
       Il pourſuit ſeulement ſes amoureux projets.&
       Autour du Fils d’Hector il a rangé ſa Garde,&
       Et croit que c’eſt luy ſeul que le peril regarde.&
       Phœnix meſme en reſpond, qui l’a conduit
 exprés&
       Dans vn Fort éloigné du Temple,
 \ampersand\ du Palais.&
       Voila, dans ſes transports, le
 ſeul ſoin qui luy reſte.\&
       
\stanza[
\enonciateur{HERMIONNE.}
]
                
                Le Perfide! Il mourra. Mais que t’a dit Oreſte?\&
       
\stanza[
\enonciateur{CLEONNE.}
]
                
                Oreſte, auec ſes Grecs,
 dans le Temple eſt entré.\&
       
\stanza[
\enonciateur{HERMIONNE.}
]
                
                Hé bien? A me vanger n’eſt-il
 pas preparé?\&
       
\stanza[
\enonciateur{CLEONNE.}
]
                
                Ie ne ſçay.\&
       
\stanza[
\enonciateur{HERMIONNE.}
]
                
                \antilabe Tu ne ſçais? Quoy donc Oreſte encore,&
       Oreſte me trahit?\&
       
\stanza[
\enonciateur{CLEONNE.}
]
                
                \antilabe Oreſte vous adore.&
       Mais de mille remords ſon eſprit combattu&
       Croit tantoſt ſon amour, \ampersand\ tantoſt ſa
 vertu.&
       Il reſpecte en Pyrrhus l’honneur du diadéme.&
       Il reſpecte en PyrrhusAchille, \ampersand\ Pyrrhus meſme.&
       Il craint les Grecs, il craint l’Vniuers en courroux.&
       Mais il ſe craint, dit-il, ſoy-meſme plus que tous.&
       Il voudroit en Vainqueur vous apporter ſa teſte.&
       Le ſeul nom d’Aſſaſſin
 l’épouuante \ampersand\ l’arreſte.&
       Enfin il eſt entré, ſans ſçauoir dans ſon cœur,&
       S’il en deuoit ſortir Coupable, ou Spectateur.\&
       
\stanza[
\enonciateur{HERMIONNE.}
]
                
                Non, non, il les verra triompher ſans obſtacle,&
       Il ſe gardera bien de troubler
 ce ſpectacle.&
       Ie ſçay de
 quels remords ſon courage eſt
 atteint.&
       Le laſche craint la mort, \ampersand\
 c’eſt tout ce qu’il craint.&
       Quoy? ſans qu’elle employaſt vne ſeule
 priere,&
       Ma Mere en ſa faueur arma la Grece entiere?&
       Ses yeux pour leur querelle, en dix ans de
 com-bats,&
       Virent perir vingt Rois, qu’ils ne connoiſſoient pas?&
       Et moy je ne prétens que la mort d’vn Parjure,&
       Et ie charge vn
 Amant du ſoin de mon injure,&
       Il peut me conquerir à ce prix, ſans danger,&
       Ie me liure moy-meſme, \ampersand\ ne puis me vanger?&
       Allons. C’eſt à moy ſeule, à me rendre juſtice.&
       Que de cris de douleur le Temple retentiſſe.&
       De leur hymen fatal troublons l’euenement,&
       Et qu’ils ne ſoient vnis, s’il ſe peut, qu’vn
 moment.&
       Ie ne choisiray point dans ce
 deſordre extréme.&
       Tout me ſera Pyrrhus, fuſt-ce Oreſte luy-meſme.&
       Ie mourray. Mais au moins ma
 mort me vangera,&
       Ie ne mourray pas ſeule, \ampersand\ quelqu’vn me ſuiura.\&
       
                      
\stanza[\section{SCENE III.}
\personscene{ORESTE, ANDROMAQVE, HERMIONNE, CLEONNE, CEPHISE, soldats
 d'ORESTE.}
\enonciateur{ORESTE.}
                ]
                
                MAdame, c’en eſt fait. Partons en diligence.&
       Venez dans mes vaiſſeaux goûter voſtre vangeance.&
       Voyez cette Captiue. Elle peut
 mieux que moy&
       Vous apprendre qu’Oreſte a dégagé ſa foy.\&
       
\stanza[
\enonciateur{HERMIONNE.}
]
                
                O Dieux! C’eſt
 Andromaque?\&
       
\stanza[
\enonciateur{ANDROMAQVE.}
]
                
                \antilabe Oüy, c’est cette Princeſſe&
       Deux fois Veuue, \ampersand\ deux fois
 l’Eſclaue de la Grece;&
       Mais qui juſque dans Sparte ira vous brauer
 tous,&
       Puis qu’elle voit ſon Fils à
 couuert de vos coups.&
       Du crime de Pyrrhus complice manifeſte,&
       I’attens ſon
 chaſtiment. Car je voy bien qu’Oreſte&
       Engagé par voſtre ordre à cét
 aſſaſſinat,&
       Vient de ce triſte exploit vous
 ceder tout l’éclat.&
       Ie ne m’attendois pas que le
 Ciel en colere&
       Puſt, ſans
 perdre mon Fils, accroiſtre ma miſere,&
       Et gardaſt à mes yeux quelque ſpectacle encor,&
       Qui fiſt couler mes pleurs pour
 vn autre qu’He-ctor.&
       Vous auez trouué ſeule vne ſanglante voye&
       De ſuſpendre en
 mon cœur le ſouvenir de Troye.&
       Plus barbare aujourd’huy qu’Achille \ampersand\ que ſon
 Fils,&
       Vous me faites pleurer mes plus grands
 Ennemis;&
       Et ce que n’auoient pû promeſſe, ny menaſſe,&
       Pyrrhus de
 mon Hectorſemble auoir pris la place.&
       Ie n’ay que trop, Madame,
 éprouué ſon courroux,&
       I’auois plus de
 ſujet de m’en plaindre que vous.&
       Pour derniere rigueur, ton amitié cruelle,&
       Pyrrhus, à
 mon Epoux me rendoit infidelle.&
       Ie t’en allois punir. Mais le
 Ciel m’eſt témoin,&
       Que je ne pouſſois pas ma vangeance ſi loin,&
       Et ſans verſer
 ton ſang, ny cauſer tant
 d’allarmes,&
       Il ne t’en euſt couſté peut-eſtre que des larmes.\&
       
\stanza[
\enonciateur{HERMIONNE.}
]
                
                Quoy? Pyrrhus eſt donc mort!\&
       
\stanza[
\enonciateur{ORESTE.}
]
                
                \antilabe Oüy, nos Grecs irritez&
       Ont laué dans ſon sang ſes infidelitez.&
       Ie vous l’auois
 promis. Et quoy que mon courage&
       Se fiſt de ce complot vne funeſte image,&
       I’ay couru vers le Temple, où
 nos Grecs diſperſez&
       Se ſont juſqu’à
 l’Autel dans la foule gliſſez.&
       Pyrrhus m’a
 reconnu. Mais ſans changer de face,&
       Il ſembloit que ma veuë excitaſt ſon audace,&
       Que tous les Grecs brauez en
 leur Ambaſſadeur&
       Deuſſent de ſon Hymen releuer la ſplendeur.&
       Enfin auec tranſport prenant ſon diadéme,&
       Sur le front d’Andromaque il l’a poſé luy-meſme.&
       Ie vous donne, a-t’il dit, ma
 Couronne, \ampersand\ ma Foy;&
       Andromaque, regnez ſur l’Epire \ampersand\ ſur moy.&
       Ie voüe à voſtre Fils vne amitié de Pere,&
       I’en atteſte
 les Dieux, je le jure à ſa Mere.&
       Pour tous mes Ennemis je declare les ſiens,&
       Et je le reconnoy pour le Roy des Troyens.&
       A ces mots, qui du Peuple attiroient le ſuffrage,&
       Nos Grecs n’ont répondu que par vn cry de rage,&
       L’Infidelle s’eſt vû par tout
 enuelopper,&
       Et je n’ay pû trouuer de place
 pour frapper.&
       Chacun ſe diſputoit la gloire de l’abbatre.&
       Ie l’ay veû dans leurs mains
 quelque temps ſe de-battre,&
       Tout ſanglant à leurs coups
 vouloir ſe dérober,&
       Mais enfin à l’Autel il eſt allé
 tomber.&
       Le Troyen eſt ſauué. Mais partons, le temps preſſe.&
       L’Epire toſt ou tard ſatisfera la Grece:&
       Cependant j’ay voulu qu’Andromaque aujour-d’huy&
       Honnoraſt mon triomphe, \ampersand\
 répondit de luy.&
       Du Peuple épouuanté la foule
 fugitiue&
       M’a laiſſé ſans obstacle enleuer ma Captiue,&
       Et regagner ces lieux, où bien-toſt nos Amis&
       Viendront couuerts du ſang, que je vous ay promis.\&
       
\stanza[
\enonciateur{HERMIONNE.}
]
                
                Qu’ont-ils fait!\&
       
\stanza[
\enonciateur{ORESTE.}
]
                
                \antilabe Pardonnez à leur impatience.&
       Ils ont, je le voy bien, trahy voſtre vangeance.&
       Vous vouliez que ma main portaſt
 les premiers coups,&
       Qu’il ſentiſt
 en mourant qu’il expiroit pour vous.&
       Mais c’eſt moy, dont l’ardeur
 leur a ſeruy d’exem-ple,&
       Ie les ay pour vous ſeule entraînez dans le Tem-ple,&
       Madame, \ampersand\ vous pouuez juſtement vous flatter&
       D’vne mort, que les Grecs n’ont
 fait qu’executer.&
       Vous ſeule auez pouſſé les
 coups.....\&
       
\stanza[
\enonciateur{HERMIONNE.}
]
                
                \antilabe Tay-toy, Perfide,&
       Et n’impute qu’à toy ton laſche
 Parricide.&
       Va faire chez tes Grecs admirer ta fureur,&
       Va, je la deſauouë, \ampersand\ tu me fais horreur.&
       Barbare qu’as-tu fait? auec
 quelle furie&
       As-tu tranché le cours d’vne ſi belle vie?&
       Auez-vous pû, Cruels, l’immoler aujourd’huy,&
       Sans que tout voſtre ſang ſe ſoûleuaſt pour luy?&
       Mais parle. De ſon sort qui t’a
 rendu l’arbitre?&
       Pourquoy l’aſſaſſiner? Qu’a-t'il fait? A
 quel titre?&
       Qui te l’a dit?\&
       
\stanza[
\enonciateur{ORESTE.}
]
                
                \antilabe O dieux! Quoy ne m’auez-vous pas&
       Vous-meſme, icy, tantoſt, ordonné ſon trépas?\&
       
\stanza[
\enonciateur{HERMIONNE.}
]
                
                Ah! Falloit-il en croire vne
 Amante inſenſée?&
       Ne deuois-tu pas lire au fonds
 de ma penſée?&
       Et ne voyois-tu pas dans mes emportemens,&
       Que mon cœur démentoit ma bouche à tous mo-mens?&
       Quand je l’aurois voulu, falloit-il y ſouſcrire?&
       N’as-tu pas dû cent fois te le faire redire,&
       Toy-meſme auant
 le coup me venir conſulter,&
       Y reuenir encore, ou plûtoſt m’éuiter?&
       Que ne me laiſſois-tu le ſoin de ma vangeance?&
       Qui t’ameine en des lieux, où l’on fuit ta preſence?&
       Voila de ton amour le deteſtable
 fruit.&
       Tu m’apportois, Cruel, le malheur qui te ſuit.&
       C’eſt toy, dont l’Ambaſſade à tous les deux fatale,&
       L’a fait pour ſon malheur
 pancher vers ma Riuale;&
       Nous le verrions encor nous partager ſes ſoins,&
       Il m’aimeroit peut-eſtre , il le
 feindroit du moins.&
       Adieu. Tu peux partir. Ie
 demeure en Epire,&
       Ie renonce à la Grece, à
 Sparte, à ſon
 Empire,&
       A toute ma Famille. Et c’eſt aſſez pour moy,&
       Traiſtre, qu’elle ait produit vn monstre comme toy.&
       Allons, Madame, allons. C’eſt
 moy qui vous dé-liure.&
       Pyrrhus ainſi l’ordonne, \ampersand\ vous pouuez me ſuiure.&
       De nos derniers deuoirs allons
 nous dégager.&
       Montrons qui de nous deux ſçaura
 mieux le van-ger.\&
       
                      
\stanza[\section{SCENE IV.}
\personscene{ORESTE, Soldats d'Oreſte.}
\enonciateur{ORESTE.}
                ]
                
                QVe voy-je? Eſt-ce Hermionne? Et que
 viens-je d’entendre?&
       Pour qui coule le ſang que ie viens de répandre?&
       Ie ſuis, ſi ie l’en crois, vn
 Traiſtre, vn Aſſaſſin.&
       Eſt-ce Pyrrhus qui meurt? \ampersand\ ſuis-je
 Oreſte enfin?&
       Quoy? j’étouffe en mon cœur la raiſon qui m’é-claire.&
       I’aſſaſſine à regret vn Roy que ie reuere.&
       Ie viole en vn
 jour les droits des Souuerains,&
       Ceux des Ambaſſadeurs, \ampersand\ tous ceux des Humains,&
       Ceux meſme des Autels, où ma
 fureur l’aſſiege.&
       Ie deuiens
 Parricide, Aſſaſſin, Sacrilege.&
       Pour qui? Pour vne Ingrate, à
 qui ie le promets,&
       Qui meſme, s’il ne meurt, ne me
 verra jamais,&
       Dont j’épouſe la rage. Et quand
 ie l’ay ſeruie,&
       Elle me redemande \ampersand\ ſon ſang \ampersand\ ſa vie!&
       Elle l’aime! \ampersand\ ie ſuis un monſtre furieux!&
       Ie la voy pour jamais s’éloigner
 de mes yeux,&
       Et l’Ingrate, en fuyant, me laiſſe pour ſalaire&
       Tous les noms odieux que j’ay pris pour luyplaireluy plaire.\&
       
                      
\stanza[\section{SCENE V.}
\personscene{ORESTE, PYLADE, Soldats d'Oreſte.}
\enonciateur{PYLADE.}
                ]
                
                IL faut partir, Seigneur.
 Sortons de ce Palais,&
       Ou bien reſoluons nous de n’en
 ſortir jamais.&
       Nos Grecs pour vn moment en
 défendent la Porte.&
       Tout le Peuple aſſemblé nous pourſuit à main for-te.&
       Aux ordres d’Andromaque icy tout eſt ſoûmis,&
       Ils la traitent en Reyne, \ampersand\ nous comme
 Ennemis.&
       Andromaque
 elle meſme à Pyrrhusſi rebelle,&
       Luy rend tous les deuoirs d’vne Veuve fidelle,&
       Commande qu’on le vange. Et peut-eſtre qu’en-cor,&
       Elle pourſuit ſur nous la vangeance d’Hector.&
       Allons. N’attendons pas que l’on nous enuiron-ne,&
       Nos Grecs nous rejoindront. Et tandis
 qu’Her-mionne&
       Tient encore le Peuple autour d’elle arreſté,&
       Voila noſtre chemin, ſortons en ſeureté.\&
       
\stanza[
\enonciateur{ORESTE.}
]
                
                Non, non, c’eſt Hermionne, Amis, que ie
 veux ſuiure,&
       A ſon dernier courroux, ie ne puis plus ſuruiure:&
       Partez, j’ay fait le crime, \ampersand\ ie vais l’expier.\&
       
\stanza[
\enonciateur{PYLADE.}
]
                
                Hermionne,
 Seigneur? Il la faut oublier.&
       Quoy touſiours l’inſtrument, \ampersand\ l’objet de ſa ra-ge,&
       Cherchez-vous chez les Morts quelque nouuel ou-trage?&
       Et parce qu’elle meurt, faut-il que vous
 mouriez?\&
       
\stanza[
\enonciateur{ORESTE.}
]
                
                Elle meurt? Dieux. Qu’entens-je!\&
       
\stanza[
\enonciateur{PYLADE.}
]
                
                \antilabe Hé quoy? vous l’ignoriez?&
       En rentrant dans ces lieux nous l’auons rencon-trée,&
       Qui couroit vers le Temple, inquiete, égarée.&
       Elle a trouué Pyrrhus, porté ſur des
 Soldats,&
       Que ſon ſang
 excitoit à vanger ſon trépas.&
       Sans doute, à cét objet ſa rage
 s’eſt émuë.&
       Mais du haut de la porte enfin nous l’auons veuë,&
       Vn poignard à la main, ſur Pyrrhusſe courber,&
       Leuer les yeux au Ciel, ſe frapper, \ampersand\ tomber.\&
       
\stanza[
\enonciateur{ORESTE.}
]
                
                Grace aux Dieux! Mon mal-heur paſſe mon eſperance.&
       Oüy, ie te louë, ô Ciel! de ta
 perſeuerance.&
       Appliqué ſans relaſche au ſoin de me punir,&
       Au comble des douleurs tu m’as fait parue-nir.&
       Ta haine a pris plaiſir à former ma miſere,&
       I’eſtois né,
 pour ſeruir d’exemple à ta colere,&
       Pour eſtre du Mal-heur vn modelle accomply;&
       Hé bien, ie meurs content, \ampersand\
 mon ſort eſt rem-ply.&
       Où ſont ces deux Amans? Pour
 couronner ma joye,&
       Dans leur ſang, dans le mien, il
 faut que ie me noye:&
       L’vn \ampersand\ l’autre en mourant,
 ie les veux regarder.&
       Réüniſſons
 trois cœurs, qui n’ont pû s’accorder,&
       Mais quelle épaiſſe nuit tout à coup m’enuironne?&
       De quel coſté ſortir? D’où vient que ie friſſonne?&
       Quelle horeur me ſaiſit? Grace au Ciel, j’entreuoy.&
       Dieux! Quels ruiſſeaux de ſang coulent autour de moy!\&
       
\stanza[
\enonciateur{PYLADE.}
]
                
                Ah! Seigneur!\&
       
\stanza[
\enonciateur{ORESTE.}
]
                
                \antilabe Quoy, Pyrrhus, ie te rencontre encore?&
       Trouueray-je par tout vn Riual que j’abhorre?&
       Percé de tant de coups comment t’es-tu ſauué?&
       Tiens, tiens, voila le coup que ie t’ay reſerué.&
       Mais que vois-je? A mes yeux Hermionne l’em-braſſe?&
       Elle vient l’arracher au coup qui le menaſſe?&
       Dieux, quels affreux regards elle jette ſur moy!&
       Quels Démons, quels ſerpens
 traiſne-t’elle apres ſoy?&
       Hé bien, Filles d’Enfer, vos mains ſont-elles pre-ſtes?&
       Pour qui ſont ces Serpens qui ſifflent ſur vos teſtes?&
       A qui deſtinez-vous l’appareil
 qui vous ſuit?&
       Venez-vous m’enleuer dans
 l’eternelle Nuit?&
       Venez, à vos fureurs Oreſte s’abandonne.&
       Mais non, retirez-vous, laiſſez faire Hermionne;&
       L’Ingrate mieux que vous ſçaura
 me déchirer,&
       Et ie luy porte enfin mon cœur à
 deuorer.\&
       
\stanza[
\enonciateur{PYLADE.}
]
                
                Il perd le ſentiment. Amis, le  temps nous preſſe,&
       Ménageons les momens que ce tranſport nous laiſſe.&
       Sauuons-le. Nos efforts deuiendroient impuiſſans,&
       S’il reprenoit icy ſa rage auec ſes ſens.\&
       \endnumbering
\end{Leftside}
\begin{Rightside}
\beginnumbering\stanza[\chapter{ACTE PREMIER.}
\section{SCENE PREMIERE.} 
    \personscene{ORESTE, PYLADE.}  
    \enonciateur{ORESTE.} 
    ]
    
    O VY, puis que je retrouve un Amy si fidelle,&
       Ma Fortune va prendre une face nou-uelle;&
       Et déja son courroux semble s'estre adouci,&
       Depuis qu'elle a pris soin de nous
 rejoindre ici.&
       Qui m'eust
 dit, qu'un rivage à mes vœux si funeste,&
       Présenteroit d'abord Pylade aux yeux d'Oreste,&
       Qu'apres plus de six mois que je t'avois perdu,&
       A la Cour de Pyrrhus tu me serois rendu!\&
       
\stanza[
\enonciateur{PYLADE.}
]
                
                I'en rends graces au Ciel, qui m'arrestant sans cesse,&
       Sembloit m'avoir fermé le chemin de
 la Gréce,&
       Depuis le jour fatal que la fureur des Eaux,&
       Presque aux yeux de Mycéne, écarta nos Vaisseaux.&
       Combien dans cét exil ay-je souffert
 d'allarmes?&
       Combien à vos malheurs ay-je donné de larmes?&
       Craignant toûjours pour vous quelque nouveau danger&
       Que ma triste Amitié ne pouvoit partager.&
       Sur tout je redoutois cette
 Mélancolie&
       Où j'ay veu si long-temps vostre Ame ensevelie.&
       Je craignois que le Ciel, par un cruel secours,&
       Ne vous offrît la mort, que vous cherchiez
 toûjours.&
       Mais je vous voy, Seigneur, \ampersand\ si j'ose le dire,&
       Un Destin plus
 heureux vous conduit en Epire.&
       Le pompeux Appareil qui suit icy vos
 pas,&
       N'est point d'un
 Malheureux qui cherche le trépas.\&
       
\stanza[
\enonciateur{ORESTE.}
]
                
                Helas! qui peut sçavoir le Destin qui m'ameine?&
       L'Amour me fait icy chercher une
 Inhumaine.&
       Mais qui sçait ce qu'il doit
 ordonner de mon Sort,&
       Et si je viens
 chercher, ou la vie, ou la mort?\&
       
\stanza[
\enonciateur{PYLADE.}
]
                
                Quoy! vostre Ame à l'Amour, en Esclave asseruie,&
       Se repose sur luy
 du soin de vostre vie?&
       Par quels charmes, apres
 tant de tourmens soufferts&
       Peut-il vous inuiter à rentrer dans ses fers?&
       Pensez-vous qu'Hermionne, à Sparte inéxorable,&
       Vous prépare en Epireun Sort plus favorable?&
       Honteux d'avoir poussé tant de vœux superflus,&
       Vous l'abhorriez. Enfin, vous ne m'en parliez
 plus.&
       Vous me trompiez, Seigneur.\&
       
\stanza[
\enonciateur{ORESTE.}
]
                
                \antilabe Je me trompois moy-méme.&
       Amy, n'insulte point un Malheureux
 qui t'aime.&
       T'ay-je iamais caché mon cœur \ampersand\ mes desirs?&
       Tu vis naistre ma flâme \ampersand\ mes
 premiers soûpirs.&
       Enfin, quand \edgls{Ménélas} disposa de sa Fille&
       En faveur de Pyrrhus, vangeur de sa
 Famille;&
       Tu vis mon desespoir, \ampersand\ tu m’as veu depuis&
       Traîner de Mers en Mers ma chaîne \ampersand\ mes
 ennuis.&
       Je te vis à regret, en cét estat funeste,&
       Prest à suiure par
 tout le déplorable Oreste,&
       Toûjours de ma fureur interrompre le cours,&
       Et de moy-mesme enfin me sauver tous les jours.&
       Mais quand je me souvins, que parmy tant d’al-larmes&
       Hermionne à
 Pyrrhus prodiguoit tous ses charmes,&
       Tu sçais de quel courroux mon cœur
 alors épris&
       Voulut, en l’oubliant, vanger\footnoteA{Cf. Subligny, \emph{La folle querelle}, \href{https://gallica.bnf.fr/ark:/12148/bpt6k111119r/f17.item}{préface}.} tous ses mépris.&
       Je fis croire, \ampersand\ je crûs ma victoire certaine.&
       Je pris tous mes transports pour des transports de haine;&
       Détestant ses
 rigueurs, rabaissant ses attraits,&
       Je défiois ses yeux
 de me troubler iamais.&
       Voila comme je crûs étouffer ma
 tendresse.&
       Dans ce
 calme trompeur j’arrivay dans la Gréce;&
       Et je trouvay
 d’abord ses Princes rassemblez,&
       Qu’un péril assez grand sembloit avoir troublez.&
       J’y courus. Je
 pensay que la Guerre, \ampersand\ la Gloire,&
       De soins plus importans rempliroient
 ma memoire;&
       Que mes sens reprenant leur premiere
 vigueur,&
       L’Amour acheveroit de sortir de mon Cœur.&
       Mais admire avec moy le Sort, dont
 la poursuite&
       Me fait courir moy-mesme\footnoteA{Cf. Subligny, \emph{La folle
 querelle}, \href{https://gallica.bnf.fr/ark:/12148/bpt6k111119r/f153.item}{p.126}.} au piege que j’éuite.&
       J’entens de tous costez qu’on menace Pyrrhus.&
       Toute la Gréce éclate en murmures confus.&
       On se plaint, qu’oubliant son Sang, \ampersand\ sa promesse,&
       Il éleve en sa Cour
 l’Ennemy de la Gréce,&
       \edgls{Astyanax}, d'\edgls{Hector}
 jeune \ampersand\ malheureux Fils,&
       Reste de tant de Roys sous Troye ensevelis.&
       J’apprens, que pour ravir son enfance au Suplice,&
       Andromaque
 trompa l’ingénieux \edgls{Ulysse},&
       Tandis qu’un autre Enfant arraché de
 ses bras,&
       Sous le nom de son Fils, fut conduit au trépas.&
       On dit, que peu sensible aux charmes d’Hermionne,&
       Mon Rival porte ailleurs son Cœur \ampersand\ sa Couronne;&
       \edgls{Ménélas}, sans le croire, en paroist affligé,&
       Et se plaint d’un
 Hymen si long-temps negligé.&
       Parmy les déplaisirs où son Ame se noye,&
       Il s’éleve en la mienne une secrette joye.&
       Je triomphe; \ampersand\ pourtant je me flate d’abord&
       Que la seule vengeance excite ce
 transport.&
       Mais l’Ingrate en on Cœur
 reprit bientost sa place,&
       De mes feux mal éteints je reconnus
 la trace,&
       Je sentis que ma
 haine alloit finir son cours,&
       Ou plûtost je sentis que je l’aimois toûjours.&
       Ainsi de tous les Grecs je brigue le suffrage.&
       On m’enuoye à Pyrrhus. I’entreprens ce voyage.&
       Je viens voir si
 l’on peut arracher de ses bras&
       Cét Enfant, dont la vie allarme tant d’Estats.&
       Heureux, si je
 pouvois dans l’ardeur qui me presse,&
       Au lieu d’\edgls{Astyanax}, luy ravir ma Princesse.&
       Car enfin n’attens pas que mes feux redoublez,&
       Des périls les plus grands, puissent estre troublez.&
       Puis qu’apres tant d’efforts ma resistance est vaine,&
       Je me liure en aveugle au transport qui m’entraîne,&
       J’aime, je viens
 chercher Hermionne en ces lieux,&
       La fléchir, l’enlever, ou mourir à
 ses yeux.&
       Toy qui connois Pyrrhus, que penses-tu qu’il fasse?&
       Dans sa Cour, dans son Cœur, dy-moy ce qui se passe.&
       Mon Hermionne
 encor le tient-elle asseruy?&
       Me rendra-t'il, Pylade, un Cœurqu’il m’a rauy?\&
       
\stanza[
\enonciateur{PYLADE.}
]
                
                Je vous abuserois, si j’osois vous promettre&
       Qu’entre vos mains, Seigneur, il voulut la
 remettre.&
       Non, que de sa Conqueste il paroisse flaté.&
       Pour la Veuve d’\edgls{Hector}ses feux ont éclaté.&
       Il l’aime. Mais enfin cette Veuve
 inhumaine&
       N’a payé jusqu’icy son amour que
 de haine,&
       Et chaque jour encore on luy voit tout tenter,&
       Pour fléchir sa Captive, ou pour
 l’épouvanter.&
       Il luy cache son Fils, il menasse sa teste,&
       Et fait couler des pleurs, qu’aussi-tost il arreste.&
       Hermionne
 elle-mesme a veu plus de cent fois&
       Cet Amant irrité revenir sous ses loix,&
       Et de ses vœux troublez luy
 rapportant l’hommage,&
       Soûpirer à ses pieds moins
 d’amour, que de rage.&
       Ainsi n’attendez pas, que l’on
 puisse aujourd’huy&
       Vous répondre d’un Cœur, si peu maistre de luy.&
       Il peut, Seigneur, il peut dans ce desordre extré-me,&
       Epouser ce qu’il hait, \ampersand\
 perdre ce qu’il
 aime.\&
       
\stanza[
\enonciateur{ORESTE.}
]
                
                Mais dy-moy, de quel œil Hermionne peut voir&
       Ses attraits offensez, \ampersand\ ses yeux sans pouvoir.\footnoteA{Cf. Subligny, \emph{La
 folle querelle}, \href{https://gallica.bnf.fr/ark:/12148/bpt6k111119r/f158.item}{p.131}.}\&
       
\stanza[
\enonciateur{PYLDADE.}
]
                
                Hermionne,
 Seigneur, au moins en apparance,&
       Semble de son Amant dédaigner
 l’inconstance,&
       Et croit que trop heureux d’appaiser\footnoteA{Cf. Subligny, \emph{La
 folle querelle}, \href{https://gallica.bnf.fr/ark:/12148/bpt6k111119r/f17.item}{préface}.}sa rigueur,&
       Il la viendra presser de reprendre son Cœur.&
       Mais je l’ay veuë enfin me confier
 ses larmes.&
       Elle pleure en secret le mépris de
 ses charmes.&
       Toûjours preste à partir, \ampersand\
 demeurant toûjours,&
       Quelquefois elle appelle Oreste à son secours.\&
       
\stanza[
\enonciateur{ORESTE.}
]
                
                Ah! si je le
 croyois, j’irois bientost, Pylade,&
       Me jetter....\&
       
\stanza[
\enonciateur{PYLADE.}
]
                
                \antilabe Achevez, Seigneur, vostre Ambassade.&
       Vous attendez le Roy. Parlez, \ampersand\ luy montrez&
       Contre le Fils d’Hector tous
 les Grecs conjurez.&
       Loin de leur accorder ce Fils de sa Maistresse,&
       Leur haine ne fera qu’irriter sa
 tendresse.&
       Plus on les veut broüiller, plus on va les unir.&
       Pressez. Demandez
 tout, pour ne rien obtenir.&
       Il vient.\&
       
\stanza[
\enonciateur{ORESTE.}
]
                
                \antilabe Hé bien, va donc disposer la Cruelle&
       A revoir un Amant
 qui ne vient que pour elle.\&
       
                      
\stanza[\section{SCENE II.}
\personscene{PYRRHVS, ORESTE, PHOENIX.}
\enonciateur{ORESTE.}
                ]
                
                AVant que tous les Grecs
 vous parlent par ma voix,&
       Souffrez que ie me flate en
 secret de leur choix,&
       Et qu’à vos yeux, Seigneur, je montre quelque
 joye&
       De voir le Fils d’\edgls{Achille}, \ampersand\ le Vainqueur de
 Troye.&
       Oüy: Comme ses exploits, nous
 admirons vos coups;&
       Hector tomba sous luy; Troye expira sous vous;&
       Et vous avez montré, par une heureuse audace,&
       Que le Fils seul d’\edgls{Achille} a pû remplir sa
 place.&
       Mais ce qu’il n’eust point fait,
 la Gréce avec douleur&
       Vous voit du Sang Troyen relever
 le malheur,&
       Et vous laissant
 toucher d’une pitié funeste,&
       D’une Guerre si
 longue entretenir le reste.&
       Ne vous souvient-il plus, Seigneur, quel fut \edgls{Hector}?&
       Nos peuples affoiblis s’en souviennent encor.&
       Son nom seul fait frémir nos Veuves, \ampersand\ nos Filles,&
       Et dans toute la Gréce, il
 n’est point de Familles,&
       Qui ne demandent conte à ce malheureux Fils,&
       D’un Pere, ou d’un Epoux, qu’\edgls{Hector} leur a ravis.&
       Et qui sçait ce qu’un jour ce Fils peut entreprendre?&
       Peut-estre dans nos Ports nous le
 verrons descendre,&
       Tel qu’on a veu son Pere embrazer
 nos Vaisseaux,&
       Et la flâme à la main, les suiure
 sur les Eaux.&
       Oseray-je, Seigneur, dire ce que
 je pense?&
       Vous-mesme de vos soins craignez la recom-pense,&
       Et que dans vostre sein ce Serpent éleué&
       Ne vous punisse
 un jour de l’avoir conserué.&
       Enfin, de tous les Grecs satisfaites l’enuie,&
       Assurez leur
 vangeance, assurez vostre vie.&
       Perdez un Ennemy d’autant plus
 dangereux,&
       Qu’il s’essayra
 sur vous à combattre contre eux.\&
       
\stanza[
\enonciateur{PYRRHVS.}
]
                
                La Gréce en ma faveur est trop inquiétée.&
       De soins plus importans je l’ay cruë agitée,&
       Seigneur, \ampersand\ sur le nom de son Ambassadeur,&
       J’avois dans ses projets conceu plus de grandeur.&
       Qui croiroit en effet, qu’une
 telle entreprise&
       Du Fils d’\edgls{Agamemnon} meritast l’entremise,&
       Qu’un Peuple tout entier, tant de
 fois triom-phant,&
       N’eust daigné conspirer que la mort d’un Enfant?&
       Mais à qui pretend-on que je le sacrifie?&
       La Gréce a-t'elle encor quelque droit sur sa vie?&
       Et seul de tous les Grecs ne m’est-il pas permis&
       D’ordonner des
 Captifs que le Sort m’a soûmis?&
       Oüy, Seigneur, lors qu’au pied des murs fumans de
 Troye,&
       Les Vainqueurs tout sanglans
 partagerent leur Proye,&
       Le Sort, dont les Arrests furent
 alors suivis,&
       Fit tomber en mes mains Andromaque \ampersand\ son Fils.&
       \edgls{Hécube}, pres
 d’\edgls{Ulysse}, acheva sa misere;&
       \edgls{Cassandre}, dans Argos, a suiuy vostre
 Pere.&
       Sur eux, sur leurs Captifs, ay-je
 étendu mes droicts?&
       Ay-je enfin disposé du fruit de leurs Exploits?&
       On craint, qu’avec \edgls{Hector}Troyeun jour ne re-naisse:&
       Son Fils peut me ravir le jour que
 je luy laisse:&
       Seigneur, tant de prudence entraisne trop de soin.&
       Je ne sçay point
 préuoir les malheurs de si loin.&
       Je songe quelle
 estoit autrefois cette Ville,&
       Si superbe en Rampars, en Héros si fertile,&
       Maistresse de l’Asie, \ampersand\ je regarde
 enfin&
       Quel fut le Sort de Troye, \ampersand\ quel est son Destin.&
       Je ne voy que des Tours, que la
 cendre a couvertes,&
       Un fleuve teint
 de sang, des Campagnes desertes,&
       Un Enfant dans les fers, \ampersand\ je
 ne puis songer&
       Que Troye en
 cet estat aspire à se
 vanger.&
       Ah! si du Fils d’\edgls{Hector} la perte estoit
 jurée,&
       Pourquoy d’un an entier l’avons-nous differée?&
       Dans le sein de \edgls{Priam} n’a-t'on pû l’immoler?&
       Sous tant de Morts, sous
 Troye, il falloit l’accabler.&
       Tout estoit juste
 alors. La Vieillesse \ampersand\
 l’Enfance&
       En vain sur leur foiblesse appuyoient leur defance.&
       La Victoire, \ampersand\ la Nuit, plus cruelles que
 nous,&
       Nous excitoient au meurtre, \ampersand\ confondoient nos
 coups.&
       Mon courroux aux Vaincus ne fut que trop severe.&
       Mais que ma Cruauté suruive à ma Colere?&
       Que malgré la pitié dont je me sens saisir,&
       Dans le sang d’un
 Enfant je me baigne à loisir?&
       Non, Seigneur. Que les Grecs cherchent quelque
 autre Proye,&
       Qu’ils poursuivent ailleurs ce qui reste de Troye,&
       De mes inimitiez le cours est
 acheué,&
       L’Epire sauvera
 ce que Troye a sauvé.\&
       
\stanza[
\enonciateur{ORESTE.}
]
                
                Seigneur, vous sçavez trop, avec quel artifice&
       Un faux \edgls{Astyanax} fut offert au Suplice&
       Où le seul Fils d’\edgls{Hector} devoit estre conduit.&
       Ce n’est pas les Troyens, c’est \edgls{Hector} qu’on
 pour-suit.&
       Oüy, les Grecs sur le Fils persecutent le Pere.&
       Il a par trop de sang acheté leur
 colere.&
       Ce n’est que dans le sien qu’elle peut expirer,&
       Et jusques dans l’Epire il les
 peut attirer.&
       Préuenez les.\&
       
\stanza[
\enonciateur{PVRRHVSPYRRHVS}
]
                
                \antilabe Non, non. I’y consens avec joye.&
       Qu’ils cherchent dans l’Epire une
 seconde Troye.&
       Qu’ils confondent leur haine, \ampersand\ ne distinguent plus&
       Le sang qui les fit vaincre, \ampersand\
 celuy des Vaincus.&
       Aussi-bien ce
 n’est pas la premiere injustice,&
       Dont la Gréce, d’\edgls{Achille} a payé le seruice.&
       \edgls{Hector} en
 profita, Seigneur, \ampersand\ quelque jour&
       Son Fils en pourroit bien profiter à son tour.\&
       
\stanza[
\enonciateur{ORESTE.}
]
                
                Ainsi la Gréce en vous trouve un Enfant rebelle?\&
       
\stanza[
\enonciateur{PYRRHVS.}
]
                
                Et je n’ay donc vaincu que pour
 dépendre d’elle?\&
       
\stanza[
\enonciateur{ORESTE.}
]
                
                Hermionne,
 Seigneur, arrestera vos coups;&
       Ses yeux s’opposeront entre son Pere \ampersand\ vous.\&
       
\stanza[
\enonciateur{PYRRHVS.}
]
                
                Hermionne,
 Seigneur, peut m’estre toûjours chere,&
       Je puis l’aimer, sans estre Esclave de son Pere.&
       Et je sçauray
 peut-estre accorder en ce
 jour&
       Les soins de ma grandeur, \ampersand\
 ceux de mon amour.&
       Vous pouvez cependant voir la
 Fille d’\edgls{Hélène}.&
       Du sang qui vous unit je sçay l’étroite
 chaîne.&
       Apres cela, Seigneur, je ne vous
 retiens plus,&
       Et vous pourrez aux Grecs annoncer mon refus.\&
       
                      
\stanza[\section{SCENE III.}
\personscene{PYRRHVS, PHOENIX.}
\enonciateur{PHOENIX.}
                ]
                
                AInsi vous
 l’enuoyez aux pieds de sa Maistresse?\&
       
\stanza[
\enonciateur{PYRRHVS.}
]
                
                On dit qu’il a long-temps brûlé pour la Princesse.\&
       
\stanza[
\enonciateur{PHOENIX.}
]
                
                Mais si ce feu, Seigneur, vient à
 se rallumer,&
       S’il luy rendoit son Cœur, s’il
 s’en faisoit aimer?\&
       
\stanza[
\enonciateur{PYRRHVS.}
]
                
                Ah! qu’ils s’aiment, Phœnix, j’y
 consens. Qu’elle parte.&
       Que charmez l’un de l’autre, ils
 retournent à Sparte.&
       Tous nos Ports sont ouverts \ampersand\ pour elle \ampersand\ pour luy.&
       Qu’elle m’épargneroit de contrainte \ampersand\
 d’ennuy!\&
       
\stanza[
\enonciateur{PHOENIX.}
]
                
                Seigneur....\&
       
\stanza[
\enonciateur{PYRRHVS.}
]
                
                \antilabe Une autre fois je
 t’ouvriray mon Ame,&
       Andromaque paroist.\&
       
                      
\stanza[\section{SCENE IV.}
\personscene{PYRRHVS, ANDROMAQVE, CEPHISE.}
\enonciateur{PYRRHVS.}
                ]
                
                \antilabe ME cherchiez-vous, Madame?&
       Un espoir si charmant me seroit-il permis?\&
       
\stanza[
\enonciateur{ANDROMAQVE.}
]
                
                Je passois jusqu’aux lieux, où l’on garde mon
 Fils.&
       Puis qu’une fois le jour vous souffrez que je voye&
       Le seul bien qui me reste, \ampersand\ d’Hector \ampersand\ de Troye,&
       J’allois, Seigneur, pleurer un moment avec luy,&
       Je ne l’ay point encore embrassé d’aujourd’huy.\&
       
\stanza[
\enonciateur{PYRRHVS.}
]
                
                Ah, Madame! les Grecs, si j’en croy leurs allar-mes,&
       Vous donneront bientost d’autres
 sujets de larmes.\&
       
\stanza[
\enonciateur{ANDROMAQVE.}
]
                
                Et quelle est cette peur dont leur
 Cœur est frappé,&
       Seigneur? Quelque Troyen vous est-il échappé?\&
       
\stanza[
\enonciateur{PYRRHVS.}
]
                
                Leur haine pour \edgls{Hector} n’est pas encore éteinte.&
       Ils redoutent son
 Fils.\&
       
\stanza[
\enonciateur{ANDROMAQVE.}
]
                
                \antilabe Digne Objet de leur crainte!&
       Un Enfant malheureux, qui ne sçait pas encor&
       Que Pyrrhus est son Maistre, \ampersand\
 qu’il est Fils d’\edgls{Hector}.\&
       
\stanza[
\enonciateur{PYRRHVS.}
]
                
                Tel qu’il est, tous les Grecs
 demandent qu’il perisse.&
       Le Fils d’\edgls{Agamemnon} vient haster son suplice.\&
       
\stanza[
\enonciateur{ANDROMAQVE.}
]
                
                Et vous prononcerez un Arrest si cruel?&
       Est-ce mon interest qui le rend criminel?&
       Helas! on ne craint point qu’il vange un jour son Pere.&
       On craint qu’il n’essuyast les larmes de sa
 Mere.&
       Il m’auroit tenu lieu d’un Pere,
 \ampersand\ d’un Epoux,&
       Mais il me faut tout perdre, \ampersand\ toûjours par vos
 coups.\&
       
\stanza[
\enonciateur{PYRRHVS.}
]
                
                Madame, mes refus ont préuenu vos larmes.&
       Tous les Grecs m’ont déja menassé de leurs Armes.&
       Mais dûssent-ils
 encore, en repassant les Eaux,&
       Demander vostre Fils, avec mille Vaisseaux:&
       Coustast- il tout le sang
 qu’\edgls{Hélène} a fait répandre,&
       Dûssay-je apres
 dix ans voir mon Palais en cendre,&
       Je ne balance point, je vole à son secours,&
       Je defendray sa
 vie aux despens de mes jours.&
       Mais parmy ces perils, où je cours
 pour vous plaire,&
       Me refuserez-vous un regard moins severe?&
       Haï de tous les Grecs, pressé de tous costez,&
       Me faudra-t'il combattre encor vos cruautez?&
       Je vous offre mon Bras. Puis-je
 esperer encore&
       Que vous accepterez un Cœur qui
 vous adore?&
       En combattant pour vous, me sera-t’il permis&
       De ne vous point conter parmy mes Ennemis?\&
       
\stanza[
\enonciateur{ANDROMAQVE.}
]
                
                Seigneur, que faites-vous, \ampersand\ que dira la
 Gréce?&
       Faut-il qu’un si
 grand Cœur montre tant de foi-blesse?&
       Voulez-vous qu’un dessein si beau, si genereux,&
       Passe pour le
 transport d’un Esprit amoureux?&
       Captive, toûjours triste, importune à moy-méme,&
       Pouvez-vous souhaiter qu’Andromaque vous
 aime?&
       Que feriez-vous, helas! d’un Cœur
 infortuné&
       Qu’à des pleurs éternels vous avez condamné?&
       Non, non, d’un Ennemy respecter la Misere,&
       Sauver des Malheureux, rendre un Fils à sa Mere,&
       De cent Peuples pour luy combattre la rigueur,&
       Sans me faire payer son salut de mon Cœur,&
       Malgré moy, s’il le faut, luy donner un azile,&
       Seigneur, voila des soins dignes
 du Fils d’\edgls{Achille}.\&
       
\stanza[
\enonciateur{PYRRHVS.}
]
                
                Hé quoy? Vostre courroux
 n’a-t’il pas eû son cours?&
       Peut-on haïr sans cesse? Et punit-on toûjours?&
       J’ay fait des Malheureux, sans doute, \ampersand\ la Phrygie&
       Cent fois de vostre sang a veu ma main rougie.&
       Mais que vos yeux sur moy se sont bien exercez!&
       Qu’ils m’ont vendu bien cher les pleurs qu’ils ont
 versez!&
       De combien de remords m’ont-ils rendu la Proye?&
       Je souffre tous les maux que j’ay faits devant Troye.&
       Vaincu, chargé de fers, de regrets consumé,&
       Brûlé de plus de feux que je n’en
 allumé,&
       Tant de soins, tant de pleurs,
 tant d’ardeurs in-quiétes....&
       Helas! fus-je iamais si cruel que
 vous l’estes?&
       Mais enfin, tour à tour, c’est assez nous punir.&
       Nos Ennemis communs déuroient nous reünir.&
       Madame, dites-moy seulement que j’espere,&
       Je vous rens vostre Fils, \ampersand\ je luy sers de
 Pere.&
       Je l’instruiray
 moy-mesme à vanger les Troyens.&
       J’iray punir les Grecs de vos maux
 \ampersand\ des miens.&
       Animé d’un regard, je puis tout entreprendre.&
       Vostre Ilion encor peut sortir de sa cendre.&
       Je puis, en moins de temps que les Grecs ne l’ont pris,&
       Dans ses Murs relevez couronner vostre Fils.\&
       
\stanza[
\enonciateur{ANDROMAQVE.}
]
                
                Seigneur, tant de grandeurs ne nous touchent plus
 guére,&
       Je les luy promettois tant qu’a
 vescu son Pere.&
       Non, vous n’esperez plus de nous
 revoir encor,&
       Sacrez Murs, que n’a pû conseruer
 mon \edgls{Hector}.&
       A de moindres faveurs des
 Malheureux prétendent,&
       Seigneur. C’est un Exil que mes pleurs vous de-mandent.&
       Souffrez que loin des Grecs, \ampersand\ mesme loin de vous,&
       J’aille cacher mon Fils, \ampersand\
 pleurer mon Epoux.&
       Vostre amour contre nous allume
 trop de haine.&
       Retournez, retournez à la Fille d’\edgls{Hélène}.\&
       
\stanza[
\enonciateur{PYRRHVS.}
]
                
                Et le puis-je, Madame? Ah, que vous me gesnez!&
       Comment luy rendre un Cœur que vous me retenez?&
       Je sçay que de
 mes veux on luy promit l’empire.&
       Je sçay que pour
 regner elle vint dans l’Epire.&
       Le Sort vous y voulut l’une \ampersand\
 l’autre amener,&
       Vous pour porter des fers, Elle pour en donner.&
       Cependant ay-je pris quelque soin
 de luy plaire?&
       Et ne diroit-on pas, en voyant au contraire,&
       Vos charmes tout-puissans, \ampersand\ les siens dédaignez,&
       Qu’elle est icy Captive, \ampersand\ que vous y regnez?&
       Ah! qu’un seul
 des soûpirs, que mon Cœur vous enuoye,&
       S’il s’échapoit vers elle, y porteroit de joye!\&
       
\stanza[
\enonciateur{ANDROMAQVE.}
]
                
                Et pourquoy vos soûpirs seroient-ils repoussez?&
       Auroit-elle oublié vos seruices
 passez?&
       Troye,
 \edgls{Hector}, contre vous revoltent-ils son Ame?&
       Aux cendres d’un Epoux doit-elle
 enfin sa flâme?&
       Et quel Epoux encore! Ah souvenir cruel!&
       Sa mort seule a rendu vostre Pere immortel.&
       Il doit au sang d’\edgls{Hector} tout l’éclat de ses
 armes,&
       Et vous n’estes tous deux connus
 que par mes larmes.\&
       
\stanza[
\enonciateur{PYRRHVS.}
]
                
                Hé bien, Madame, hé bien, il faut vous obeïr.&
       Il faut vous oublier, ou plûtost
 vous haïr.&
       Oüy, mes vœux ont trop loin poussé leur violence,&
       Pour ne plus s’arrester que dans
 l’indifference.&
       Songez-y bien. Il faut desormais
 que mon Cœur,&
       S’il n’aime avec transport, haïsse avec fureur.&
       Je n’épargneray rien dans ma juste colere.&
       Le Fils me répondra des mépris de la Mere,&
       La Gréce le demande, \ampersand\ je ne prétens pas&
       Mettre toûjours ma gloire à sauver des Ingrats.\&
       
\stanza[
\enonciateur{ANDROMAQVE.}
]
                
                Helas! il mourra donc. Il n’a pour sa défense,&
       Que les pleurs de sa Mere, \ampersand\
 que son Innocence.&
       Et peut-estre apres tout, en l’estat où je suis,&
       Sa mort avancera la fin de mes
 ennuis.&
       Je prolongeois pour luy ma vie,
 \ampersand\ ma misere.&
       Mais enfin sur ses pas j’iray revoir son
 Pere.&
       Ainsi tous trois, Seigneur, par
 vos soins reünis,&
       Nos Cœurs…\&
       
\stanza[
\enonciateur{PYRRHVS.}
]
                
                \antilabe Allez, Madame, allez voir vostre
 Fils.&
       Peut-estre, en le voyant, vostre amour plus timide,&
       Ne prendra pas toûjours sa Colere
 pour guide.&
       Pour sçavoir nos
 Destins, j’iray vous retrouver.&
       Madame, en l’embrassant, songez à le sauver.\&
       \stanza[\chapter{ACTE II.}
\section{SCENE PREMIERE.} 
    \personscene{HERMIONNE, CLEONNE.}  
    \enonciateur{HERMIONNE.} 
    ]
    
    I E fay ce que tu veux. Je consens qu’il me voye.&
       Je luy veux bien encore accorder
 cette joye;&
       Pylade va
 bien-tost conduire icy ses
 pas.&
       Mais, si je m’en
 croyois, je ne le verrois pas.\&
       
\stanza[
\enonciateur{CLEONNE.}
]
                
                Et qu’est-ce que sa veuë a pour vous de funeste?&
       Madame, n’est-ce pas toûjours le
 mesme Oreste,&
       Dont vous avez cent fois souhaité le retour,&
       Et dont vous regrettiez la constance \ampersand\ l’amour?\&
       
\stanza[
\enonciateur{HERMIONNE.}
]
                
                C’est cet amour payé de trop
 d’ingratitude,&
       Qui me rend en ces Lieux sa presence si rude.&
       Quelle honte pour moy! Quel triomphe pour luy,&
       De voir mon infortune égaler son
 ennuy!&
       Est-ce là, dira-t’il, cette fiere
 Hermionne?&
       Elle me dédaignoit, un autre
 l’abandonne.&
       L’Ingrate, qui mettoit son Cœur à
 si haut prix,&
       Apprend donc à son tour à souffrir des mépris?&
       Ah Dieux!\&
       
\stanza[
\enonciateur{CLEONNE.}
]
                
                \antilabe Ah! dissipez ces
 indignes allarmes.&
       Il a trop bien senty le pouvoir de vos charmes.&
       Vous croyez qu’un Amant vienne
 vous insulter?&
       Il vous rapporte un Cœur qu’il n’a
 pû vous oster.&
       Mais vous ne dites point ce que vous mande un Pere.\&
       
\stanza[
\enonciateur{HERMIONNE.}
]
                
                Dans ses retardemens si Pyrrhus persevere,&
       A la mort du Troyen s’il ne veut consentir,&
       Mon Pere avec les Grecs m’ordonne
 de partir.\&
       
\stanza[
\enonciateur{CLEONNE.}
]
                
                Hé bien, Madame, hé bien, écoutez donc Oreste.&
       Pyrrhus a
 commencé, faites au moins le reste.&
       Pour bien faire, il faudroit que vous le préuinssiez.&
       Ne m’avez-vous pas dit que vous le
 haïssiez?\&
       
\stanza[
\enonciateur{HERMIONNE.}
]
                
                Si je le hais Cleonne? Il y va de ma gloire,&
       Apres tant de bontez dont il perd la memoire.&
       Luy qui me fut si cher, \ampersand\ qui
 m’a pû trahir?&
       Ah! je l’ay trop aimé, pour ne le
 point haïr.\&
       
\stanza[
\enonciateur{CLEONNE.}
]
                
                Fuyez-le donc, Madame. Et puis qu’on vous
 adore....\&
       
\stanza[
\enonciateur{HERMIONNE.}
]
                
                Ah! laisse à ma
 fureur le temps de croistre encore.&
       Contre mon Ennemy laisse-moy m’assurer,&
       Cleonne, avec horreur je m’en veux separer.&
       Il n’y travaillera que trop bien,
 l’Infidelle.\&
       
\stanza[
\enonciateur{CLEONNE.}
]
                
                Quoy! vous en attendez quelque injure nouvelle?&
       Aimer une Captive, \ampersand\ l’aimer à vos yeux,&
       Tout cela n’a donc pû vous le rendre odieux?&
       Apres ce qu’il a fait, que sçauroit-il donc faire?&
       Il vous auroit déplû, s’il pouvoit
 vous déplaire.\&
       
\stanza[
\enonciateur{HERMIONNE.}
]
                
                Pourquoy veux-tu, Cruelle, irriter mes ennuis?&
       Je crains de me connoistre, en l’estat où je suis.&
       De tout ce que tu vois tâche de ne rien croire.&
       Croy que je n’aime plus. Vante moy
 ma victoire.&
       Croy que dans son dépit mon Cœur
 est endurcy.&
       Helas! \ampersand\ s’il se peut, fay-le
 moy croire aussy.&
       Tu veux que je le fuye. Hé bien,
 rien ne m’arreste.&
       Allons. N’enuions plus son indigne
 conqueste.&
       Que sur luy sa
 Captive étende son pouvoir.&
       Fuyons. Mais si l’Ingrat rentroit
 dans son de-uoir!&
       Si la Foy dans son Cœur retrouvoit quelque place!&
       S’il venoit à mes pieds me demander sa Grace!&
       Si sous mes Loix, Amour, tu pouvois l’engager!&
       S’il vouloit!… Mais l’Ingrat ne veut que m’outrager.&
       Demeurons toutefois, pour troubler leur
 fortune.&
       Prenons quelque plaisir à leur estre importune.&
       Ou le forçant de rompre un nœud si solemnel,&
       Aux yeux de tous les Grecs rendons-le criminel.&
       J’ay déja sur le
 Fils attiré leur colere.&
       Je veux qu’on vienne encor luy demander la Mere.&
       Rendons-luy les tourmens qu’elle me fait souffrir.&
       Qu’elle le perde, ou bien qu’il la fasse périr.\&
       
\stanza[
\enonciateur{CLEONNE.}
]
                
                Pensez-vous que des yeux toûjours
 ouverts aux larmes,&
       Songent à balancer le pouvoir de
 vos charmes?&
       Et qu’un Cœur accablé de tant de
 déplaisirs,&
       De son Persecuteur ait brigué les soûpirs?&
       Voyez si sa
 douleur en paroist soulagée.&
       Pourquoy don les chagrins
 où son Ame est plongée?&
       Pourquoy tant de froideurs? Pourquoy cette
 fierté?\&
       
\stanza[
\enonciateur{HERMIONNE.}
]
                
                Helas! pour mon malheur je l’ay
 trop écouté.&
       Je n’ay point du silence affecté le mystere.&
       Je croyois sans
 péril pouvoir estre sincere.&
       Et sans armer mes yeux d’un moment de rigueur,&
       Je n’ay pour luy parler, consulté que mon Cœur.&
       Et qui ne se seroit comme moy declarée,&
       Sur la foy d’une amour si saintement jurée?&
       Me voyoit-il de l’œil qu’il me voit
 aujourd’huy?&
       Tu t’en souviens
 encor, tout conspiroit pour luy.&
       Ma Famille vangée, \ampersand\ les Grecs dans la
 joye,&
       Nos Vaisseaux
 tout chargez des dépoüilles de Troye,&
       Les Exploits de son Pere, effacez
 par les siens,&
       Ses feux que je croyois plus
 ardans que les miens,&
       Mon Cœur, toy-mesme enfin de sa gloire ébloüye,&
       Auant qu’il me trahist, vous m’avez tous trahie.&
       Mais c’en est trop, Cleonne, \ampersand\ quel que soit
 Pyr-rhus,&
       Hermionne est sensible, Oreste a des vertus.&
       Il sçait aimer du moins, \ampersand\
 mesme sans qu’on l’aime;&
       Et peut-estre il sçaura se faire aimer luy-méme.&
       Allons. Qu’il vienne enfin.\&
       
\stanza[
\enonciateur{CLEONNE.}
]
                
                \antilabe Madame, le voicy.\&
       
\stanza[
\enonciateur{HERMIONNE.}
]
                
                Ah! je ne croyois pas qu’il fust si prés d’icy.\&
       
                      
\stanza[\section{SCENE II.}
\personscene{HERNMIONNE, ORESTE, CLEONNE.}
\enonciateur{HERMIONNE.}
                ]
                
                LE croiray-je, Seigneur,
 qu’un reste de tendresse&
       Ait suspendu les
 soins dont vous charge la Gréce?&
       Ou ne dois-je imputer qu’à vostre
 seul devoir,&
       L’heureux empressement qui vous porte à me voir?\&
       
\stanza[
\enonciateur{ORESTE.}
]
                
                Tel est de mon amour l’aveuglement funeste.&
       Vous le sçavez,
 Madame, \ampersand\ le destin d’Oreste&
       Est de venir sans
 cesse adorer vos attraits,&
       Et de jurer toûjours qu’il n’y viendra iamais.&
       Je sçay que vos
 regards vont rouvrir mes blessures,&
       Que tous mes pas vers vous sont
 autant de parjures.&
       Je le sçay, j’en rougis. Mais j’atteste les Dieux,&
       Témoins de la fureur de mes derniers adieux,&
       Que j’ay couru par tout, où ma
 perte certaine&
       Dégageoit mes sermens, \ampersand\
 finissoit ma peine.&
       J’ay mandié la Mort, chez des
 Peuples cruels&
       Qui n’apaisoient leurs Dieux que
 du sang des Mor-tels:&
       Ils m’ont fermé leur Temple, \ampersand\ ces Peuples
 barbares&
       De mon sang prodigué sont devenus avares.&
       Enfin je viens à vous, \ampersand\ je me voy reduit&
       A chercher dans vos yeux une mort,
 qui me fuit.&
       Mon desespoir
 n’attend que leur indifference,&
       Ils n’ont qu’à m’interdire un
 reste d’esperance.&
       Ils n’ont, pour avancer cette mort
 où je cours,&
       Qu’à me dire une fois ce qu’ils
 m’ont dit toûjours.&
       Voila depuis un an le seul soin qui m’anime.&
       Madame, c’est à vous de prendre une Victime,&
       Que les Scythes auroient dérobée à vos coups,&
       Si j’en avois
 trouvé d’aussi
 cruels que Vous.\&
       
\stanza[
\enonciateur{HERMIONNE.}
]
                
                Non, non, ne pensez pas
 qu’Hermionne dispose.&
       D’un sang, sur qui la Gréce aujourd’huy se repose&
       Mais vous-mesme, est-ce ainsi que vous executez&
       Les vœux de tant d’Estats que vous
 representez?&
       Faut-il que d’un transport leur Vangeance dépende?&
       Est-ce le sang d’Oreste enfin qu’on vous
 demande?&
       Dégagez-vous des soins dont vous
 estes chargé.\&
       
\stanza[
\enonciateur{ORESTE.}
]
                
                Les refus de Pyrrhus m’ont assez
 dégagé,&
       Madame, il me renuoye, \ampersand\ et quelque autre Puissance&
       Luy fait du Fils d’\edgls{Hector} embrasser la
 défence.\&
       
\stanza[
\enonciateur{HER MIONNHERMIONNE.}
]
                
                L’infidelle!\&
       
\stanza[
\enonciateur{ORESTE.}
]
                
                \antilabe Ainsi donc il ne me reste rien,&
       Qu’à venir prendre icy la place du Troyen:&
       Nous sommes Ennemis, luy
 des Grecs, moy le vostre,&
       Pyrrhus protege
 l’un, \ampersand\ je vous liure
 l’autre.\&
       
\stanza[
\enonciateur{HERMIONNE.}
]
                
                Hé quoy? Dans vos chagrins sans
 raison affermy,&
       Vous croirez-vous toûjours, Seigneur, mon En-nemy?&
       Quelle est cette rigueur tant de
 fois alleguée?&
       J’ay passé dans l’Epire où j’estois releguée.&
       Mon Pere l’ordonnoit. Mais qui sçait si depuis,&
       Je n’ay point en secret partagé vos ennuis?&
       Pensez-vous avoir
 seul éprouvé des allarmes?&
       Que l’Epire iamais n’ait veû
 couler mes larmes?&
       Enfin, qui vous a dit, que malgré mon devoir,&
       Je n’ay pas quelquefois souhaitté de vous voir?\&
       
\stanza[
\enonciateur{ORESTE.}
]
                
                Souhaitté de me voir? Ah divine
 Princesse....&
       Mais de grace, est-ce à moy que ce
 discours s’a-dresse?&
       Ouvrez les yeux. Songez
 qu’Oreste est devant vous,&
       Orestesi long-temps l’objet de leur courroux.\&
       
\stanza[
\enonciateur{HERMIONNE.}
]
                
                Oüy, c’est vous dont l’amour naissant avec leurs
 charmes,&
       Leur apprit le premier le pouvoir
 de leurs armes,&
       Vous que mille vertus me forçoient d’estimer,&
       Vous que j’ay plaint, enfin que je voudrois aimer.\&
       
\stanza[
\enonciateur{ORESTE.}
]
                
                Je vous entens. Tel est mon partage funeste.&
       Le Cœur est pour Pyrrhus, \ampersand\ les vœux pour Oreste.\&
       
\stanza[
\enonciateur{HERMIONNE.}
]
                
                Ah! ne souhaittez-passouhaittez pas le destin de Pyrrhus,&
       Je vous haïrois trop.\&
       
\stanza[
\enonciateur{ORESTE.}
]
                
                \antilabe Vous m’en aimeriez plus.&
       Ah! que vous me verriez d’un
 regard bien con-traire!&
       Vous me voulez aimer, \ampersand\ je ne
 puis vous plaire,&
       Et l’Amour seul alors se faisant obeïr,&
       Vous m’aimeriez, Madame, en me voulant haïr.&
       O dieux! Tant de respects, une amitié si tendre…&
       Que de raisons pour moy, si vous pouviez m’en-tendre!&
       Vous seule pour Pyrrhus disputez
 aujourd’huy,&
       Peut-estre malgré vous, sans doute malgré luy.&
       Car enfin il vous hait. Son ame ailleurs éprise&
       N’a plus…\&
       
\stanza[
\enonciateur{HERMIONNE.}
]
                
                \antilabe Qui vous l’a dit. Seigneur, qu’il me méprise?&
       Ses regards, ses discours vous l’ont-ils donc appris?&
       Iugez vous que ma veuë inspire des
 mépris?&
       Qu’elle allume en un cœur des feux
 si peu durables?&
       Peut-estre d’autres yeux me sont plus favorables.\&
       
\stanza[
\enonciateur{ORESTE.}
]
                
                Poursuivez. Il
 est beau de m’insulter ainsi.&
       Cruelle, c’est donc moy qui vous
 méprise ici.&
       Vos yeux n’ont pas assez éprouvé ma constance.&
       Je suis donc un témoin de leur peu de puissance.&
       Je les ay méprisez? Ah. Qu’ils voudroient bien voir&
       Rival, comme moy, mépriser leur pouvoir.\&
       
\stanza[
\enonciateur{HERMIONNE.}
]
                
                Que m’importe, Seigneur, sa haine,
 ou sa tendresse?&
       Allez contre un Rebelle armer
 toute la Gréce.&
       Rapportez-luy le prix de sa
 rebellion.&
       Qu’on fasse de
 l’Epire un second Ilion.&
       Allez. Apres cela, direz-vous que je l’aime?\&
       
\stanza[
\enonciateur{ORESTE.}
]
                
                Madame, faites plus, \ampersand\ venez-y vous-mesme.&
       Voulez-vous demeurer pour ostage
 en ces lieux?&
       Venez dans tous les cœurs faire parler vos
 yeux.&
       Faisons de nostre
 haine une commune attaque.\&
       
\stanza[
\enonciateur{HERMIONNE.}
]
                
                Mais, Seigneur, cependant s’il épouse Andromaque?\&
       
\stanza[
\enonciateur{ORESTE.}
]
                
                Hé Madame!\&
       
\stanza[
\enonciateur{HERMIONNE.}
]
                
                \antilabe Songez quelle honte pour nous,&
       Si d’une Phrygienne il devenoit l’Espoux.\&
       
\stanza[
\enonciateur{ORESTE.}
]
                
                Et vous le haïssez? Auoüez-le, Madame,&
       L’Amour n’est pas un feu qu’on renferme en une ame.&
       Tout nous trahit, la voix, le silence, les yeux.&
       Et les feux mal couverts n’en
 éclatent que mieux.\&
       
\stanza[
\enonciateur{HERMIONNE.}
]
                
                Seigneur, je le voy bien, vostre ame préuenuë&
       Répand sur mes discours le venin qui la tuë,&
       Toûjours dans mes raisons cherche
 quelque dé-tour,&
       Et croit qu’en moy la haine est un effort d’amour.&
       Il faut donc m’expliquer. Vous agirez en suite.&
       Vous sçavez qu’en
 ces lieux mon devoir m’a con-duite,&
       Mon devoir m’y retient, \ampersand\ je n’en puis partir,&
       Que mon Pere, ou Pyrrhus ne m’en fasse sortir.&
       Au nom de \edgls{Ménélas} allez luy faire entendre,&
       Que l’Ennemy des Grecs ne peut estre son Gendre.&
       Du Troyen, ou de moy, faites le decider:&
       Qu’il songe qui des deux, il veut
 rendre, ou garder.&
       Enfin qu’il me renuoye, ou bien qu’il vous le
 liure.&
       Adieu, s’il y consent, je suis preste à vous
 suiure.\&
       
                      
\stanza[\section{SCENE III.}
\personscene{ORESTE.}
\enonciateur{}
                ]
                
                OVy, oüy, vous me suivrez, n’en doutez nul-lement.&
       Je vous respons
 déja de son consentement.&
       Je ne crains pas, enfin, que
 Pyrrhus la retienne,&
       Il n’a devant les yeux que sa chere Troyenne,&
       Tout autre objet le blesse, \ampersand\ peut-estre au-jourd’huy&
       Il n’attend qu’un prétexte à
 l’éloigner de luy.&
       Nous n’avons qu’à parler. C’en est fait. Quelle joye&
       D’enlever à l’Epire une si belle Proye!&
       Sauve tout ce qui reste, \ampersand\ de Troye, \ampersand\
 d’Hector.&
       Garde son Fils, sa Veuve, \ampersand\ mille autres encor:&
       Epire, c’est assez qu’Hermionne
 renduë&
       Perde à jamais tes bords, \ampersand\ ton Prince de
 veuë.&
       Mais un heureux destin le conduit en ces lieux.&
       Parlons. A tant d’attraits, Amour, ferme ses yeux.\&
       
                      
\stanza[\section{SCENE IV.}
\personscene{PYRRHVS, ORESTE, PHOENIX.}
\enonciateur{PYRRHVS.}
                ]
                
                IE vous cherchois,
 Seigneur. Un peu de violence&
        M’a fait de vos raisons combattre
 la puissance,&
       Je l’avouë. Et
 depuis que je vous ay quitté,&
       J’en ay senty la
 force, \ampersand\ connu l’équité.&
       J’ay songé comme
 vous, qu’à la Grece, à mon Pere,&
       A moy-mesme en un
 mot je devenois contraire,&
       Que je relevois
 Troye, \ampersand\ rendois imparfait&
       Tout ce qu’a fait \edgls{Achille}, \ampersand\ tout ce que j’ay fait.&
       Je ne condamne plus un courroux legitime,&
       Et l’on vous va, Seigneur, liurer vostre Victime.\&
       
\stanza[
\enonciateur{ORESTE.}
]
                
                Seigneur, par ce conseil prudent
 \ampersand\ rigoureux,&
       C’est acheter la Paix du sang d’un Malheureux.\&
       
\stanza[
\enonciateur{PYRRHVS.}
]
                
                Oüy. Mais je veux, Seigneur, l’assurer davantage.&
       D’une eternelle Paix Hermionne est le gage.&
       Je l’espouse. Il sembloit qu’un
 spectacle si doux&
       N’attendist en ces lieux qu’un Tesmoin tel que vous.&
       Vous y représentez tous les Grecs
 \ampersand\ son Pere,&
       Puis qu’en vous \edgls{Ménélas} voit reviure son
 Frere.&
       Voyez-la donc. Allez. Dites-luy que demain&
       J’attens, avec la
 Paix, son cœur de vostre Main.\&
       
\stanza[
\enonciateur{ORESTE.}
]
                
                Ah dieux!\&
       
                      
\stanza[\section{SCENE V.}
\personscene{PYRRHVS, PHOENIX.}
\enonciateur{PYRRHUS.}
                ]
                
                \antilabe HE bien, Phœnix, l’Amour est-il le
 Maistre?&
       Tes yeux refusent-ils encor de me
 connaistre?\&
       
\stanza[
\enonciateur{PHOENIX.}
]
                
                Ah! je vous reconnois, \ampersand\ ce
 juste courroux&
       Ainsi qu’à tous les Grecs,
 Seigneur, vous rend à vous.&
       Et qui l’auroit pensé, qu’une si noble audace&
       D’un long abbaissement prendroit si-tost la place?&
       Que l’on pût si-tost vaincre un poison si charmant?&
       Mais Pyrrhus,
 quand il veut, sçait vaincre en un
 moment.&
       Ce n’est plus le jouët d’une flamme seruile.&
       C’est Pyrrhus. C’est le Fils, \ampersand\ le Rival
 d’\edgls{Achille},&
       Que la Gloire à la fin rameine sous ses lois,&
       Qui triomphe de Troyeune seconde fois.\&
       
\stanza[
\enonciateur{PYRRHUS.}
]
                
                Dy plutost, qu’aujourd’huy
 commence ma Vi-ctoire.&
       D’aujourd’huy seulement je jouïs de ma gloire,&
       Et mon cœur aussi
 fier, que tu l’as veû soûmis,&
       Croit avoir en l’Amour vaincu
 mille Ennemis.&
       Considere, Phœnix, les troubles que j’éuite,&
       Quelle foule de maux l’Amour traisne à sa suite;&
       Que d’Amis, de devoirs j’allois sacrifier;&
       Quels perils.... Un regard m’eust tout fait oublier.&
       Tous les Grecs conjurez fondoient sur un Rebelle.&
       Je trouvois du
 plaisir à me perdre pour Elle.\&
       
\stanza[
\enonciateur{PHOENIX.}
]
                
                Oüy, je benis, Seigneur,
 l’heureuse cruauté&
       Qui vous rend....\&
       
\stanza[
\enonciateur{PYRRHUS.}
]
                
                \antilabe Tu l’as veû comme elle m’a traitté.&
       Je pensois, en
 voyant sa tendresse
 allarmée,&
       Que son Fils me la dust renuoyer desarmée.&
       J’allois voir le succez de ses embrassemens.&
       Je n’ay trouvé
 que pleurs mélez d’emportemens.&
       Sa misere l’aigrit. Et tousiours plus farouche&
       Cent fois le nom d’Hector est sorti de sa
 bouche.&
       Vainement à son Fils j’assurois mon secours,&
       C’est Hector,
 (disoit-elle en l’embrassant toû-jours;)&
       Voila ses yeux, sa bouche, \ampersand\ déja son audace,&
       C’est luy-mesme,
 c’est toy cher Espoux que j’em-brasse.&
       Et quelle est sa
 pensée? Attend-elle en ce iour&
       Que je luy laisse un Fils pour nourrir son amour?\&
       
\stanza[
\enonciateur{PHOENIX.}
]
                
                Sans doute. C’est le prix que vous
 gardoit l’Ingrate.&
       Mais laissez-la, Seigneur.\&
       
\stanza[
\enonciateur{PYRRHUS.}
]
                
                \antilabe Je voy ce qui la flatte.&
       Sa beauté la rassure, \ampersand\ malgré mon courroux.&
       L’Orgueilleuse m’attend encore à
 ses genoux.&
       Je la verrois aux miens, Phœnix, d’un œil tranquile.&
       Elle est Veuve
 d’\edgls{Hector}. Et je suis Fils d’Achile.&
       Trop de haine separe Andromaque \ampersand\ Pyrrhus.\&
       
\stanza[
\enonciateur{PHOENIX.}
]
                
                Commencez donc, Seigneur, à ne m’en parler
 plus.&
       Allez voir Hermionne, \ampersand\ content de luy plaire,&
       Oubliez à ses pieds iusqu’à vostre colere.&
       Vous-mesme à cét hymen venez la
 disposer?&
       Est-ce sur un Rival qu’il s’en faut reposer?&
       Il ne l’aime que trop.\&
       
\stanza[
\enonciateur{PYRRHUS.}
]
                
                \antilabe Crois-tu, si je l’espouse,&
       Qu’Andromaque en secret n’en sera pas jalouse?\&
       
\stanza[
\enonciateur{PHOENIX.}
]
                
                Quoy tousiours Andromaque occupe vostre
 esprit?&
       Que vous importe, ô Dieux! sa
 joye, ou son despit?&
       Quel charme malgré vous vers elle vous attire?\&
       
\stanza[
\enonciateur{PYRRHUS.}
]
                
                Non, je n’ay pas bien dit tout ce
 qu’il luy faut dire.&
       Ma colere à ses yeux n’a paru qu’à
 demy.&
       Elle ignore à quel point je suis son Ennemy.&
       Retournons-y. Je veux la braver à sa veuë,&
       Et donner à ma haine une libre estenduë.&
       Vien voir tous ses attraits,
 Phœnix, humiliez.&
       Allons.\&
       
\stanza[
\enonciateur{PHOENIX.}
]
                
                \antilabe Allez, Seigneur, vous jeter à ses
 piez.&
       Allez, en luy jurant que vostre
 ame l’adore,&
       A de nouveaux mespris l’encourager
 encore.\&
       
\stanza[
\enonciateur{PYRRHUS.}
]
                
                Je le voy bien, tu crois que prest à l’excuser.&
       Mon Cœur court apres elle, \ampersand\ cherche à s’apaiser.\&
       
\stanza[
\enonciateur{PHOENIX.}
]
                
                Vous aimez, c’est assez.\&
       
\stanza[
\enonciateur{PYRRHUS.}
]
                
                \antilabe Moy l’aimer? Une Ingrate,&
       Qui me hait d’autant plus que mon amour la
 flate?&
       Sans Parens, sans Amis, sans espoir que sur
 moy.&
       Je puis perdre son Fils, peut-estre je le
 doy.&
       Estrangere.... Que dis-je? Esclave dans l’Epire,&
       Je luy donne son
 Fils, mon Ame, mon Empire,&
       Et je ne puis gagner dans son perfide Cœur&
       D’autre rang que celuy de son
 Persecuteur?&
       Non, non, je l’ay juré, ma
 vangeance est certaine.&
       Il faut bien une fois justifier sa haine.&
       J’abandonne son
 Fils. Que de pleurs vont couler!&
       De quel nom sa douleur me
 va-t’elle appeller?&
       Quel spectacle pour elle
 aujourd’huy se dispose!&
       Elle en mourra, Phœnix, \ampersand\ j’en seray la cause.&
       C’est luy mettre moy-mesme un poignard dans le sein.\&
       
\stanza[
\enonciateur{PHOENIX.}
]
                
                Et pourquoy donc en faire éclater le dessein?&
       Que ne consultiez-vous tantost vostre foiblesse?\&
       
\stanza[
\enonciateur{PYRRHUS.}
]
                
                Je t’entens. Mais excuse un reste de tendresse.&
       Crains-tu pour ma colere un si foible combat?&
       D’un amour qui s’esteint c’est le dernier éclat.&
       Allons. A tes conseils, Phœnix, je m’abandonne.&
       Faut-il liurer son Fils? Faut-il
 voir Hermionne?\&
       
\stanza[
\enonciateur{PHOENIX.}
]
                
                Oüy, voyez-la, Seigneur, \ampersand\ par des vœux soûmis&
       Protestez-luy…\&
       
\stanza[
\enonciateur{PYRRHUS.}
]
                
                \antilabe Faisons tout ce que j’ay promis.\&
       \stanza[\chapter{ACTE III.}
\section{SCENE PREMIERE.} 
    \personscene{ORESTE, PYLADE.}  
    \enonciateur{PYLADE.} 
    ]
    
    M oderez donc, Seigneur, cette fureur extréme.&
       Je ne vous connoy plus. Vous n’estes plus vous-mesme.&
       Souffrez....\&
       
\stanza[
\enonciateur{ORESTE.}
]
                
                \antilabe Non, tes conseils ne sont plus de saison,&
       Pylade, je suis las d’écouter la raison.&
       C’est traisner
 trop long-temps ma vie \ampersand\ mon sup-plice.&
       Il faut que je l’enleve, ou bien que je perisse.&
       Le dessein en est pris, je le veux achever.&
       Oüy je le veux.\&
       
\stanza[
\enonciateur{PYLADE.}
]
                
                \antilabe Hé bien il la faut enlever.&
       J’y consens. Mais
 songez cependant où vous estes.&
       Que croira-t’on de vous à voir ce que vous
 faites?\&
       
\stanza[
\enonciateur{ORESTE.}
]
                
                Que sçay-je? De moy-mesme estois-je alors le maistre?&
       La fureur m’emportoit, \ampersand\ je
 venois peut-estre &
       Menasser à la
 fois l’Ingrate \ampersand\ son Amant.\&
       
\stanza[
\enonciateur{PYLADE.}
]
                
                Et quel estoit le fruit de cet
 emportement?\&
       
\stanza[
\enonciateur{ORESTE.}
]
                
                Et quelle ame, dy-moy, ne seroit
 éperduë&
       Du coup dont ma raison vient d’estre confonduë?&
       Il épouse, dit-il, Hermionne demain.&
       Il veut pour m’honorer la tenir de ma main.&
       Ah! plustost
 cette main dans le sang du Barbare....\&
       
\stanza[
\enonciateur{PYLADE.}
]
                
                Vous l’accusez, Seigneur, de ce
 destin bizare.&
       Cependant tourmenté de ses propres
 desseins,&
       Il est peut-estre
 à plaindre, autant que je vous plains.\&
       
\stanza[
\enonciateur{ORESTE.}
]
                
                Non, non, je le connoy, mon desespoir le flate&
       Sans moy, sans mon amour il
 dédaignoit l’Ingrate&
       Ses charmes iusques-là n’avoient pû le toucher.&
       Le Cruel ne la prend que pour me l’arracher.&
       Ah dieux! c’en estoit fait.
 Hermionne gagnée&
       Pour iamais de sa veuë alloit estre éloignée.&
       Son cœur entre l’amour \ampersand\ le dépit confus.&
       Pour se donner à moy n’attendoit
 qu’un refus.&
       Ses yeux s’ouvroient, Pylade. Elle écoutoit Oreste,&
       Luy parloit, le plaignoit. Un mot
 eust fait le reste.\&
       
\stanza[
\enonciateur{PYLADE.}
]
                
                Vous le croyez.\&
       
\stanza[
\enonciateur{ORESTE.}
]
                
                \antilabe Hé quoy? ce courroux enflammé&
       Contre un Ingrat…\&
       
\stanza[
\enonciateur{PYLADE.}
]
                
                \antilabe Iamais il ne fut plus aimé.&
       Pensez-vous, quand Pyrrhus vous l’auroit accordée,&
       Qu’un prétexte tout prest ne l’eust pas retardée?&
       M’en croirez-vous? Lassé de ses trompeurs attraits,&
       Au lieu de l’enlever, Seigneur, je la fuirais.&
       Quoy? Vostre amour se veut charger d’une Furie&
       Qui vous détestera, qui toute vostre vie&
       Regrettant un hymen tout prest à s’achever,&
       Voudra…\&
       
\stanza[
\enonciateur{ORESTE.}
]
                
                \antilabe C’est pour cela que je
 veux l’enlever.&
       Tout luy riroit, Pylade, \ampersand\ moy, pour mon partage,&
       Je n’emporterois donc qu’une inutile rage?&
       J’irois loin d’elle encor, tascher de l’oublier?&
       Non, non, à mes tourmens je veux
 l’associer.&
       C’est trop gémir tout seul. Je suis las qu’on me
 plaigne.&
       Je prétens qu’à mon tour
 l’Inhumaine me craigne,&
       Et que ses Yeux cruels à pleurer
 condannez,&
       Me rendent tous les noms, que je
 leur ay donnez.\&
       
\stanza[
\enonciateur{PYLADE.}
]
                
                Voila donc le succez qu’aura vostre Ambassade,&
       Oreste ravisseur.\&
       
\stanza[
\enonciateur{ORESTE.}
]
                
                \antilabe Et qu’importe, Pylade?&
       Quand nos Estats vangez jouïront
 de mes soins,&
       L’Ingrate de mes pleurs jouïra-t’elle moins?&
       Et que me seruira que la Gréce
 m’admire&
       Tandis que je seray la fable de l’Epire?&
       Que veux-tu? Mais, s’il faut ne te rien déguiser,&
       Mon Innocence enfin commence à me peser.&
       Je ne sçay de
 tout temps quelle injuste Puissance&
       Laisse le Crime
 en paix, \ampersand\ poursuit l’Innocence.&
       De quelque part sur moy que je tourne les yeux,&
       Je ne voy que malheurs qui
 condannent les Dieux.&
       Meritons leur courroux, justifions
 leur haine,&
       Et que le fruit du Crime en précede la peine.&
       Mais toy, par quelle erreur veux-tu toûjours sur toy&
       Destourner un
 courroux qui ne cherche que moy?&
       Assez \ampersand\ trop
 long-temps mon amitié t’accable.&
       Euite un Malheureux, abandonne un Coupable.&
       Cher Pylade,
 croy-moy, mon tourment me suffit,&
       Laisse-moy des
 perils dont j’attens tout le fruit.&
       Porte aux Grecs cet Enfant que Pyrrhus m’abandonne.&
       Va-t’en.\&
       
\stanza[
\enonciateur{PYLADE.}
]
                
                \antilabe Allons, Seigneur, enlevons Hermionne.&
       Au travers des perils un grand Cœur se fait iour.&
       Que ne peut l’amitié conduite par l’amour?&
       Allons de tous vos Grecs encourager le zele.&
       Nos vaisseaux sont tous prests, \ampersand\ le vent nous
 appelle.&
       Je sçay de ce
 Palais tous les détours obscurs.&
       Vous voyez que la Mer en vient battre les Murs.&
       Et cette Nuit sans peine une secrette voye,&
       Iusqu’en vostre
 vaisseau conduira vostre Proye.\&
       
\stanza[
\enonciateur{ORESTE.}
]
                
                J’abuse, cher
 Amy, de ton trop d’amitié.&
       Mais pardonne à des maux, dont toy seul as pitié.&
       Excuse un
 Malheureux, qui perd tout ce qu’il aime,&
       Que tout le monde hait, \ampersand\ qui se hait luy-mesme.&
       Que ne puis-je à mon tour, dans un
 sort plus heu-reux…\&
       
\stanza[
\enonciateur{PYLADE.}
]
                
                Dissimulez,
 Seigneur, c’est tout ce que je
 veux.&
       Gardez qu’avant le coup vostre dessein
 n’éclate.&
       Oubliez iusque-là qu’Hermionne est ingrate.&
       Oubliez vostre amour. Elle vient,
 je la voy.\&
       
\stanza[
\enonciateur{ORESTE.}
]
                
                Va-t’en. Répons-moy d’elle, \ampersand\ je répons de moy.\&
       
                      
\stanza[\section{SCENE II.}
\personscene{HERMIONNE, ORESTE, CLEONNE.}
\enonciateur{ORESTE.}
                ]
                
                HE bien? Mes soins vous ont rendu vostre Conqueste.&
       J’ay veû Pyrrhus, Madame, \ampersand\ vostre hymen
 s’apreste.\&
       
\stanza[
\enonciateur{HERMIONNE.}
]
                
                On le dit. Et de plus, on vient de m’assurer,&
       Que vous ne me cherchiez que pour m’y préparer.\&
       
\stanza[
\enonciateur{ORESTE.}
]
                
                Et vostre ame à ses vœux ne sera pas rebelle?\&
       
\stanza[
\enonciateur{HERMIONNE.}
]
                
                Qui l’eust crû, que Pyrrhus ne fust pas
 infidelle?&
       Que sa flamme attendroit si tard pour éclater,&
       Qu’il reviendroit à moy, quand je l’allois quitter?&
       Je veux croire avec vous, qu’il redoute la Grece,&
       Qu’il suit son
 interest plûtost que sa tendresse,&
       Que mes yeux sur vostre ame estoient plus
 absolus.\&
       
\stanza[
\enonciateur{ORESTE.}
]
                
                Non, Madame, il vous aime, \ampersand\ je n’en doute plus.&
       Vos yeux ne font-ils pas tout ce qu’ils veulent
 faire?&
       Et vous ne vouliez pas sans doute
 luy déplaire.\&
       
\stanza[
\enonciateur{HERMIONNE.}
]
                
                Mais que puis-je, Seigneur? On a promis ma foy.&
       Luy raviray-je un
 bien, qu’il ne tient pas de moy?&
       L’Amour ne regle pas le sort d’une Princesse.&
       La gloire d’obeïr est tout ce
 qu’on nous laisse.&
       Cependant je partois, \ampersand\ vous
 avez pû voir&
       Combien je relaschois pour vous de mon devoir.\&
       
\stanza[
\enonciateur{ORESTE.}
]
                
                Ah! que vous sçaviez bien, Cruelle… Mais, Ma-dame,&
       Chacun peut à son choix disposer de son ame.&
       La vostre estoit
 à vous. I’esperois. Mais enfin&
       Vous l’avez pû donner sans me faire un larcin.&
       Je vous accuse
 aussi, bien moins que la
 Fortune.&
       Et pourquoy vous lasser d’une plainte importune?&
       Tel est vostre
 devoir, je l’avouë.
 Et le mien&
       Est de vous épargner un si triste entretien.\&
       
                      
\stanza[\section{SCENE III.}
\personscene{HERNMIONNE, CLEONNE.}
\enonciateur{HERMIONNE.}
                ]
                
                ATendois-tu, Cleonne, un courroux si mo-deste?\&
       
\stanza[
\enonciateur{CLEONNE.}
]
                
                La douleur qui se taist n’en est que plus funeste.&
       Je le plains. D’autant plus
 qu’Auteur de son ennuy,&
       Le coup qui l’a perdu n’est parti
 que de luy.&
       Contez, depuis quel temps
 vostre hymen se prépare.&
       Il a parlé, Madame, \ampersand\ Pyrrhusse déclare.\&
       
\stanza[
\enonciateur{HERMIONNE.}
]
                
                Tu crois que Pyrrhus craint? Et que craint-il encor?&
       Des Peuples, qui dix ans ont fuy devant \edgls{Hector}?&
       Qui cent fois effrayez de l’absence d’\edgls{Achille},&
       Dans leurs vaisseaux brulans ont cherché leur azile,&
       Et qu’on verroit encor, sans
 l’appuy de son Fils,&
       Redemander \edgls{Hélène} aux Troyens impunis?&
       Non, Cleonne, il
 n’est point Ennemy de luy-mesme,&
       Il veut tout ce qu’il fait, \ampersand\ s’il m’espouse, il m’aime.&
       Mais qu’Oreste à son gré m’impute ses douleurs.&
       N’avons-nous d’entretien que celuy
 de ses pleurs?&
       Pyrrhus revient à nous. Hé bien, chere Cleonne,&
       Conçois-tu les transports de
 l’heureuse Hermionne?&
       Sçais-tu quel est Pyrrhus? T’éses-tu fait raconter&
       Le nombre des Exploits… Mais qui les peut
 conter?&
       Intrepide, \ampersand\ par tout suiuy de
 la Victoire,&
       Charmant, Fidelle enfin, rien ne manque à sa Gloire.&
       Songe....\&
       
\stanza[
\enonciateur{CLEONNE.}
]
                
                \antilabe Dissimulez. Vostre Rivale en pleurs,&
       Vient à vos pieds sans doute
 apporter ses douleurs.\&
       
\stanza[
\enonciateur{HERMIONNE.}
]
                
                Dieux! ne puis-je à ma joye abandonner mon ame?&
       Sortons. Que luy dirois-je?\&
       
                      
\stanza[\section{SCENE IV.}
\personscene{ANDROMAQVE, HERNMIONNE, CLEONNE, CEPHISE.}
\enonciateur{ANDROMAQVE.}
                ]
                
                \antilabe OV fuyez-vous, Madame?&
       N’est-ce point à vos yeux, un spectacle assez doux&
       Que la Veuve d’\edgls{Hector} pleurante à vos genoux?&
       Je ne viens point icy, par de
 jalouses larmes,&
       Vous enuier un Cœur, qui se rend à vos charmes.&
       Par les mains de son Pere, helas!
 j’ay veû percer&
       Le seul, où mes regards
 pretendoient s’adresser.&
       Ma flamme par \edgls{Hector} fut jadis allumée,&
       Auec luy dans la tombe elle s’est
 enfermée.&
       Mais il me reste un Fils. Vous sçaurez quelque iour,&
       Madame, pour un Fils iusqu’où va nostre amour.&
       Mais vous ne sçaurez pas, du moins
 je le souhaitte,&
       En quel trouble mortel son
 interest nous jette,&
       Lors que de tant de biens, qui pouvoient nous flatter,&
       C’est le seul qui
 nous reste, \ampersand\ qu’on veut nous l’oster.&
       Helas! Lors que lassez de dix ans de misere,&
       Les Troyens en courroux menaçoient vostre Mere,&
       J’ay sçeû de mon
 \edgls{Hector} luy procurer l’appuy;&
       Vous pouvez sur
 Pyrrhus, ce que j’ay pû sur luy.&
       Que craint-on d’un Enfant, qui suruit à sa perte?&
       Laissez-moy le
 cacher en quelque Isle deserte.&
       Sur les soins de sa Mere on peut s’en assurer,&
       Et mon Fils avec moy n’aprendra
 qu’à pleurer.\&
       
\stanza[
\enonciateur{HERMIONNE.}
]
                
                Je conçoy vos douleurs. Mais un devoir austere,&
       Quand mon Pere a parlé, m’ordonne de me taire.&
       C’est luy, qui de Pyrrhus fait agir le courroux.&
       S’il faut fléchir Pyrrhus, qui le peut mieux que vous?&
       Vos yeux assez
 long-temps ont regné sur son ame.&
       Faites-le prononcer, j’y souscriray, Madame.\&
       
                      
\stanza[\section{SCENE V.}
\personscene{ANDROMAQVE, CEPHIZE.}
\enonciateur{ANDROMAQVE.}
                ]
                
                QVel mépris la Cruelle
 attache à ses refus!\&
       
\stanza[
\enonciateur{CEPHIZE.}
]
                
                Je croirois ses
 conseils, \ampersand\ je verrois Pyrrhus.&
       Un regard confondroit Hermionne \ampersand\ la Gréce..&
       Mais luy-mesme il vous
 cherche.\&
       
                      
\stanza[\section{SCENE VI.}
\personscene{PYRRHVS, ANDROMAQVE, PHOENIX, CEPHIZE.}
\enonciateur{PYRRHVS}
                ]
                
                \antilabe OV donc est la
 Princesse?&
       Ne m’avois-tu pas dit qu’elle estoit en ces lieux?\&
       
\stanza[
\enonciateur{PHOENIX.}
]
                
                Je le croyois.\&
       
\stanza[
\enonciateur{ANDROMAQVE}
]
                
                \antilabe Tu vois le pouvoir de mes yeux.\&
       
\stanza[
\enonciateur{PYRRHVS.}
]
                
                Que dit-elle, Phœnix?\&
       
\stanza[
\enonciateur{ANDROMAQVE.}
]
                
                \antilabe Helas! tout m’abandonne.\&
       
\stanza[
\enonciateur{PHOENIX.}
]
                
                Allons, Seigneur, marchons sur les
 pas d’Hermionne.\&
       
\stanza[
\enonciateur{CEPHISE.}
]
                
                Qu’attendez-vous? Forcez ce silence obstiné.\&
       
\stanza[
\enonciateur{ANDROMAQVE.}
]
                
                Il a promis mon Fils.\&
       
\stanza[
\enonciateur{CEPHISE.}
]
                
                \antilabe Il ne l’a pas donné.\&
       
\stanza[
\enonciateur{ANDROMAQVE.}
]
                
                Non, non, j’ay beau pleurer, sa mort est resoluë.\&
       
\stanza[
\enonciateur{PYRRHVS.}
]
                
                \antilabe Daigne-t’elle sur nous tourner au moins
 la veuë?&
       Quel orgueil!\&
       
\stanza[
\enonciateur{ANDROMAQVE.}
]
                
                Je ne fay que l’irriter encor.&
       Sortons.\&
       
\stanza[
\enonciateur{PYRRHVS.}
]
                
                \antilabe Allons aux Grecs liurer le Fils
 d’Hector.\&
       
\stanza[
\enonciateur{ANDROMAQVE.}
]
                
                Ah, Seigneur, arrestez. Que
 prétendez-vous faire?&
       Si vous liurez le Fils, liurez-leur donc la
 Mere.&
       Vos sermens m’ont tantost iuré tant d’amitié.&
       Dieux! N’en reste-t’il pas du
 moins quelque pitié?&
       Sans espoir de pardon m’avez-vous condamnée?\&
       
\stanza[
\enonciateur{PYRRHVS.}
]
                
                Phœnix vous le
 dira, ma parole est donnée.\&
       
\stanza[
\enonciateur{ANDROMAQVE.}
]
                
                Vous qui braviez pour moy tant de
 perils divers?\&
       
\stanza[
\enonciateur{PYRRHVS.}
]
                
                J’estois aveugle alors, mes yeux se sont ouvers.&
       Sa grace à vos desirs pouvoit estre accordée.&
       Mais vous ne l’avez pas seulement demandée.&
       C’en est fait.\&
       
\stanza[
\enonciateur{ANDROMAQVE.}
]
                
                \antilabe Ah! Seigneur, vous entendiez assez&
       Des soupirs, qui craignoient de se voir repoussez.&
       Pardonnez à l’éclat d’une illustre fortune&
       Ce reste de fierté, qui craint
 d’estre importune.&
       Vous ne l’ignorez pas, Andromaquesans vous&
       N’auroit iamais d’un Maistre embrassé les
 genoux.\&
       
\stanza[
\enonciateur{PYRRHVS.}
]
                
                Non, vous me haïssez. Et dans le fonds de l’ame&
       Vous craignez de devoir quelque
 chose à ma flâme.&
       Ce Fils mesme, ce Fils, l’objet de
 tant de soins,&
       Si je l’avois sauvé, vous l’en aimeriez moins.&
       La haine, le mespris, contre moy
 tout s’assemble.&
       Vous me haïssez
 plus que tous les Grecs ensemble.&
       Ioüissez à loisir d’un si noble
 courroux.&
       Allons, Phœnix.\&
       
\stanza[
\enonciateur{ANDROMAQVE.}
]
                
                \antilabe Allons rejoindre mon Espoux.\&
       
\stanza[
\enonciateur{CEPHISE.}
]
                
                Madame....\&
       
\stanza[
\enonciateur{ANDROMAQVE.}
]
                
                \antilabe Et que veux-tu que je lui dise
 encore?&
       Auteur de tous mes maux crois-tu qu’il les
 ignore?&
       Seigneur, voyez l’estat où vous me
 reduisez?&
       J’ay veu mon Pere mort, \ampersand\ nos
 Murs embrasez,&
       J’ay veû trancher les iours de ma
 Famille entiere,&
       Et mon Espoux sanglant traisné sur la poussiere,&
       Son Fils seul avec moy reserué pour les fers.&
       Mais que ne peut un Fils, je respire, je sers.&
       J’ay fait plus. Je me suis quelquefois consolée&
       Qu’icy plûtost qu’ailleurs le sort m’eust exhilée;&
       Qu’heureux dans son malheur, le
 Fils de tant de Rois.&
       Puis qu’il devoit seruir, fust tombé sous vos
 lois.&
       J’ay crû que sa
 Prison deviendroit son Azile.&
       Iadis \edgls{Priam}soûmis fut respecté
 d’Achile.&
       J’attendois de son Fils encor plus de bonté.&
       Pardonne, cher \edgls{Hector}, à ma credulité.&
       Je n’ay pû soupçonner ton Ennemy d’un crime,&
       Malgré luy-mesme enfin je l’ay crû
 magnanime.&
       Ah! s’il l’estoit assez, pour nous laisser du moins&
       Au Tombeau qu’à ta Cendre ont éleué mes soins;&
       Et que finissant-là sa haine \ampersand\ nos miseres,&
       Il ne séparast
 point des despoüilles si cheres!\&
       
\stanza[
\enonciateur{PYRRHVS.}
]
                
                Va m’attendre, Phœnix.\&
       
                      
\stanza[\section{SCENE VII.}
\personscene{PYRRHVS, ANDROMAQVE, CEPHISE.}
\enonciateur{PYRRHVS}
                ]
                
                \antilabe MAdame, demeurez;&
       On peut vous rendre encor ce Fils que vous
 pleurez.&
       Oüy, je sens à
 regret, qu’en excitant vos larmes,&
       Je ne fay contre moy que vous
 donner des armes.&
       Je croyois apporter plus de haine
 en ces lieux.&
       Mais, Madame, du moins tournez vers moy les
 yeux.&
       Voyez si mes regards sont d’un Iuge severe,&
       S’ils sont d’un
 Ennemy qui cherche à vous déplaire.&
       Pourquoy me forcez-vous
 vous-mesme à vous
 trahir?&
       Au nom de vostre Fils, cessons de nous haïr.&
       A le sauver
 enfin, c’est moy qui vous conuie.&
       Faut-il que mes soûpirs vous
 demandent sa vie?&
       Faut-il qu’en sa faveur j’embrasse vos genoux?&
       Pour la derniere fois, sauvez-le, sauvez-vous.&
       Je sçay de quels
 sermens je romps pour vous les
 chaisnes,&
       Combien je vais sur moy faire éclater de haines.&
       Je renuoye Hermionne, \ampersand\ je mets
 sur son front,&
       Au lieu de ma Couronne, un éternel
 affront.&
       Je vous conduis au Temple, où son Hymen s’ap-preste.&
       Je vous ceins du Bandeau, préparé
 pour sa Teste.&
       Mais ce n’est plus, Madame, une offre à dédai-gner.&
       Je vous le dis, il faut ou perir,
 ou regner.&
       Mon cœur, desesperé d’un an d’ingratitude,&
       Ne peut plus de son sort souffrir l’incertitude.&
       C’est craindre, menasser, \ampersand\ gemir trop long-temps.&
       Je meurs, si je vous pers, mais je meurs, si j’attens.&
       Songez-y, je vous laisse, \ampersand\ je viendray vous
 prendre,&
       Pour vous mener au Temple, où ce Fils doit
 m’at-tendre.&
       Et là vous me verrez soûmis, ou
 furieux,&
       Vous couronner, Madame, ou le perdre à vos
 yeux.\&
       
                      
\stanza[\section{SCENE VIII.}
\personscene{ANDROMAQVE, CEPHISE.}
\enonciateur{CEPHISE.}
                ]
                
                HE bien, je vous l'ay dit, qu'en despit de la 
 Grece&
       De vostre sort
 encor vous seriez la Maistresse.\&
       
\stanza[
\enonciateur{ANDROMAQVE.}
]
                
                Helas! De quel effet tes discours
 sont suivis?&
       Il ne me restoit plus qu’à
 condamner mon Fils.\&
       
\stanza[
\enonciateur{CEPHISE.}
]
                
                Madame, à vostre Espoux c’est estre assez fidelle.&
       Trop de vertu pourroit vous rendre criminelle.&
       Luy-mesme il porteroit vostre ame à la douceur.\&
       
\stanza[
\enonciateur{ANDROMAQVE.}
]
                
                Quoy, je luy donnerois Pyrrhus pour successeur?\&
       
\stanza[
\enonciateur{CEPHISE.}
]
                
                Ainsi le veut son
 Fils, que les Grecs vous ravissent.&
       Pensez-vous qu’aprés tout ses Manes en rougis-sent?&
       Qu’il mesprisast, Madame, un Roy victorieux,&
       Qui vous fait remonter au rang de vos Ayeux;&
       Qui foule aux pieds pour vous vos Vainqueurs en
 colere,&
       Qui ne se souvient plus qu’\edgls{Achille} estoit son
 Pere,&
       Qui dément ses Exploits, \ampersand\ les
 rend superflus?\&
       
\stanza[
\enonciateur{ANDROMAQVE.}
]
                
                Dois-je les oublier, s’il ne s’en souvient plus?&
       Dois-je oublier \edgls{Hector} priué de funerailles,&
       Et traisné sans
 honneur autour de nos murailles?&
       Dois-je oublier son Pere à mes
 pieds renuersé,&
       Ensanglantant l’Autel qu’il
 tenoit embrassé?&
       Songe, songe, Cephise, à cette Nuit
 cruelle,&
       Qui fut pour tout un Peuple une Nuit éternelle.&
       Figure-toy Pyrrhus les yeux étincelans,&
       Entrant à la lueur de nos Palais brûlans;&
       Sur tous mes Freres morts se
 faisant un passage,&
       Et de sang tout couvert échauffant le carnage.&
       Songe aux cris des Vainqueurs, songe aux cris des Mourans,&
       Dans la flamme étouffez, sous le
 fer expirans.&
       Peins-toy dans ces horreurs Andromaque es-perduë.&
       Voila comme Pyrrhus vint s’offrir à ma veuë,&
       Voila par quels exploits il sçeût se couronner,&
       Enfin voila l’Espoux que tu me
 veux donner.&
       Non, je ne seray point complice de ses crimes.&
       Qu’il nous prenne, s’il veut, pour dernieres
 Victimes.&
       Tous mes ressentimens luy seroient asseruis.\&
       
\stanza[
\enonciateur{CEPHISE.}
]
                
                Hé bien, allons donc voir expirer vostre Fils.&
       On n’attend plus que vous. Vous fremissez, Ma-dame?\&
       
\stanza[
\enonciateur{ANDROMAQVE.}
]
                
                Ah! de quel souvenir viens-tu frapper mon ame?&
       Quoy, Cephise, j’iray voir expirer encor&
       Ce Fils, ma seule joye, \ampersand\
 l’image d’Hector?&
       Ce Fils que de sa flamme il me
 laissa pour gage?&
       Helas! il m’en souvient, le iour que son courage,&
       Luy fit chercher Achille, ou
 plûtost le trespas;&
       Il demanda son Fils, \ampersand\ le
 prit dans ses bras.&
       Chere Espouse,
 dit-il, en essuyant mes larmes,&
       J’ignore quel succez le sort garde à mes armes,&
       Je te laisse mon Fils, pour gage de ma foy;&
       S’il me perd, je prétens qu’il
 me retrouve en toy.&
       Si d’un heureux hymen la memoire
 t’est chere,&
       Montre au Fils à quel point tu cherissois le Pere.&
       Et je puis voir respandre un sang si
 pretieux?&
       Et je laisse avec luy perir tous ses Ayeux?&
       Roy barbare, faut-il que mon crime l’entraisne?&
       Si je te haïs, est-il coupable de ma haine?&
       T’a-t’il de tous les siens
 reproché le trespas?&
       S’est-il plaint à tes yeux des
 maux qu’il ne sent pas?&
       Mais cependant, mon Fils, tu meurs, si je n’arreste&
       Le fer, que ce Cruel tient leué sur ta teste.&
       Je l’en puis détourner, \ampersand\ je t’y vais offrir?&
       Non tu ne mourras point, je ne
 le puis souffrir.&
       Allons trouver Pyrrhus. Mais non, chere Cephise,&
       Va le trouver pour
 moy.\&
       
\stanza[
\enonciateur{CEPHISE.}
]
                
                \antilabe Que faut-il que je dise?\&
       
\stanza[
\enonciateur{ANDROMAQVE.}
]
                
                Dy-luy que de mon Fils l’amour est assez fort…&
       Crois-tu que dans son cœur il
 ait juré sa mort?&
       L’amour peut-il si loin pousser
 sa barbarie?\&
       
\stanza[
\enonciateur{CEPHISE.}
]
                
                Madame, il va bien-tost revenir en furie.\&
       
\stanza[
\enonciateur{ANDROMAQVE.}
]
                
                Hé bien, va l’assurer…\&
       
\stanza[
\enonciateur{CEPHISE.}
]
                
                \antilabe De quoy? de vostre foy?\&
       
\stanza[
\enonciateur{ANDROMAQVE.}
]
                
                Helas! pour la promettre est-elle encore à moy?&
       O cendres d’un Espoux! ô Troyens! ô mon Pere!&
       O mon Fils, que tes iours coustent cher à ta Mere!&
       Allons.\&
       
\stanza[
\enonciateur{CEPHISE.}
]
                
                \antilabe Où donc, Madame, \ampersand\ que resoluez-vous?\&
       
\stanza[
\enonciateur{ANDROMAQVE.}
]
                
                Allons, sur son
 tombeau, consulter mon Espoux.\&
       \stanza[\chapter{ACTE IV.}
\section{SCENE I.} 
    \personscene{ANDROMAQVE, CEPHISE.}  
    \enonciateur{CEPHISE.} 
    ]
    
    A h! je n’en doute point. C’est vostre Espoux, Madame,&
       C’est \edgls{Hector} qui produit ce miracle en vostre ame.&
       Il veut que Troye encor se puisse re-lever,&
       Auec cet heureux Fils, qu’il vous fait conseruer.&
       Pyrrhus vous
 l’a promis. Vous venez de l’entendre,&
       Madame, il n’attendoit qu’un mot
 pour vous le rendre.&
       Croyez en ses transports. Pere, Sceptre, Alliez,&
       Content de vostre cœur, il met
 tout à vos pieds.&
       Sur luy, sur tout son Peuple il vous rend Souve-raine.&
       Est-ce là ce Vainqueur digne de
 tant de haine?&
       Déja contre les Grecs plein d’un
 noble courroux,&
       Le soin de vostre Fils le touche autant que vous;&
       Il previent leur fureur, il luy
 laisse sa
 Garde,&
       Pour ne pas l’exposer luy-mesme il se hazarde.&
       Mais tout s’apreste au Temple.
 Et vous avez pro-mis....\&
       
\stanza[
\enonciateur{ANDROMAQVE.}
]
                
                Ouy, je m’y trouverai. Mais
 allons voir mon Fils.\&
       
\stanza[
\enonciateur{CEPHIZE.}
]
                
                Madame, qui vous presse? Il suffit que sa
 veuë&
       Desormais à vos yeux ne soit plus deffenduë.&
       Vous luy pourrez bientost
 prodiguer vos bontez,&
       Et vos embrassemens ne seront plus contez.&
       Quel plaisir! d’élever un Enfant qu’on voit craistre,&
       Non plus comme un Esclave éleué pour son Maî-tre,&
       Mais pour voir avec luy renaistre tant de Rois.\&
       
\stanza[
\enonciateur{ANDROMAQVE.}
]
                
                Cephize, allons le voir pour la derniere
 fois.\&
       
\stanza[
\enonciateur{CEPHIZE.}
]
                
                Que dites-vous? ô Dieux!\&
       
\stanza[
\enonciateur{ANDROMAQVE.}
]
                
                \antilabe O! ma chere Cephize,&
       Ce n’est point avec toy que mon cœur se déguise.&
       Ta foy dans mon malheur s’est
 montrée à mes yeux.&
       Mais j’ay crû qu’à mon tour tu me connoissois mieux.&
       Quoy donc as-tu pensé
 qu’Andromaque infidelle,&
       Pûst trahir un
 Espoux qui croit reviure en
 elle,&
       Et que de tant de Morts réueillant la
 douleur,&
       Le soin de mon repos me fist troubler le leur?&
       Est-ce là cette ardeur tant
 promise à sa cendre.&
       Mais son Fils perissoit, il l’a falu défendre?&
       Pyrrhus en
 m’épousant s’en déclare l’appuy.&
       Il suffit. Je
 veux bien m’en reposer sur luy.&
       Je sçay quel
 est Pyrrhus.
 Violent, mais sincere,&
       Cephize, il
 fera plus qu’il n’a promis de faire.&
       Sur le courroux des Grecs, je m’en repose encor,&
       Leur haine va donner un Pere au
 Fils d’Hector.&
       Je vais donc, puisqu’il faut que je me sacrifie,&
       Assurer à
 Pyrrhus le reste
 de ma vie.&
       Je vais en recevant sa foy sur les Autels,&
       L’engager à mon Fils par des nœuds îmmortels.&
       Mais aussi-tost ma main, à moy seule funeste,&
       D’une infidelle vie abbregera le
 reste,&
       Et sauvant ma
 vertu, rendra ce que je doy,&
       A Pyrrhus, à
 mon Fils, à mon Espoux, à moy.&
       Voila de mon amour l’innocent stratagéme;&
       Voila ce qu’un Espoux m’a commandé luy-méme.&
       J’iray seule
 rejoindre \edgls{Hector}, \ampersand\ mes Ayeux.&
       Cephize,
 c’est à toy de me fermer les yeux.\&
       
\stanza[
\enonciateur{CEPHIZE.}
]
                
                Ah! ne pretendez pas que je
 puisse suruiure.\&
       
\stanza[
\enonciateur{ANDROMAQVE.}
]
                
                Non, non, je te deffens,
 Cephize, de me suiure.&
       Je confie à tes soins mon unique tresor,&
       Si tu vivois pour moy, vy pour
 le Fils d’\edgls{Hector}.&
       De l’espoir des Troyens seule dépositaire,&
       Songe, à combien de Roys tu deviens necessaire.&
       Veille auprés de Pyrrhus. Fay-luy garder sa foy.&
       S’il le faut, je consens que tu parles de moy.&
       Fais-luy valoir l’hymen, où je
 me suis rangée;&
       Dy-luy, qu’avant ma mort je luy fus engagée,&
       Que ses ressentimens doivent estre effacez,&
       Qu’en luy laissant mon Fils, c’est l’estimer assez.&
       Fay connoistre à mon Fils les
 Heros de sa Race;&
       Autant que tu pourras, conduy-le sur leur trace.&
       Dy-luy, par quels exploits leurs noms ont
 éclaté,&
       Plustost ce
 qu’ils ont fait, que ce qu’ils ont esté.&
       Parle luy tous les jours des Vertus de son Pere,&
       Et quelquefois aussi parle luy de sa Mere.&
       Mais qu’il ne songe plus,
 Cephize, à nous vanger&
       Nous luy laissons un Maistre, il le doit
 ménager.&
       Qu’il ait de ses Ayeux un souvenir modeste,&
       Il est du sang
 d’\edgls{Hector}, mais il en est le reste.&
       Et pour ce reste enfin j’ay moy-mesme en un
 jour,&
       Sacrifié mon sang, ma haine,
 \ampersand\ mon amour.\&
       
\stanza[
\enonciateur{CEPHISE.}
]
                
                Helas!\&
       
\stanza[
\enonciateur{ANDROMAQVE.}
]
                
                \antilabe Ne me suis point, si
 ton cœur en allarmes,&
       Prevoit qu’il ne pourra
 commander à tes larmes,&
       On vient. Cache tes pleurs, Cephize, \ampersand\ souviens-toy,&
       Que le sort d’Andromaque est commis à
 ta foy.&
       C’est Hermionne. Allons, fuyons sa
 violence.\&
       
                      
\stanza[\section{SCENE II.}
\personscene{HERMIONNE, CLEONNE.}
\enonciateur{CLEONNE.}
                ]
                
                NOn, je
 ne puis assez admirer ce silence.&
       Vous vous taisez, Madame, \ampersand\
 ce cruel mépris&
       N’a pas du moindre trouble agité vos esprits?&
       Vous soustenez
 en paix une si rude attaque?&
       Vous qu’on voyoit fremir au seul
 nom d’Andro-maque?&
       Vous qui sans desespoir ne pouviez endurer&
       Que Pyrrhus
 d’un regard la voulust honorer?&
       Il l’épouse. Il luy donne avec son Diadéme&
       La foy, que vous venez de recevoir vous-mesme;&
       Et vostre bouche encor muette à
 tant d’ennuy,&
       N’a pas daigné s’ouvrir pour se plaindre de luy?&
       Ah! que je crains, Madame, un calme si funeste!&
       Et qu’il vaudroit bien mieux....\&
       
\stanza[
\enonciateur{HERMIONNE.}
]
                
                \antilabe Fais-tu venir Oreste!\&
       
\stanza[
\enonciateur{CLEONNE.}
]
                
                Il vient, Madame, il vient. Et vous pouvez juger,&
       Que bientost à vos pieds il
 alloit se ranger.&
       Prest à seruir
 toûjours sans espoir de salaire,&
       Vos yeux ne sont que trop assurez de luy plaire.&
       Mais il entre.\&
       
                      
\stanza[\section{SCENE III.}
\personscene{ORESTE, HERMIONNE, CLEONNE,.}
\enonciateur{ORESTE.}
                ]
                
                \antilabe AH Madame? Est-il vray qu’une fois&
       Oreste en vous cherchant obeïsse à vos lois?&
       Ne m’a-t-on point flatté d’une
 fausse esperance?&
       Auez-vous en-effet souhaitté ma
 presence?&
       Croiray-ie que vos yeux à la fin desarmez&
       Veulent.....\&
       
\stanza[
\enonciateur{HERMIONNE.}
]
                
                \antilabe Je veux sçavoir, Seigneur, si vous m’aimez.\&
       
\stanza[
\enonciateur{ORESTE.}
]
                
                Si je vous aime? O Dieux! mes sermens, mes par-jures,&
       Ma fuite, mon retour, mes respects, mes injures,&
       Mon desespoir,
 mes yeux de pleurs toûjours noyez,&
       Quels témoins croirez-vous, si
 vous ne les croyez?\&
       
\stanza[
\enonciateur{HERMIONNE.}
]
                
                Vangez-moy, je croy
 tout.\&
       
\stanza[
\enonciateur{ORESTE.}
]
                
                \antilabe Hé bien allons, Madame.&
       Mettons encore un coup toute la
 Grece en flame.&
       Prenons, en signalant mon bras,
 \ampersand\ vostre nom,&
       Vous la place d’Helene,
 \ampersand\ moy d’Agamemnon.&
       De Troye en ce pays
 réueillons les miseres,&
       Et qu’on parle de nous, ainsi
 que de nos Peres.&
       Partons, je suis tout prest.\&
       
\stanza[
\enonciateur{HERMIONNE.}
]
                
                \antilabe Non, Seigneur, demeurons,&
       Je ne veux pas si loin porter de tels affrons.&
       Quoy de mes ennemis couronnant l’insolence,&
       J’irois attendre ailleurs une lente vengeance,&
       Et je m’en remettrois au destin des combats,&
       Qui peut-estre à la fin ne me
 vangeroit pas?&
       Je veux qu’à mon depart toute
 l’Epire pleure.&
       Mais si vous me vangez,
 vangez-moydans une heure.&
       Tous vos retardemens sont pour
 moy des refus.&
       Courez au Temple. Il faut
 immoler....\&
       
\stanza[
\enonciateur{ORESTE.}
]
                
                \antilabe Qui?\&
       
\stanza[
\enonciateur{HERMIONNE.}
]
                
                \antilabe Pyrrhus.\&
       
\stanza[
\enonciateur{ORESTE.}
]
                
                Pyrrhus, Madame?\&
       
\stanza[
\enonciateur{HERMIONNE.}
]
                
                \antilabe Hé! quoy? vostre haine chancelle?&
       Ah! courez, \ampersand\ craignez que je ne vous rappelle.&
       N’alleguez point des droits que je veux oublier.&
       Et ce n’est pas à vous à le justifier.\&
       
\stanza[
\enonciateur{ORESTE.}
]
                
                Moy, je l’excuserois? Ah! vos bontez, Madame,&
       Ont graué trop avant ses crimes dans mon ame.&
       Vangeons-nous, j’y consens. Mais
 par d’autres chemins.&
       Soyons ses Ennemis, \ampersand\ non ses Assassins.&
       Faisons de sa
 ruine une juste Conqueste.&
       Quoy pour réponse aux Grecs
 porteray-je sa teste?&
       Et n’ay-je pris sur moy le soin de tout l’Estat,&
       Que pour m’en acquitter par un
 Assassinat?&
       Souffrez, au nom des Dieux, que la Grece
 s’explique,&
       Et qu’il meure chargé de la haine publique.&
       Souvenez-vous qu’il regne, \ampersand\
 qu’un front cou-ronné…\&
       
\stanza[
\enonciateur{HERMIONNE.}
]
                
                Ne vous suffit-il pas que je l’ay condamné?&
       Ne vous suffit-il pas que ma
 Gloire offensée&
       Demande une Victime, à moy seule adressée;&
       Qu’Hermionne est le prix d’un Tyran opprimé,&
       Que je le hais, enfin, Seigneur,
 que je l’aimay?&
       Je ne m’en cache point. L’ingrat
 m’avoit sçeû plaire,&
       Soit qu’ainsi l’ordonnast mon amour, ou mon Pere,&
       N’importe. Mais enfin reglez-vous là-dessus.&
       Malgré mes vœux, Seigneur, honteusement deceûs,&
       Malgré la juste horreur que son crime me donne,&
       Tant qu’il viura, craignez que je ne luy pardonne.&
       Doutez iusqu’à sa mort d’un courroux incertain,&
       S’il ne meurt aujourd’huy, je
 puis l’aimer demain.\&
       
\stanza[
\enonciateur{ORESTE.}
]
                
                Hé bien, il faut le perdre, \ampersand\ préuenir sa grace.&
       Il faut.... Mais cependant, que faut-il que je fasse?&
       Comment puis-je si-tost seruir vostre
 courroux?&
       Quel chemin iusqu'à luy peut
 conduire mes coups?&
       A peine suis-je encore arriué
 dans l’Epire,&
       Vous voulez par mes mains renuerser un Empire.&
       Vous voulez qu’un Roy meure,
 \ampersand\ pour son chasti-ment,&
       Vous ne donnez qu’un iour, qu’une heure, qu’un moment.&
       Aux yeux de tout son Peuple, il
 faut que je l’opprime?&
       Laissez-moy
 vers l’Autel conduire ma Victime.&
       Je ne m’en défens plus. Et je ne veux qu’aller&
       Reconnoistre la place où je dois l’immoler.&
       Cette Nuit je vous sers. Cette Nuit je l’attaque.\&
       
\stanza[
\enonciateur{HERMIONNE.}
]
                
                Mais cependant ce Iour il espouse Andromaque.&
       Dans le Temple déja le trosne
 est éleué.&
       Ma honte est confirmée, \ampersand\ son Crime acheué.&
       Enfin qu’attendez-vous? Il vous offre sa Teste.&
       Sans Gardes, sans défense il marche à cette Feste.&
       Autour du Fils d’\edgls{Hector} il les fait tous ranger.&
       Il s’abandonne au bras qui me voudra vanger.&
       Voulez-vous, malgré luy, prendre soin de sa vie?&
       Armez avec vos Grecs, tous ceux
 qui m’ont suivie.&
       Soulevez vos Amis. Tous les
 miens sont à vous.&
       Il me trahit, vous trompe, \ampersand\ nous mesprise tous.&
       Mais quoy? Déja leur haine est
 égale à la mienne.&
       Elle espargne à regret l’Espoux d’une Troyenne.&
       Parlez. Mon Ennemy ne vous peut échapper.&
       Ou plûtost, il ne faut que les
 laisser frapper.&
       Conduisez, ou suivez une fureur si belle.&
       Revenez tout couvert du sang de l’Infidelle.&
       Allez, en cét estat soyez seûr de mon cœur.\&
       
\stanza[
\enonciateur{ORESTE.}
]
                
                Mais, Madame, songez…\&
       
\stanza[
\enonciateur{HERMIONNE.}
]
                
                \antilabe Ah! c’en est trop, Seigneur.&
       Tant de raisonnemens offensent ma colere.&
       J’ay voulu vous donner les
 moyens de me plaire,&
       Rendre Oreste content. Mais enfin je voy
 bien,&
       Qu’il veut tousiours se plaindre, \ampersand\ ne meriter rien.&
       Partez. Allez ailleurs vanter vostre constance,&
       Et me laissez
 icy le soin de ma vangeance.&
       De mes lasches bontez mon
 courage est confus,&
       Et c’est trop en un iour essuyer de refus.&
       Je m’en vais seule au Temple, où leur hymen s’apreste,&
       Où vous n’osez aller meriter ma
 conqueste.&
       Là, de mon Ennemy je sçauray m’approcher.&
       Je perceray le Cœur, que je n’ay pû toucher.&
       Et mes sanglantes mains sur moy-mesme tournées,&
       Aussi-tost, malgré luy, joindront nos destinées,&
       Et tout Ingrat qu’il est, il me
 sera plus doux,&
       De mourir avec luy, que de viure
 avec vous.\&
       
\stanza[
\enonciateur{ORESTE.}
]
                
                Non, je vous priveray de ce plaisir funeste,&
       Madame. Il ne mourra que de la main d’Oreste.&
       Vos Ennemis par moy vont vous estre immolez.&
       Et vous reconnoistrez mes soins, si vous voulez.&
       Mais que dis-je? Ah plûtost!
 permettez que j’espere.&
       Excusez un
 Amant, que trouble sa misere,&
       Qui tout prest d’estre heureux, enuie encor le sort&
       D’un Ingrat, condamné par
 vous-mesme à la mort.\&
       
\stanza[
\enonciateur{HERMIONNE.}
]
                
                Allez. De vostre sort laissez-moy la
 conduite.&
       Et que tous vos Vaisseaux soient prests pour
 nostre fuite.\&
       
                      
\stanza[\section{SCENE IV.}
\personscene{HERMIONNE, CLEONNE.}
\enonciateur{}
                ]
                
                VOus vous perdez, Madame.
 Et vous devez songer....\&
       
\stanza[
\enonciateur{HERMIONNE.}
]
                
                Que je me perde, ou non, je songe à me vanger.&
       Je ne sçay mesme encor, quoy qu’il m’ait pû pro-mettre,&
       Sur d’autres que sur moy, si je doy m’en remettre.&
       Pyrrhus n’est pas coupable à ses yeux, comme aux
 miens,&
       Et je tiendrois mes coups bien
 plus seûrs que les siens.&
       Quel plaisir! de vanger moy-mesme mon injure,&
       De retirer mon bras teint du sang du Parjure,&
       Et pour rendre sa peine \ampersand\
 mes plaisirs plus grands,&
       De cacher ma Rivale à ses regards mourans.&
       Ah! si du moins Oreste, en punissant son crime,&
       Luy laissoit le
 regret de mourir ma Victime.&
       Va le trouver. Dy-luy qu’il
 aprenne à l’Ingrat,&
       Qu’on l’immole à ma haine, \ampersand\ non pas à l’Estat.&
       Chere Cleonne
 cours. Ma vangeance est perduë,&
       S’il ignore, en mourant, que c’est moy qui le tuë.\&
       
\stanza[
\enonciateur{CLEONNE.}
]
                
                Je vous obeïray. Mais qu’est-ce que je voy?&
       O Dieux! Qui l’auroit crû, Madame? C’est le Roy.\&
       
\stanza[
\enonciateur{HERMIONNE.}
]
                
                Ah! cours apres Oreste, \ampersand\ dy-luy, ma Cleonne,&
       Qu’il n’entreprenne rien sans
 revoir Hermionne.\&
       
                      
\stanza[\section{SCENE V.}
\personscene{PYRRHVS, HERMIONNE, PHOENIX.}
\enonciateur{PYRRHVS.}
                ]
                
                VOus ne m’attendiez pas,
 Madame, \ampersand\ je voy bien&
       Que mon abord icy trouble vostre
 entretien.&
       Je ne viens point armé d’un indigne artifice&
       D’un voile d’équité couvrir mon injustice.&
       Il suffit, que mon cœur me
 condamne tout bas,&
       Et je soûtiendrois mal ce que je ne croy pas.&
       J’épouse une Troyenne. Oüy, Madame, \ampersand\ j’avouë&
       Que je vous ay promis la foy,
 que je luy vouë.&
       Un autre vous diroit, que dans
 les champs Troyens&
       Nos deux Peres sans nous
 formerent ces liens,&
       Et que sans consulter ny mon cœur ny le vostre,&
       Nous fusmes sans amour engagez l’un à l’autre.&
       Mais c’est assez pour moy que je me sois soûmis.&
       Par mes Ambassadeurs mon cœur vous fut promis.&
       Loin de les revoquer, je voulus y souscrire.&
       Je vous vis avec eux arriver en Epire.&
       Et quoy que d’un autre œil
 l’éclat victorieux&
       Eust déja préuenu le pouvoir de vos yeux;&
       Je ne m’arrestay point à cette ardeur nouvelle.&
       Je voulus m’obstiner à vous estre fidelle.&
       Je vous receûs en Reine, \ampersand\
 iusques à ce jour,&
       J’ay cru que mes sermens me tiendroient lieu d’a-mour.&
       Mais cét amour l’emporte. Et par un coup funeste,&
       Andromaque
 m’arrache un cœur qu’elle déteste.&
       L’un par l’autre entraisnez, nous courons à l’Autel&
       Nous jurer, malgré nous, un
 amour immortel.&
       Apres cela, Madame, éclatez contre un Traistre,&
       Qui l’est avec
 douleur, \ampersand\ qui pourtant veut l’estre.&
       Pour moy, loin de contraindre un
 si iuste cour-roux,&
       Il me soulagera peut-estre autant que vous.&
       Donnez-moy tous les noms destinez aux Parjures.&
       Je crains vostre silence, \ampersand\ non pas vos injures,&
       Et mon Cœur soulevant mille secrets tesmoins,&
       M’en dira d’autant plus que vous m’en direz
 moins.\&
       
\stanza[
\enonciateur{HERMIONNE.}
]
                
                Seigneur, dans cét aveu despoüillé d’artifice,&
       J’aime à voir que du moins vous
 vous rendiez ju-stice,&
       Et que voulant bien rompre un
 nœud si solennel,&
       Vous vous abandonniez au crime en criminel.&
       Est-il juste
 apres tout, qu’un Conquerant s’abaisse&
       Sous la seruile loy de garder sa promesse?&
       Non, non, la Perfidie a dequoyde quoy vous tenter.&
       Et vous ne me cherchez que pour vous en
 vanter.&
       Quoy? Sans que ny serment, ny
 devoir vous re-tienne,&
       Rechercher une Grecque, Amant
 d’une Troyenne?&
       Me quitter, me reprendre, \ampersand\ retourner
 encor&
       De la Fille d’Helene, à la
 Veuve d’\edgls{Hector}?&
       Couronner tour à tour l’Esclave, \ampersand\ la Princesse,&
       Immoler Troye aux Grecs, au Fils d’\edgls{Hector} la Grece?&
       Tout cela part d’un cœur tousiours maistre de soy,&
       D’un Heros qui n’est point Esclave de sa foy.&
       Pour plaire à vostre Espouse, il vous faudroit peut-estre&
       Prodiguer les doux noms de Parjure, \ampersand\ de
 Traistre.&
       Vostre grand cœur sans doute attend apres mes pleurs,&
       Pour aller dans ses bras joüir
 de mes douleurs?&
       Chargé de tant d’honneur il veut qu’on le
 renuoye?&
       Mais, Seigneur, en un iour ce seroit trop de joye.&
       Et sans chercher ailleurs des
 titres empruntez,&
       Ne vous suffit-il pas de ceux
 que vous portez?&
       Du vieux Pere d’\edgls{Hector} la valeur abbattuë&
       Aux pieds de sa Famille
 expirante à sa veuë,&
       Tandis que dans son sein vostre bras enfoncé&
       Cherche un reste de sang que l’âge avoit
 glacé;&
       Dans des ruisseaux de sang Troye ardante plongée,&
       De vostre propre main Polyxene esgorgée&
       Aux yeux de tous les Grecs indignez contre
 vous,&
       Que peut-on refuser à ces
 genereux coups?\&
       
\stanza[
\enonciateur{PYRRHVS.}
]
                
                Madame, je sçay
 trop, à quel excez de rage&
       L’ardeur de vous vanger emporta mon courage.&
       Je puis me plaindre à vous du sang que j’ay versé.&
       Mais enfin je consens d’oublier le passé.&
       Je rends graces au Ciel, que
 vostre indifference&
       De mes heureux soûpirs m’aprenne
 l’innocence.&
       Mon cœur, je le voy bien, trop
 prompt à se gesner,&
       Devoit mieux vous connoistre, \ampersand\ mieux s’examiner.&
       Mes remords vous faisoient une injure mortelle,&
       Il faut se croire aimé, pour se croire infidelle.&
       Vous ne prétendiez point m’arrester dans vos fers.&
       J’ay craint de vous trahir,
 peut-estre je vous sers.&
       Nos Cœurs n’estoient point faits
 dépendans l’un de l’autre.&
       Je suivois mon
 devoir, \ampersand\ vous cediez au vostre.&
       Rien ne vous engageoit à m’aimer en effet.\&
       
\stanza[
\enonciateur{HERMIONNE.}
]
                
                Je ne t'ay point aimé, Cruel?
 Qu'ay-je donc fait?&
       J’ay desdaigné
 pour toy les vœux de tous nos Princes,&
       Je t’ay cherché moy-mesme au fond de tes Pro-uinces.&
       J’y suis encor,
 malgré tes infidelitez,&
       Et malgré tous mes Grecs honteux de mes
 bontez.&
       Je leur ay commandé de cacher
 mon injure,&
       J’attendois en secret le retour d’un Parjure,&
       J’ay creu que tost ou tard à ton devoir rendu,&
       Tu me rapporterois un Cœur qui
 m’estoit dû.&
       Je t’aimois inconstant, qu’aurois-je fait fidelle?&
       Et mesme en ce moment, où ta
 bouche cruelle&
       Vient si tranquillement
 m’annoncer le trespas,&
       Ingrat, je doute encor, si je ne t’aime pas.&
       Mais, Seigneur, s’il le faut, si
 le Ciel en colere&
       Reserue à d’autres yeux la
 gloire de vous plaire,&
       Achevez vostre
 hymen, j’y consens. Mais du moins&
       Ne forcez pas mes yeux d’en estre les tesmoins.&
       Pour la derniere fois je vous
 parle peut-estre ,&
       Differez-le d’un iour, demain
 vous serez maistre.&
       Vous ne respondez point.
 Perfide, je le voy,&
       Tu contes les momens que tu perds avec moy.&
       Ton cœur impatient de revoir sa Troyenne,&
       Ne souffre qu’à regret qu’un autre t’entretienne,&
       Tu luy parles du cœur, tu la cherches des
 yeux.&
       Je ne te retiens plus, sauve-toy de ces lieux.&
       Va luy jurer la foy, que tu m’avois jurée.&
       Va profaner des Dieux la Majesté
 sacrée.&
       Ces Dieux, ces justes Dieux
 n’auront pas ou-blié,&
       Que les mesmes sermens avec moy t’ont lié.&
       Porte aux pieds des Autels ce Cœur qui m’aban-donne.&
       Va, cours. Mais crains encor d’y trouver Her-mionne.\&
       
                      
\stanza[\section{SCENE VI.}
\personscene{PYRRHVS, PHOENIX.}
\enonciateur{PHOENIX.}
                ]
                
                SEigneur, vous
 l’entendez. Gardez de negliger&
       Une Amante en fureur, qui
 cherche à se van-ger.&
       Elle n’est en ces lieux que trop
 bien appuyée,&
       La querelle des Grecs à la sienne est liée.&
       Oreste l’aime encore. Et peut-estre à ce
 prix…\&
       
\stanza[
\enonciateur{PHOENIXPYRRHVS.}
]
                
                Andromaque m’attend. Phœnix, garde son Fils.\&
       \stanza[\chapter{ACTE V.}
\section{SCENE PREMIERE.} 
    \personscene{}  
    \enonciateur{HERMIONNE} 
    ]
    
    O v suis-je? Qu’ay-je fait? Que dois-je
 faire encore?&
       Quel transport me saisit? Quel cha-grin me devore?&
       Errante, \ampersand\ sans dessein, je cours dans
  ce Palais.&
       Ah! ne puis-je sçavoir si j’aime, ou si je
 hais!&
       Le Cruel! De quel œil il m’a congédiée?&
       Sans pitié, sans douleur, au
 moins estudiée.&
       Ay-je veû ses regards se troubler un moment?&
       En ay-je pû tirer un seul gemissement?&
       Muet à mes soupirs, tranquille à
 mes allarmes,&
       Sembloit-il seulement qu’il eust part à mes larmes?&
       Et je le plains encore? Et pour
 comble d’ennuy&
       Mon cœur, mon lasche cœur
 s’interesse pour luy?&
       Je tremble au seul penser du coup qui le menasse?&
       Et preste à me vanger je luy fais déja grace?&
       Non, ne revoquons point l’arrest de mon courroux.&
       Qu’il perisse.
 Aussi bien il ne vit plus pour
 nous.&
       Le Perfide triomphe, \ampersand\ se
 rit de ma rage.&
       Il pense voir en pleurs dissiper cét orage.&
       Il croit que tousiours foible,
 \ampersand\ d’un cœur incer-tain,&
       Je pareray d’un
 bras les coups de l’autre main.&
       Il juge encor de moy par mes bontez passées.&
       Mais plûtost le Perfide a bien
 d’autres pensées.&
       Triomphant dans le Temple, il ne s’informe
 pas&
       Si l’on souhaite ailleurs sa vie, ou son trespas.&
       Il me laisse,
 l’Ingrat! cét embarras funeste.&
       Non, non, encore un coup, laissons agir Oreste.&
       Qu’il meure, puis qu’enfin il a dû le prevoir,&
       Et puis qu’il m’a forcée enfin à le vouloir.&
       A le vouloir? Hé quoy? C’est
 donc moy qui l’or-donne?&
       Sa Mort sera l’effet de l’amour
 d’Hermionne?&
       Ce Prince, dont mon cœur se
 faisoit autrefois,&
       Auec tant de plaisir, redire les
 Exploits,&
       A qui mesme en secret je m’estois destinée,&
       Auant qu’on eust conclu ce fatal
 hymenée,&
       Je n’ay donc traversé tant de mers, tant d’Estats,&
       Que pour venir si loin preparer
 son trespas,&
       L’assassiner, le perdre? Ah devant qu’il expire…\&
       
                      
\stanza[\section{SCENE II.}
\personscene{HERMIONNE, CLEONNE.}
\enonciateur{HERMIONNE.}
                ]
                
                AH! qu’ay-je fait,
 Cleonne? Et que viens-tu me
 dire?&
       Que fait Pyrrhus?\&
       
\stanza[
\enonciateur{CLEONNE.}
]
                
                \antilabe Il est au comble de ses vœux,&
       Le plus fier des Mortels, \ampersand\ le plus
 amoureux.&
       Je l’ay veû vers le Temple, où
 son hymen s’apreste,&
       Mener en Conquerant sa nouvelle Conqueste,&
       Et d’un œil qui déja devoroit son espoir,&
       S’enyurer, en marchant, du plaisir de la voir.&
       Andromaque, au travers de mille cris de
 joye,&
       Porte iusqu’aux Autels le souvenir de Troye,&
       Incapable tousiours d’aimer,
 \ampersand\ de haïr,&
       Sans joye, \ampersand\ sans murmure
 elle semble obeïr.\&
       
\stanza[
\enonciateur{HERMIONNE.}
]
                
                Et l’Ingrat? Iusqu’au bout il a
 poussé l’outrage?&
       Mais as-tu bien, Cleonne, obserué son
 visage?&
       Gouste-t’il des plaisirs tranquilles \ampersand\ parfaits?&
       N’a-t’il point détourné ses yeux
 vers le Palais?&
       Dy-moy, ne t’es-tu point presentée à sa veuë?&
       L’Ingrat a-t’il rougy, lors qu’il t’a
 reconnuë?&
       Son trouble avoüoit-il son infidelité?&
       A-t’il iusqu’à la fin soûtenu sa fierté?\&
       
\stanza[
\enonciateur{CLEONNE.}
]
                
                Madame, il ne voit rien. Son salut, \ampersand\ sa gloire&
       Semble estre avec vous sortis de sa memoire.&
       Sans songer qui le suit, Ennemis, ou Sujets,&
       Il poursuit seulement ses amoureux projets.&
       Autour du Fils d’\edgls{Hector} il a rangé sa Garde,&
       Et croit que c’est luy seul que le peril regarde.&
       Phœnix mesme en respond, qui l’a conduit
 exprés&
       Dans un Fort éloigné du Temple,
 \ampersand\ du Palais.&
       Voila, dans ses transports, le
 seul soin qui luy reste.\&
       
\stanza[
\enonciateur{HERMIONNE.}
]
                
                Le Perfide! Il mourra. Mais que t’a dit Oreste?\&
       
\stanza[
\enonciateur{CLEONNE.}
]
                
                Oreste, avec ses Grecs,
 dans le Temple est entré.\&
       
\stanza[
\enonciateur{HERMIONNE.}
]
                
                Hé bien? A me vanger n’est-il
 pas preparé?\&
       
\stanza[
\enonciateur{CLEONNE.}
]
                
                Je ne sçay.\&
       
\stanza[
\enonciateur{HERMIONNE.}
]
                
                \antilabe Tu ne sçais? Quoy donc Oreste encore,&
       Oreste me trahit?\&
       
\stanza[
\enonciateur{CLEONNE.}
]
                
                \antilabe Oreste vous adore.&
       Mais de mille remords son esprit combattu&
       Croit tantost son amour, \ampersand\ tantost sa
 vertu.&
       Il respecte en Pyrrhus l’honneur du diadéme.&
       Il respecte en Pyrrhus\edgls{Achille}, \ampersand\ Pyrrhus mesme.&
       Il craint les Grecs, il craint l’Univers en courroux.&
       Mais il se craint, dit-il, soy-mesme plus que tous.&
       Il voudroit en Vainqueur vous apporter sa teste.&
       Le seul nom d’Assassin
 l’épouvante \ampersand\ l’arreste.&
       Enfin il est entré, sans sçavoir dans son cœur,&
       S’il en devoit sortir Coupable, ou Spectateur.\&
       
\stanza[
\enonciateur{HERMIONNE.}
]
                
                Non, non, il les verra triompher sans obstacle,&
       Il se gardera bien de troubler
 ce spectacle.&
       Je sçay de
 quels remords son courage est
 atteint.&
       Le lasche craint la mort, \ampersand\
 c’est tout ce qu’il craint.&
       Quoy? sans qu’elle employast une seule
 priere,&
       Ma Mere en sa faveur arma la Grece entiere?&
       Ses yeux pour leur querelle, en dix ans de
 com-bats,&
       Virent perir vingt Rois, qu’ils ne connoissoient pas?&
       Et moy je ne prétens que la mort d’un Parjure,&
       Et je charge un
 Amant du soin de mon injure,&
       Il peut me conquerir à ce prix, sans danger,&
       Je me liure moy-mesme, \ampersand\ ne puis me vanger?&
       Allons. C’est à moy seule, à me rendre justice.&
       Que de cris de douleur le Temple retentisse.&
       De leur hymen fatal troublons l’evenement,&
       Et qu’ils ne soient unis, s’il se peut, qu’un
 moment.&
       Je ne choisiray point dans ce
 desordre extréme.&
       Tout me sera Pyrrhus, fust-ce Oreste luy-mesme.&
       Je mourray. Mais au moins ma
 mort me vangera,&
       Je ne mourray pas seule, \ampersand\ quelqu’un me suiura.\&
       
                      
\stanza[\section{SCENE III.}
\personscene{ORESTE, ANDROMAQVE, HERMIONNE, CLEONNE, CEPHISE, soldats
 d'ORESTE.}
\enonciateur{ORESTE.}
                ]
                
                MAdame, c’en est fait. Partons en diligence.&
       Venez dans mes vaisseaux goûter vostre vangeance.&
       Voyez cette Captive. Elle peut
 mieux que moy&
       Vous apprendre qu’Oreste a dégagé sa foy.\&
       
\stanza[
\enonciateur{HERMIONNE.}
]
                
                O Dieux! C’est
 Andromaque?\&
       
\stanza[
\enonciateur{ANDROMAQVE.}
]
                
                \antilabe Oüy, c’est cette Princesse&
       Deux fois Veuve, \ampersand\ deux fois
 l’Esclave de la Grece;&
       Mais qui jusque dans Sparte ira vous braver
 tous,&
       Puis qu’elle voit son Fils à
 couvert de vos coups.&
       Du crime de Pyrrhus complice manifeste,&
       J’attens son
 chastiment. Car je voy bien qu’Oreste&
       Engagé par vostre ordre à cét
 assassinat,&
       Vient de ce triste exploit vous
 ceder tout l’éclat.&
       Je ne m’attendois pas que le
 Ciel en colere&
       Pust, sans
 perdre mon Fils, accroistre ma misere,&
       Et gardast à mes yeux quelque spectacle encor,&
       Qui fist couler mes pleurs pour
 un autre qu’\edgls{Hector}.&
       Vous avez trouvé seule une sanglante voye&
       De suspendre en
 mon cœur le souvenir de Troye.&
       Plus barbare aujourd’huy qu’\edgls{Achille} \ampersand\ que son
 Fils,&
       Vous me faites pleurer mes plus grands
 Ennemis;&
       Et ce que n’avoient pû promesse, ny menasse,&
       Pyrrhus de
 mon \edgls{Hector}semble avoir pris la place.&
       Je n’ay que trop, Madame,
 éprouvé son courroux,&
       J’avois plus de
 sujet de m’en plaindre que vous.&
       Pour derniere rigueur, ton amitié cruelle,&
       Pyrrhus, à
 mon Epoux me rendoit infidelle.&
       Je t’en allois punir. Mais le
 Ciel m’est témoin,&
       Que je ne poussois pas ma vangeance si loin,&
       Et sans verser
 ton sang, ny causer tant
 d’allarmes,&
       Il ne t’en eust cousté peut-estre que des larmes.\&
       
\stanza[
\enonciateur{HERMIONNE.}
]
                
                Quoy? Pyrrhus est donc mort!\&
       
\stanza[
\enonciateur{ORESTE.}
]
                
                \antilabe Oüy, nos Grecs irritez&
       Ont laué dans son sang ses infidelitez.&
       Je vous l’avois
 promis. Et quoy que mon courage&
       Se fist de ce complot une funeste image,&
       J’ay couru vers le Temple, où
 nos Grecs dispersez&
       Se sont jusqu’à
 l’Autel dans la foule glissez.&
       Pyrrhus m’a
 reconnu. Mais sans changer de face,&
       Il sembloit que ma veuë excitast son audace,&
       Que tous les Grecs bravez en
 leur Ambassadeur&
       Deussent de son Hymen relever la splendeur.&
       Enfin avec transport prenant son diadéme,&
       Sur le front d’Andromaque il l’a posé luy-mesme.&
       Je vous donne, a-t’il dit, ma
 Couronne, \ampersand\ ma Foy;&
       Andromaque, regnez sur l’Epire \ampersand\ sur moy.&
       Je voüe à vostre Fils une amitié de Pere,&
       J’en atteste
 les Dieux, je le jure à sa Mere.&
       Pour tous mes Ennemis je declare les siens,&
       Et je le reconnoy pour le Roy des Troyens.&
       A ces mots, qui du Peuple attiroient le suffrage,&
       Nos Grecs n’ont répondu que par un cry de rage,&
       L’Infidelle s’est vû par tout
 enuelopper,&
       Et je n’ay pû trouver de place
 pour frapper.&
       Chacun se disputoit la gloire de l’abbatre.&
       Je l’ay veû dans leurs mains
 quelque temps se de-battre,&
       Tout sanglant à leurs coups
 vouloir se dérober,&
       Mais enfin à l’Autel il est allé
 tomber.&
       Le Troyen est sauvé. Mais partons, le temps presse.&
       L’Epire tost ou tard satisfera la Grece:&
       Cependant j’ay voulu qu’Andromaque aujour-d’huy&
       Honnorast mon triomphe, \ampersand\
 répondit de luy.&
       Du Peuple épouvanté la foule
 fugitive&
       M’a laissé sans obstacle enlever ma Captive,&
       Et regagner ces lieux, où bien-tost nos Amis&
       Viendront couverts du sang, que je vous ay promis.\&
       
\stanza[
\enonciateur{HERMIONNE.}
]
                
                Qu’ont-ils fait!\&
       
\stanza[
\enonciateur{ORESTE.}
]
                
                \antilabe Pardonnez à leur impatience.&
       Ils ont, je le voy bien, trahy vostre vangeance.&
       Vous vouliez que ma main portast
 les premiers coups,&
       Qu’il sentist
 en mourant qu’il expiroit pour vous.&
       Mais c’est moy, dont l’ardeur
 leur a seruy d’exem-ple,&
       Je les ay pour vous seule entraînez dans le Tem-ple,&
       Madame, \ampersand\ vous pouvez justement vous flatter&
       D’une mort, que les Grecs n’ont
 fait qu’executer.&
       Vous seule avez poussé les
 coups.....\&
       
\stanza[
\enonciateur{HERMIONNE.}
]
                
                \antilabe Tay-toy, Perfide,&
       Et n’impute qu’à toy ton lasche
 Parricide.&
       Va faire chez tes Grecs admirer ta fureur,&
       Va, je la desavouë, \ampersand\ tu me fais horreur.&
       Barbare qu’as-tu fait? avec
 quelle furie&
       As-tu tranché le cours d’une si belle vie?&
       Auez-vous pû, Cruels, l’immoler aujourd’huy,&
       Sans que tout vostre sang se soûlevast pour luy?&
       Mais parle. De son sort qui t’a
 rendu l’arbitre?&
       Pourquoy l’assassiner? Qu’a-t'il fait? A
 quel titre?&
       Qui te l’a dit?\&
       
\stanza[
\enonciateur{ORESTE.}
]
                
                \antilabe O dieux! Quoy ne m’avez-vous pas&
       Vous-mesme, icy, tantost, ordonné son trépas?\&
       
\stanza[
\enonciateur{HERMIONNE.}
]
                
                Ah! Falloit-il en croire une
 Amante insensée?&
       Ne devois-tu pas lire au fonds
 de ma pensée?&
       Et ne voyois-tu pas dans mes emportemens,&
       Que mon cœur démentoit ma bouche à tous mo-mens?&
       Quand je l’aurois voulu, falloit-il y souscrire?&
       N’as-tu pas dû cent fois te le faire redire,&
       Toy-mesme avant
 le coup me venir consulter,&
       Y revenir encore, ou plûtost m’éuiter?&
       Que ne me laissois-tu le soin de ma vangeance?&
       Qui t’ameine en des lieux, où l’on fuit ta presence?&
       Voila de ton amour le detestable
 fruit.&
       Tu m’apportois, Cruel, le malheur qui te suit.&
       C’est toy, dont l’Ambassade à tous les deux fatale,&
       L’a fait pour son malheur
 pancher vers ma Rivale;&
       Nous le verrions encor nous partager ses soins,&
       Il m’aimeroit peut-estre , il le
 feindroit du moins.&
       Adieu. Tu peux partir. Je
 demeure en Epire,&
       Je renonce à la Grece, à
 Sparte, à son
 Empire,&
       A toute ma Famille. Et c’est assez pour moy,&
       Traistre, qu’elle ait produit un monstre comme toy.&
       Allons, Madame, allons. C’est
 moy qui vous dé-liure.&
       Pyrrhus ainsi l’ordonne, \ampersand\ vous pouvez me suiure.&
       De nos derniers devoirs allons
 nous dégager.&
       Montrons qui de nous deux sçaura
 mieux le van-ger.\&
       
                      
\stanza[\section{SCENE IV.}
\personscene{ORESTE, Soldats d'Oreſte.}
\enonciateur{ORESTE.}
                ]
                
                QVe voy-je? Est-ce Hermionne? Et que
 viens-je d’entendre?&
       Pour qui coule le sang que je viens de répandre?&
       Je suis, si je l’en crois, un
 Traistre, un Assassin.&
       Est-ce Pyrrhus qui meurt? \ampersand\ suis-je
 Oreste enfin?&
       Quoy? j’étouffe en mon cœur la raison qui m’é-claire.&
       J’assassine à regret un Roy que je revere.&
       Je viole en un
 jour les droits des Souverains,&
       Ceux des Ambassadeurs, \ampersand\ tous ceux des Humains,&
       Ceux mesme des Autels, où ma
 fureur l’assiege.&
       Je deviens
 Parricide, Assassin, Sacrilege.&
       Pour qui? Pour une Ingrate, à
 qui je le promets,&
       Qui mesme, s’il ne meurt, ne me
 verra jamais,&
       Dont j’épouse la rage. Et quand
 je l’ay seruie,&
       Elle me redemande \ampersand\ son sang \ampersand\ sa vie!&
       Elle l’aime! \ampersand\ je suis un monstre furieux!&
       Je la voy pour jamais s’éloigner
 de mes yeux,&
       Et l’Ingrate, en fuyant, me laisse pour salaire&
       Tous les noms odieux que j’ay pris pour luyplaireluy plaire.\&
       
                      
\stanza[\section{SCENE V.}
\personscene{ORESTE, PYLADE, Soldats d'Oreſte.}
\enonciateur{PYLADE.}
                ]
                
                IL faut partir, Seigneur.
 Sortons de ce Palais,&
       Ou bien resoluons nous de n’en
 sortir jamais.&
       Nos Grecs pour un moment en
 défendent la Porte.&
       Tout le Peuple assemblé nous poursuit à main for-te.&
       Aux ordres d’Andromaque icy tout est soûmis,&
       Ils la traitent en Reyne, \ampersand\ nous comme
 Ennemis.&
       Andromaque
 elle mesme à Pyrrhussi rebelle,&
       Luy rend tous les devoirs d’une Veuve fidelle,&
       Commande qu’on le vange. Et peut-estre qu’en-cor,&
       Elle poursuit sur nous la vangeance d’\edgls{Hector}.&
       Allons. N’attendons pas que l’on nous enuiron-ne,&
       Nos Grecs nous rejoindront. Et tandis
 qu’Her-mionne&
       Tient encore le Peuple autour d’elle arresté,&
       Voila nostre chemin, sortons en seureté.\&
       
\stanza[
\enonciateur{ORESTE.}
]
                
                Non, non, c’est Hermionne, Amis, que je
 veux suiure,&
       A son dernier courroux, je ne puis plus suruiure:&
       Partez, j’ay fait le crime, \ampersand\ je vais l’expier.\&
       
\stanza[
\enonciateur{PYLADE.}
]
                
                Hermionne,
 Seigneur? Il la faut oublier.&
       Quoy tousiours l’instrument, \ampersand\ l’objet de sa ra-ge,&
       Cherchez-vous chez les Morts quelque nouvel ou-trage?&
       Et parce qu’elle meurt, faut-il que vous
 mouriez?\&
       
\stanza[
\enonciateur{ORESTE.}
]
                
                Elle meurt? Dieux. Qu’entens-je!\&
       
\stanza[
\enonciateur{PYLADE.}
]
                
                \antilabe Hé quoy? vous l’ignoriez?&
       En rentrant dans ces lieux nous l’avons rencon-trée,&
       Qui couroit vers le Temple, inquiete, égarée.&
       Elle a trouvé Pyrrhus, porté sur des
 Soldats,&
       Que son sang
 excitoit à vanger son trépas.&
       Sans doute, à cét objet sa rage
 s’est émuë.&
       Mais du haut de la porte enfin nous l’avons veuë,&
       Un poignard à la main, sur Pyrrhusse courber,&
       Lever les yeux au Ciel, se frapper, \ampersand\ tomber.\&
       
\stanza[
\enonciateur{ORESTE.}
]
                
                Grace aux Dieux! Mon mal-heur passe mon esperance.&
       Oüy, je te louë, ô Ciel! de ta
 perseverance.&
       Appliqué sans relasche au soin de me punir,&
       Au comble des douleurs tu m’as fait parue-nir.&
       Ta haine a pris plaisir à former ma misere,&
       J’estois né,
 pour seruir d’exemple à ta colere,&
       Pour estre du Mal-heur un modelle accomply;&
       Hé bien, je meurs content, \ampersand\
 mon sort est rem-ply.&
       Où sont ces deux Amans? Pour
 couronner ma joye,&
       Dans leur sang, dans le mien, il
 faut que je me noye:&
       L’un \ampersand\ l’autre en mourant,
 je les veux regarder.&
       Réünissons
 trois cœurs, qui n’ont pû s’accorder,&
       Mais quelle épaisse nuit tout à coup m’enuironne?&
       De quel costé sortir? D’où vient que je frissonne?&
       Quelle horeur me saisit? Grace au Ciel, j’entrevoy.&
       Dieux! Quels ruisseaux de sang coulent autour de moy!\&
       
\stanza[
\enonciateur{PYLADE.}
]
                
                Ah! Seigneur!\&
       
\stanza[
\enonciateur{ORESTE.}
]
                
                \antilabe Quoy, Pyrrhus, je te rencontre encore?&
       Trouveray-je par tout un Rival que j’abhorre?&
       Percé de tant de coups comment t’es-tu sauvé?&
       Tiens, tiens, voila le coup que je t’ay reserué.&
       Mais que vois-je? A mes yeux Hermionne l’em-brasse?&
       Elle vient l’arracher au coup qui le menasse?&
       Dieux, quels affreux regards elle jette sur moy!&
       Quels Démons, quels serpens
 traisne-t’elle apres soy?&
       Hé bien, Filles d’Enfer, vos mains sont-elles pre-stes?&
       Pour qui sont ces Serpens qui sifflent sur vos testes?&
       A qui destinez-vous l’appareil
 qui vous suit?&
       Venez-vous m’enlever dans
 l’eternelle Nuit?&
       Venez, à vos fureurs Oreste s’abandonne.&
       Mais non, retirez-vous, laissez faire Hermionne;&
       L’Ingrate mieux que vous sçaura
 me déchirer,&
       Et je luy porte enfin mon cœur à
 devorer.\&
       
\stanza[
\enonciateur{PYLADE.}
]
                
                Il perd le sentiment. Amis, le  temps nous presse,&
       Ménageons les momens que ce transport nous laisse.&
       Sauvons-le. Nos efforts deviendroient impuissans,&
       S’il reprenoit icy sa rage avec ses sens.\&
       \endnumbering
\end{Rightside}
\end{pages}
\Pages
\printglossaries
\end{document}