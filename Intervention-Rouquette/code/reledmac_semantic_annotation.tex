\newcommand{\variant}[3]{\edtext{#1}{\Afootnote{#2: #3}}}
\newcommand{\om}[2]{\variant{#1}{#2}{\emph{om.}}}
\newcommand{\add}[3]{\variant{#1}{#2}{\emph{add.} #3}}


\beginnumbering
\pstart
Le petit \variant{chat}{A}{chien} est \variant{mort}{B}{décédé}.
Il est tombé du toit.
Pourquoi est-ce \om{toujours}{C} un petit chat qui meurt et jamais un pape qui tombe du  \add{toit}{AD}{dans la rue} ?
\pend
\endnumbering
